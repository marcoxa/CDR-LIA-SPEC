\documentclass[../../CDR-IEEE-754-support.tex]{subfiles}
\begin{document}
The ANSI \CL{} Specification \cite{1996:ANSIHyperSpec} hints at possible
compliance with IEEE ``Floating Points'' (in retrospect, \IEEEFPStd{}) in
the description of the \code{*features*} variable, where:
\begin{description}
\item[\code{:ieee-floating-point}]
  If present, indicates that the
  implementation \emph{purports to conform}  to the requirements of
  \emph{IEEE Standard for Binary Floating-Point Arithmetic}. 
\end{description}

After the publication of the \CL{} standard \cite{1994:ANSICL}, other
standards were published by IEEE and ISO/IEC specifying the
Application Programming Interface (API) for, by then well established,
commonly used arithmetic and mathematical operations.  These
specifications are the \emph{IEEE Standard for Floating-Point
  Arithmetic} (\IEEEFPStd{}) \cite{2008:IEEE-754} specification (which
will be referred as \IEEEFPStd{} in this text) and the \emph{ISO/IEC
  Information technology -- Language independent arithmetic} \IECLIA{}
\cite{2012:LIA1,2001:LIA2,2004:LIA3} (in three parts, which will be
referred as LIA1, LIA2 and LIA3, or simply LIA, in this text).

\vspace*{3mm}

\noindent
The LIA specifications also contain Annexes specifying \emph{language
  bindings}, including \CL{} (cfr., LIA1, Annex~D).  They also specify
how a language definition should handle \emph{exceptional situations}
(cfr., LIA1, Section~6), which has consequences for \CL{} described
below.

\vspace*{3mm}

\noindent
Alas, the actual state of ``compliance'' with these specifications,
among different \CL{} implementations varies wildly, especially the
LIA ones, with some implementations publicizing the
\code{:ieee-floating-point} feature while providing an unclear subset
of functionalities of features.
% \footnote{We speak of the sin and not of the sinner; you can check for
%   yourself what happens in the various implementations.}.
This is in contrast with the current state of affairs in other
languages and language ecosystems; especially languages designed and
built \emph{after} the publication of the \IEEEFPStd{} and LIA
documents.

\vspace*{3mm}

% The goal of this document is to address some of this state of affairs
% by providing a specification for a minimalistic subset of
% names,\marginnote{Not very ``minimalistic'' anymore\ldots}
% functions and macros to be used in conjunction with \IEEEFPStd{}
% concepts.

The goal of this document is to put forth a rationale and a \ANSICL{}
styled specification for \CL{} that provided the facilities described
in the \IECLIA{} LIA specification.
%
The following issues will especially 
be addressed:
\begin{itemize}
\item \code{NaN}s and ``infinities''.

\item Functions and macros to handle \emph{rounding modes}; with
  special care devoted to incentivate non-invasive, yet informative,
  implementations.

\item Functions and macros to handle with the underlying
  \emph{floating point environment} and the interplay between
  \IEEEFPStd{} notion of \emph{signaling} vs.~the \CL{} notion of
  \clterm{condition signalling}.  I.e., providing a clear set of
  functionalities to handle \emph{exceptional situations} and a clear
  definition of the behavior of each operator and function with
  respect the LIA specifications.
  
\item Clarifying the behavior of the \CL{} set of mathematical
  functions (cfr.,~Section~12.2 of \cite{1996:ANSIHyperSpec}) with respect
  to the specification contained herein.
\end{itemize}

\vspace*{3mm}

The content of this document borrows ideas from the \CLang{}
specification \cite{2018:C18}, and the documentation of
several \CL{} implementations.

\paragraph{Note.} In the following, the examples shown try to be
specific for a given implementation; that is, whenever a behavior is
referring to a specific implementation (even when illustrating a
generic behavior), the ``prompt'' will refer to that implementation --
\code{LW>}, \code{SBCL>}, etc. etc.


\subsection{Impact on Current \CL{} Implementations}
\subfile{subsections/1-Impacts-On-Implementations.tex}

\end{document}