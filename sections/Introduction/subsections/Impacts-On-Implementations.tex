%%%% -*- Mode: LaTeX -*-

%%%% Impacts-On-Implementations.tex

\documentclass[../Introduction.tex]{subfiles}

\begin{document}

It is understood that different \CL{} implementations already have
made some assumptions about the \emph{floating point environment} they
are dealing with.  In particular, the implementation of several \CL{}
math functions may be sensitive to the
setting of \emph{rounding modes}.  E.g., it is quite possible that all
\CL{} implementations quietly assume a \emph{round to nearest} mode in
the implementation of the \CL{} standard floating point operations.
Another example is the treatment of comparison operators with respect
to NaNs, as depicted by the example below\footnote{There are ways to
  tell SBCL how to construct a NaN.}.

\vspace*{3mm}

\noindent
SBCL signals a \code{CL:FLOATING-POINT-INEXACT} condition on a
comparison involving \emph{quiet} NaNs.
\begin{alltt}
SBCL> \codeprompt{(< 42 quiet-nan)}
debugger invoked on a FLOATING-POINT-INEXACT:
  arithmetic error FLOATING-POINT-INEXACT signaled

Type HELP for debugger help, or (SB-EXT:EXIT) to exit from SBCL.

restarts (invokable by number or by possibly-abbreviated name):
  0: [ABORT] Exit debugger, returning to top level.

(SB-KERNEL:TWO-ARG-< 42 #<DOUBLE-FLOAT quiet NaN>)
SBCL 0]
\end{alltt}

\vspace*{3mm}

\noindent
Lispworks just returns \code{NIL}.
\begin{alltt}
LW> \codeprompt{(< 42 1D+-0)} \textcolor{red}{; LW uses this notation to denote NaNs.}
\textit{NIL}
\end{alltt}

\vspace*{3mm}

\noindent
CCL signals a \code{CL:FLOATING-POINT-INVALID-OPERATION} condition on a
comparison involving NaNs; but in this case it is unknown whether
\code{1D+-0} represents a \emph{quiet} or \emph{signaling} NaN.
\begin{alltt}
CL> \codeprompt{(< 42 1D+-0)} \textcolor{red}{; CCL uses a notation similar to LW.}
> Error: FLOATING-POINT-INVALID-OPERATION detected
> While executing: CCL::FIXNUM-DFLOAT-COMPARE, in process Listener(4).
> Type cmd-. to abort, cmd-\ for a list of available restarts.
> Type :? for other options.
CCL 1 > 
\end{alltt}

As a consequence, this specification takes care not to impose
constraints on programs already running, or to require implementations to
actually adapt their cores to make the \CL{} package math functions in
order to comply.

Therefore, in order to be self-contained, this specification provides
facilities (described below) that allow a \CL{} programmer to actually
decide how to use LIA compliant code.  Most important, and
possibly most annoying, this specification must define \emph{symbols}
that corresponds to (some of) the operations described in
\cite{2008:IEEE-754}, in order to provide the programmer with some
certainty about the behavior of floating point operations.

All in all, this specification is grafting onto the \CL{}
specification some functionality which was fully defined after the
publication of \cite{1996:ANSIHyperSpec}.

\end{document}