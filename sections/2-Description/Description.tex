\documentclass[../../CDR-IEEE-754-support.tex]{subfiles}
\begin{document}
The \IEEEFPStd{} and LIA standards introduce a number of ``names'' for
certain special values: \code{NaN}s and infinities, as an example.
Several \CL{} implementations provide such concepts; alas, not in a
consensual way.  Moreover, the LIA standards introduce ways to handle
\emph{exceptional situations}. At the time of this writing, the \CL{}
programmer does not have much control over \emph{how} to exploit the
richness of the LIA specifications.

% The goal of this document is to provide a minimal
% consensus to give the \CL{} programmer ways to be able to work with
% facilities in line with the \IEEEFPStd{} (\IECFPStd{}) and LIA
% standards.  The specifications contained in this document are as
% \textsf{common-lisp}-ish as possible, re-using as much the style and
% naming conventions established in the ANSI~\CL{} Specification
% \cite{1994:ANSICL,1996:ANSIHyperSpec}.

\noindent
As anticipated, the main issues to be clarified are the following:
\begin{itemize}
\item Infinities, \textsf{NaN}s and other floating point related
  issues.
\item Rounding issues.
\item Error handling and \emph{notification} styles.
\end{itemize}
%
A few more issues to be added to this list, regard the ``programming
ecosystem'' in which the \CL{} LIA specification will live; among
these issues are \emph{environment introspection} facilities and
\emph{conditional use/control} of every part of the \CL{}
specification. See below for some examples.


\paragraph{\CLLIAPKG{} Package.}
\label{sect:package}
%
The implementations of this specification will provide a package named
(or nicknamed) \CLLIAPKG{}.  All the symbols named in the rest
of the document are \code{export}ed from the above mentioned package.

\undecided{Features or APIs?}{Should \emph{features} or \emph{APIs} be
  provided to allow selective checks of this specification?}

\paragraph{Semi-available Floating Point Numbers Facilities.}
%
This document contains a small number of simple facilities that appear
to be present in most \CL{} implementations or that are available as
libraries.  E.g., this document describes a \code{make-float} function
and a \code{parse-float} function, which has been available as a
separate library for quite some time in the community\footnote{The
  \code{parse-float} function and ancillary ones can be downloaded
  from \Quicklisp{} \cite{2008:Beane:Quicklisp}.}.

\paragraph{Naming Conventions.}
%
The LIA specifications suggest a naming convention for its
functionalities that reuses much of \CL{} names.  some of the choices
are not particularly in line with \CL{} style.  Two examples are the
functions \code{sqrtUp} and \code{sqrtDwn}, which compute square roots
with ``up'' or ``down'' rounding modes; \CL{} style would have avoided
the ``camel case'', given that \CL{} implementations are uppercasing
out-of-the box, while preferring an hyphenated naming.  Accordingly,
this specification will use \code{sqrt-upward} and
\code{sqrt-downward} (and the more succinct \code{sqrt.<} and \code{sqrt.>}).

% Another issue with the LIA suggested naming is that it essentially
% requires an implementation to provide a set of very basic
% LIA-compliant functions -- e.g., \code{+}, \code{*}, \code{1-},
% \code{sin}, etc. -- which implies a reworking of an implementation
% core.



\subsection{Infinities, \code{NaN}s and Other Special Values}

\subfile{subsections/1-Special-Values/Special-Values.tex}

\subsection{Rounding Modes}

\subfile{subsections/2-Rounding-Modes/Rounding-Modes.tex}

\subsection{Notifications and Exception Handling}

\subfile{subsections/3-Notifications-Exceptions-Handling/Notifications-Exceptions-Handling.tex}

\end{document}