%%%% -*- Mode: LaTeX -*-

%%%% Rounding-Modes.tex

\documentclass[../Description.tex]{subfiles}
\begin{document}

One of the functionalities described in \IEEEFPStd{} (\IECFPStd{}) and LIA is
the control of \emph{rounding modes}.  Rounding modes control is
important in some numerical applications and libraries.  E.g., they
are necessary to build \emph{interval arithmetic} libraries (cfr., \cite{hickey:interval:2001,kulisch:complete:2009,revol:introIEEEIA:2017}).

The \CLang{} standard interface for floating point rounding modes
control (cfr., \cite{2018:C18} Section~7.6.3) provides the \code{fgetround}
and \code{fsetround} functions that can get and set the rounding mode;
the rounding modes being defined as \code{FE\_DOWNWARD},
\code{FE\_TONEAREST},\\
\code{FE\_TOWARDZERO} and \code {FE\_UPWARD} (which are \CLang{}
macros).  In order to change the rounding mode of (floating point)
operations, a \CLang{} programmer invokes the \code{fsetround}
function manually establishing a new state of the processing
machinery.

Within \CL{}, a programmer expects a number of facilities to simplify
coding.  To this end, apart from the expected definition of a
\code{rounding-mode} type defined as
\begin{alltt}
(member :indeterminable
        :zero
        :nearest
        :positive-infinity
        :negative-infinity)
\end{alltt}
and of the corresponding \code{get-rounding-mode} and
\code{set-rounding-mode}, this document also defines the macros
\code{with-rounding-mode}, \code{round-to-zero}, \code{round-to-near},
\code{round-upward}, and \code{round-downward}.  Their intent use is
to localize and automate the establishing of a given rounding mode for
a piece of code.

\vspace*{3mm}

\noindent
As an example, consider the following code snippet:
\begin{alltt}
CL prompt> \codeprompt{(round-upward (* 2 21.0))}
\textit{42.0}
\end{alltt}
In this case the intent of the programmer is to ensure that the
rounding mode in effect while executing the multiplication is
\emph{toward positive infinity}.  Upon returning its value, the
\code{round-upward} macro, the rounding mode is reset to the value
before its invocation.

\subsubsection{Default Rounding Mode}

\subfile{subsubsections/Default.tex}

\subsubsection{Interaction with the \CL{} Reader}

\subfile{subsubsections/Interactions-With-Reader.tex}

\end{document}