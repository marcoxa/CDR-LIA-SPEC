%%%% -*- Mode: LaTeX -*-

%%%% Notifications-Exception-Handling.tex

\documentclass[../Description.tex]{subfiles}
\begin{document}

\label{sect:notifications}

The floating point and complex arithmetic exceptional situation
handling machinery described in
\cite{2012:LIA1,2001:LIA2,2004:LIA3,2008:IEEE-754}, more specifically,
\cite{2012:LIA1} Section~6, \emph{Notification}.  Three main
notification modalities are described in LIA1:
\begin{itemize}
\item \emph{Notification by recording in indicators} (NRI -- LIA1,
  Section~6.2.1).
\item \emph{Notification by alteration of control flow} (NACF -- LIA1,
  Section~6.2.2).
\item \emph{Notification by termination with message} (NTM -- LIA1,
  Section~6.2.3).
\end{itemize}

\noindent
These notification machineries balance three viewpoints that different
engineers have.

\begin{itemize}
\item The hardware instruction set designer\footnote{Just a label; it
    is used for the purpose of illustration.}.
\item The programming language designer.
\item The specification implementor.
\end{itemize}

The \CL{} ANSI Specification provides the following conditions that
may be raised by an implementation (in an \emph{implementation
  dependent} way) in conjunction with floating point operations.
\begin{description}
\item \code{floating-point-invalid-operation},
\item \code{floating-point-inexact},
\item \code{floating-point-overflow},
\item \code{floating-point-underflow}.
\end{description}
Of course, the condition \code{division-by-zero} may also be signaled
by floating point operations.

The LIA1 specification, Annex~D, proposes that \CL{} defined the
arithmetic exception handling using the Condition System, i.e., using
the NACF notification approach.  This document specifies that
\textbf{both} alternatives NRI and NACF are actually available to the
programmer\footnote{\CMUCL{} and \SBCL{} already provide
  both modes, although in a non fully documented way. In fact:

\begin{alltt}
CMUCL> \codeprompt{(/ 42 0.0)}

Arithmetic error DIVISION-BY-ZERO signaled.
Operation was /, operands (42.0 0.0).
   [Condition of type DIVISION-BY-ZERO]

Restarts:
  0: [ABORT] Return to Top-Level.

Debug  (type H for help)
\end{alltt}
  The above is the behavior implied by LIA1 for \CL{}.  However,
  \CMUCL{} (and \SBCL{}) provide also direct manipulation of the
  underlying \emph{floating point environment}.  In that case, it
  appears that \CMUCL{} and \SBCL{} do offer the NRI machinery.
\begin{alltt}
SBCL> \codeprompt{(sb-int:with-float-traps-masked (:divide-by-zero) (/ 3 0.0))}
\textit{\#.SB-EXT:SINGLE-FLOAT-POSITIVE-INFINITY}

SBCL> \codeprompt{(sb-int:get-floating-point-modes)}\\
\textit{(:TRAPS (:OVERFLOW :INVALID :DIVIDE-BY-ZERO)\\
  :ROUNDING-MODE :NEAREST\\
  :CURRENT-EXCEPTION NIL\\
  :ACCRUED-EXCEPTION NIL\\
  :FAST-MODE NIL\\
  :PRECISION :53-BIT)}
\end{alltt}
In the case above, the division by zero is not raised or recorded and
the expected ``continuation value'' is returned (in this case the
\code{short-float} $\infty$).
}.
This causes issues in the interplay between the NRI and the NACF
exception handling methods, but it gives the programmer all the fine
control it is needed to handle many potential numerical
issues. Incidentally, note that LIA1 provides an example about how
\Fortran{} may provide some compiler directives to choose between NRI
and NTM (cfr., LIA1, Annex E).
\begin{alltt}
!LIA\$  NOTIFICATION=RECORDING
!LIA\$  NOTIFICATION=TERMINATE
\end{alltt}

In order to select and introspect what kind of exception handling
regime is selected in a given computation\footnote{In this
  specification, no mention of \emph{threading model} is made;
  however, it is assumed that an implementation can make all the
  dynamical behavior of numerical computations ``thread safe''.}, this
specification provides the following API.
\begin{itemize}
\item An enumerated type for NRI, NACF and NTM: %\\
  \code{arithmetic-notification-style}.
  \begin{alltt}
    (deftype \codelia{arithmetic-notification-style} ()
      '(member :recording    \textcolor{red}{; I.e., NRI.}
               :error        \textcolor{red}{; I.e., NACF.}
               :termination  \textcolor{red}{; I.e., NTM.}
               ))
  \end{alltt}
\item A set of functions and macros to retrieve and set the current
  notification style.
  \begin{description}
  \item \code{current-notification-style}: a function that retrieves
    the, as the name implies, the current notification style.
  \item \code{set-notification-style}, \code{(setf current-notification-style)}: functions that set the
    notification style.
  \item \code{with-notification-style}: a macro that temporarily sets
    the notification style and ensures to restore the one in effect
    before its invocation.
  \end{description}
\item A ``mirroring'' in package \CLLIAPKG{} of the ANSI CL floating
  point exceptions in order to accommodate the notion of
  \emph{continuation values}.  This will amount to have the mirrored
  ANSI CL floating point exceptions have a \code{continuation-value}
  slot with a associated initarg and reader (of the same name).
\end{itemize}

\vspace*{2mm}

\noindent
Finally, the entries in Section~\ref{sect:fpe-dictionary} specify an interface
similar the \CLang{} Library interface to the \emph{Floating Point
  Environment} provided by \verb|<fenv.h>| \cite{2018:C18}.


\subsubsection{Interaction Between the ``Low Level'' \IEEEFPStd{} and LIA
  ``Signaling'' Machinery and \CL{} Condition Handling}

\subfile{subsubsections/Interactions-IEEE-LIA-Lisp.tex}

\subsubsection{Specialized Handling of LIA-related Exceptions}

\subfile{subsubsections/Specialized-Handling.tex}

\end{document}