%%%% -*- Mode: LaTeX -*-

%%%% Interactions-IEEE-LIA-Lisp.tex

\documentclass[../Notifications-Exception-Handling.tex]{subfiles}
\begin{document}


The \IEEEFPStd{} and LIA specifications (cfr., Section~7 of
\cite{2008:IEEE-754} and Sections~4.1.3, 4.1.4 and~6 of
\cite{2012:LIA1}) appear to imply that the actual handling of
``exceptional'' situations should follow the steps below.  Given an
operation $f(x, \ldots)$.

\begin{enumerate}
\item Check the arguments of $f$ for special cases regarding
  \clieeeterm{NaN}s and \clieeeterm{infinities}.
\item Decide whether a \clieeeterm{IEEE exception} should be
  \emph{signaled}.
  \begin{enumerate}
  \item Handle the exception \emph{by default}.
  \item Raise the flags to indicate what exception was signaled (and
    handled by default).
  \end{enumerate}
\end{enumerate}

The specifications also assume that the \emph{notification} of an
\emph{exceptional situation} should either be recorded ``somewhere''
(in the \CLang{} specification, Section~7.6, in an object of type
\code{fenv\_t}) and that ``catastrophic'' events should ensue from HW
traps and signals.

The LIA specifications indicate that a language \emph{may} provide
alternative modes of ``handling'' such exceptions, acknowledging the
presence of ``exception handling'' machinery in most modern languages
\missingpart{Reference to specs ``Alternate Exception Handling'' --
  LIA1, Section~6, Annex~D};
and \CL{} is not an exception, if not for the much richer set of
features that it provides with its ``Condition System''.  As a matter
of fact, LIA1 indicates that \CL{} should define such an ``alternate
exception handling'' based on the standard Condition System.

In the following, this specification will state exactly what kind of
behavior the various functions and macros will follow.  In general, a
\CL{} \emph{condition} will be \emph{signaled} when either the \CL{}
standard already mandates so (e.g., in the case of
\code{cl:division-by-zero}) or when the LIA specification implies that
the ``continuation value'' and the recording of the exception proposed
are really a stopgap measure\footnote{This may be seen as somewhat
  arbitrary.  This interpretation is warranted, as the main goal of
  the present document is to narrow down as much as possible
  alternative behaviors -- read: \emph{implementation dependent}
  ones. \textbf{This footnote may be inconsistent with LIA1, Annex~D,
    w.r.t., \CL{}.}}.


\end{document}