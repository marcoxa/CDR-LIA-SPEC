%%%% -*- Mode: LaTeX -*-

%%%% Specialized-Handling.tex

\documentclass[../Notifications-Exception-Handling.tex]{subfiles}

\begin{document}

This document specifies a simple macro, built on top of the standard
\CL{} condition handling machinery to quickly handle LIA-related
exceptions, that is, the standard \CL{}\\
\code{floating-point-invalid-operation},
\code{floating-point-inexact},
\code{floating-point-overflow}, and
\code{floating-point-underflow} conditions.  This macro provides for
a simplified \clieeeterm{alternative exception handling}
\checkref{Alternative Exception Handling}.

The macro is named \code{trap-math} and has the following
simple syntax:
\begin{alltt}
  (trap-math (&key \textit{notify-by} \textit{before} \textit{after}) <\textit{expr}> <\textit{handler}>* )
\end{alltt}
In the simplest case \code{trap-math} is a no-op, with
respect to the behavior of the math operations in \code{<\textit{expr}>}.
\begin{alltt}
CL prompt> \codeprompt{(trap-math () (* 2 21.0))}
\textit{42.0}
\end{alltt}
\ldots but:
\begin{alltt}
LW prompt> \codeprompt{(trap-math () (/ 40 0.0))}

Error: Division-by-zero caused by / of (40.0 0.0).
  1 (continue) Return a value to use.
  2 Supply new arguments to use.
  3 (abort) Return to top loop level 0.

Type :b for backtrace or :c <option number> to proceed.
Type :bug-form "<subject>" for a bug report template or :? for other options.
\end{alltt}
On the other hand, the following example produces the LIA and \CLang{} result
(assuming that\\
\code{*read-default-float-format*} was \code{single-float}).
\begin{alltt}
CL prompt> \codeprompt{(trap-math () (/ 40 0.0)
               (division-by-zero () :continue))}
\textit{INFS}
\color{red}{;;; ... or \textit{1S++0}, or \textit{SINGLE-FLOAT-POSITIVE-INFINITY}.}
\end{alltt}
What the macro wants to convey is the fact that when
\code{division-by-zero} is caught (thus invoking the
\code{<\textit{handler}>} above), we can simply \emph{continue} with
the \emph{continuation value} that is associated to the operation that
actually signaled the condition; note that this may be different from
the usual \CL{} behavior -- \code{division-by-zero} a case in point.

The macro \code{trap-math} also makes provisions to set a local \emph{notification
  style} (cfr., the keyword argument \textit{\code{style}} as per the
macro \code{with-notification-style}.

Other ``actions'' available, besides \code{:continue}, are
\begin{itemize}
\item\code{:default} the behavior of the operation that signaled
  the condition is the default one. In the example above, the
  \code{division-by-zero} condition is re-signaled, i.e., passed
  on, but for other operations it may be different; in most cases this
  will be a no-op with respect to the specification.
    

\item\code{:clear} the low level exceptions stored in the
  \emph{floating point environment} are cleared\footnote{Cfr.,
    the \CLang{} standard function \code{feclearexcept},
    Section~7.6.2.1 of \cite{2018:C18}.}.

\item\code{(:continue <\textit{expr}>)} the value(s) returned by
  \code{\textit{expr}} are used as \emph{continuation value(s)}.
\end{itemize}
The \code{:default} and the \code{:continue} forms (also the
\code{:continue} with ``values'') are mutually exclusive.  The effect
of \code{:clear} is described hereafter. See the full documentation of
the macro \code{trap-math} for more details.

\paragraph{LIA Exceptions in the Floating Point Environment.} Whenever
an exception ``happens'', the corresponding flag is \emph{raised}
(i.e., stored in the \emph{floating point environment} and remains
available for further analysis by the program; at least when using the
\CLang{} library, which has been the inspiration for this
specification.  In \CL{} some operations do signal a
\clterm{condition} which may or may not be handled by the program.
The choice of this specification is to state that the signaling of a
\CL{} condition, say \code{cl:division-by-zero} does also raise the
corresponding \clieeeterm{floating point exception flag}.  Moreover,
unless some special machinery is used (cfr., the
\code{trap-math} macro) the handling of a \CL{} condition
(say, \code{cl:floating-point-invalid-operation}) by
\code{handler-case} or similar facilities \emph{does not clear} the
floating point environment flags (unless done explicitly by the
handler code).

\missingpart{Should I have an argument for ``before'' and
  ``after'' operations in \code{trap-math}?  Like
  \code{:restore-on-exit} \code{:clear-on-exit} \code{clear-on-entry}
  \code{preserve-on-entry}}

\noindent
As an example consider the following code snippet.
\begin{alltt}
(defun boom (x)
    (declare (type single-float x))
    (handler-case 
        (/ x 0)
      (division-by-zero (e)
        (format *error-output* "Boom ~S.~%" e)))
\end{alltt}
After this function is run, the result will be \code{NIL} \emph{and} the
floating point environment will still have the \code{divZero} flag raised.
That is, the following interaction should be expected:
\begin{alltt}
CL prompt> \codeprompt{(boom 42.0s0)}

Boom #<Condition DIVISION-BY-ZERO>
\textit{NIL}

CL prompt> \codeprompt{(fpe-test-notifications :division-by-zero)}
\textit{T}
\end{alltt}


\end{document}