\documentclass[../Float-Infinities-Nan.tex]{subfiles}
\begin{document}

\DDictionaryItem{Function \code{make-float}}
\index{M!\code{make-float}}

\DSyntax{}

\code{make-float} \varname{bytes} \code{\&optional}
\varname{float-type}
$\Rightarrow$ \varname{float}

\DArgsNValues{}

\varname{bytes} -- An integer or bit-vector representing the binary
pattern of a floating point number.\\
\varname{float-type} -- A recognizable subtype of \code{float},
defaulting to \code{*read-default-float-format*}.\\
\varname{result} -- the resulting floating point number or \code{NAN}.


\DDescription{}

The function constructs a floating point number of appropriate
\varname{float-type}, starting from the bit content of
\varname{bytes}.

If \varname{bytes} corresponds to the byte pattern of a \emph{NaN},
then a \code{NAN} is returned, regardless of \varname{float-type}.

\DExceptional{}

The function signals a \code{type-error} if either \varname{bytes} or
\varname{float-type} are not as described above.

\DSeeAlso{}

\code{NAN}.

\DNotes{}

This function is, in one form or another, already present in \CL{}
implementations.

\end{document}