\documentclass[../Floating-Point-Environment.tex]{subfiles}
\begin{document}

\DDictionaryItem{Function \code{set-floating-point-environment}}
\index{S!\code{set-floating-point-environment}}

\DSyntax{}

\begin{tabbing}
\code{set-floating-point-environment} \= \code{\&key}\\
\>\varname{traps}\\
\>\varname{rounding-mode}\\
\>\varname{current-notifications}\\
\>\varname{precision}\\
\>\code{\&allow-other-keys}\\
$\Rightarrow$ \varname{modes}
\end{tabbing}


\DArgsNValues{}

\varname{traps} -- A list of the exception conditions that should cause
traps.\\
\varname{rounding-mode} -- The rounding mode to use when the result is
not exact.\\
\varname{current-notifications} -- The argument is used to set the current
set of exceptions.\\
\varname{precision} -- An integer.\\
\varname{modes} -- An a-list containing the current floating
point modes; the indicators are keywords.

\DDescription{}

This function sets options controlling the floating-point
hardware. If a keyword is not supplied, then the current value is
preserved.

The possible values for each of the keywords are the
following.

\begin{itemize}
\item \varname{traps} is a list that can contain the keywords
  \code{:underflow}, \code{:overflow}, \code{:inexact}, \code{:invalid},
  \code{:divide-by-zero}, and \code{:denormalized-operand}.

\item \varname{rounding-mode} is the rounding mode to use when the result is
  not exact; it can assume the values \code{:nearest},
  \code{:positive-infinity}, \code{:negative-infinity} and
  \code{:zero}.

\item \varname{current-notifications} is used to set the exception flags. The
  main use is setting the accrued exceptions to \code{NIL} to clear
  them.

\item \varname{precision} can be one of the integers 24, 53 and 64, standing for
  the internal precision of the mantissa.
\end{itemize}

\DExamples{}

None.


\DNotes{}

None.


\DExceptional{}

The function can always result in a no-op if access to the underlying
hardware is not fully supported.  When this happens
\code{set-floating-point-environment} must issue a warning.


\DSeeAlso{}

\code{get-floating-point-environment}.

\end{document}