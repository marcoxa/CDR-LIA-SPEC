\documentclass[../Floating-Point-Environment.tex]{subfiles}
\begin{document}

\DDictionaryItem{Function \code{default-floating-point-environment}}
\index{D!\code{default-floating-point-environment}}

\DSyntax{}

\code{default-floating-point-environment} \textit{$<$no arguments$>$}
$\Rightarrow$ \varname{fpe}

\DArgsNValues{}

\varname{fpe} -- An object of type \code{floating-point-environment}.

\DDescription{}

The function returns an object of type
\code{floating-point-environment} that represents the \emph{default}
floating point environment in use by the implementation.

% The format of the returned \varname{fpe} a-list is such to be usable
% as an \code{apply} last argument for
% \code{set-floating-point-environment}.

\DExamples{}

\begin{alltt}
SBCL> \codeprompt{(default-floating-point-environment)}
\textit{(:TRAPS (:OVERFLOW :INVALID :DIVIDE-BY-ZERO)
 :ROUNDING-MODE :NEAREST
 :CURRENT-NOTIFICATIONS (:INEXACT)
 :FAST-MODE NIL
 :PRECISION :53-BIT)}
\textcolor{red}{;;; The format of the result for SBCL is incidental.
;;; The fpe-* readers can be made to work with such representation.}
\end{alltt}

\DNotes{}

The function \code{default-floating-point-environment} should always return
the same (or \code{equalp}) value.

\DSeeAlso{}

\code{set-floating-point-environment}, \code{get-floating-point-environment}.

\end{document}