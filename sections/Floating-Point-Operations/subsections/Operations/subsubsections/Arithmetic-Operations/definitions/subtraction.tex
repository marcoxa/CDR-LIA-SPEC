\documentclass[../Arithmetic-Operations.tex]{subfiles}
\begin{document}

\DDictionaryItem{Function \code{-}}
\index{*!\code{-}}

\DSyntax{}

\code{-} \varname{a} \varname{b} \RArrow \varname{result}\\
\code{-} \varname{a} \code{\&rest} \varname{ns} \RArrow \varname \code{n}\\

\DArgsNValues{}

\varname{a}, \varname{b}, \varname{result} -- Floating Point numbers.\\
\varname{numbers} -- A, possibly empty, list of numbers.\\
\varname{n} -- A number.

\DDescription{}

The dyadic version of the function \code{-}, when operating on
floating point numbers (or numbers that eventually are converted to
floating point numbers according to \CL{} float contagion rules -- cfr.,
Section~12.1.4 of \cite{1996:ANSIHyperSpec}) assume the behavior of the
underlying \IEEEFPStd{} specification \cite{2008:IEEE-754}.  It is assumed that
the multiple argument versions are eventually built upon the dyadic
ones (the one argument version can be seen as a special case from
the point of view of this specification).

The dyadic version of the functions \code{-} behaves
according to the \IEEEFPStd{} operations described in Section~5.4.1 of
\cite{2008:IEEE-754}, which is assumed to be the usual behavior specified
for \CL{} by \cite{1996:ANSIHyperSpec} when \varname{a} and \varname{b} are
not \emph{NaN}s or \emph{infinities}.

\noindent
The operation from \cite{2008:IEEE-754} is:

\vspace*{3mm}

\noindent
\textit{formatOf}-\textbf{subtraction}(\varname{a}, \varname{b})

\vspace*{3mm}

\noindent
where \textit{formatOf} describes the resulting floating point
format.  As already mentioned, the actual floating point format of
\varname{result} is dictated by the \CL{} standard.

When \code{-} is called with either \varname{a} or \varname{b} being a
\emph{quiet NaN}, and neither is a \emph{signaling NaN} then
\varname{result} is a (quiet) \code{NAN}.  No error (floating point
exception) is signaled in this case.

When \code{-} is called with \varname{a} being an 
\clieeeterm{infinity} and \varname{b} being a finite \clterm{number}
(or vice-versa), then \varname{result} is \varname{a}; when \code{-} is
called with \varname{b} being an \clieeeterm{infinity} and \varname{a}
being a finite \clterm{number}, then \varname{result}
is $-$\varname{b}.


\DExceptional{}

There are different exceptional situations to be considered.

\begin{enumerate}
\item When \code{-} is called with either \varname{a} or \varname{b}
  being a \emph{signaling NaN}, then the\\
  \clname{cl:floating-point-invalid-operation} error is signaled.

\item When \code{-} is called with \varname{a} being a
  \clieeeterm{positive infinity} and \varname{b} being a
  \clieeeterm{positive infinity}, or \varname{a} being a
  \clieeeterm{negative infinity} and \varname{b} being a
  \clieeeterm{positive infinity}, then the\\
  \clname{cl:floating-point-invalid-operation} error is signaled.

\item If some of \varname{a}, \varname{b}, or any element of a non-empty
  \varname{numbers} is not a \CL{} \clterm{number} then the function
  might signal a \clname{cl:type-error}.
\end{enumerate}

\end{document}