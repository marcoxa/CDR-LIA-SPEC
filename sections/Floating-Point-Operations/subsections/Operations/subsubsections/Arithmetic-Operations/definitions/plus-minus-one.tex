\documentclass[../Arithmetic-Operations.tex]{subfiles}
\begin{document}

\DDictionaryItem{Functions \code{1+}, \code{1-}}
\index{*!\code{1+}}
\index{*!\code{1-}}

\DSyntax{}

\code{1+} \varname{n} \RArrow ~ \varname{result}\\
\code{1-} \varname{n} \RArrow ~ \varname{result}\\

\DArgsNValues{}

\varname{n} -- A \clterm{number}.\\
\varname{result} -- A \clterm{number}.


\DDescription{}

The functions \code{1+} and \code{1-} are defined, 
as the \CL{} counterparts, as:
\begin{enumerate}
\item \code{(1+}
\varname{n}\code{)} $\equiv$ \code{(+}
\varname{n}\code{ 1)}.
\item \code{(1-}
\varname{n}\code{)} $\equiv$ \code{(-}
\varname{n}\code{ 1)}.
\end{enumerate}

The behaviour and the exceptional situations of \code{1+} and
\code{1-} are inherited by \code{+} and \code{-} described above.

When \varname{n} is not a \clieeeterm{NaNs} or an
\clieeeterm{infinity} the functions \code{1+} and \code{1-}
assume the usual behavior specified for \CL{} (cfr.
Section~12.2 of \cite{1996:ANSIHyperSpec}).

When \code{1+} or \code{1-} is called with \varname{n} being a
\clieeeterm{quiet NaN} then \varname{result} is a \clieeeterm{quiet NaN}.

When \code{1+} or \code{1-} is called with \varname{n} being an
\clieeeterm{infinity} then \varname{result}  is an
\clieeeterm{infinity} equal to \varname{n}.

\DExceptional{}

There are different exceptional situations to be considered.

\begin{enumerate}
\item When \code{1+} or \code{1-} is called with \varname{n}
  being a \emph{signaling NaN}, then the\\
  \clname{cl:floating-point-invalid-operation} error is signaled.

\item If \varname{n} is not a \CL{} \clterm{number} then the function
  signals a \clname{cl:type-error}.
\end{enumerate}

\DSeeAlso{}

\code{+}, \code{-}.

\end{document}