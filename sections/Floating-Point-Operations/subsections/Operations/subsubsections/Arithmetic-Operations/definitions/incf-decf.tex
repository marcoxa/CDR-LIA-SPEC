\documentclass[../Arithmetic-Operations.tex]{subfiles}
\begin{document}

\DDictionaryItem{Macros \code{incf}, \code{decf}}
\index{I!\code{incf}}
\index{D!\code{decf}}

\DSyntax{}

\code{incf} \varname{place} \code{\&optional} \varname{d} \RArrow ~ \varname{result}\\
\code{decf} \varname{place} \code{\&optional} \varname{d} \RArrow ~ \varname{result}

\DArgsNValues{}

\varname{d} -- A \clterm{number}, the default is \code{1}.\\
\varname{result} -- A \clterm{number}.\\
\varname{place} -- A \CL{} \clterm{place}.\\


\DDescription{}

The macros \code{incf} and \code{decf} are defined,
as the \CL{} counterparts, as:
\begin{itemize}
\item \code{(incf} \varname{x}\code{)} $\equiv$ \code{(setf}
  \varname{x} \code{(+} \varname{x}\code{ 1))}
\item \code{(decf} \varname{x}\code{)} $\equiv$ \code{(setf}
  \varname{x} \code{(-} \varname{x}\code{ 1))}
\item \code{(incf} \varname{x}\code{ d)} $\equiv$ \code{(setf}
  \varname{x} \code{(+} \varname{x}\code{ d))}
\item \code{(decf} \varname{x}\code{ d)} $\equiv$ \code{(setf}
  \varname{x} \code{(-} \varname{x}\code{ d))}
\end{itemize}

The behaviour and the exceptional situations of the macros \code{incf} and
\code{incf} are inherited by \code{+} and \code{-} described above.

When the value of \varname{place} and \varname{d} are not
\clieeeterm{NaNs} or \clieeeterm{infinity} the macros \code{incf} and
\code{decf} assume the usual behavior specified for \CL{} (cfr.
Section~12.2 of \cite{1996:ANSIHyperSpec}).

When \code{incf} or \code{decf} is called with either the value of
\varname{place} or \varname{d} being a \clieeeterm{quiet NaN}, and
neither is a \clieeeterm{signaling NaN} then \varname{result} is a
(quiet) \code{NAN}. No error (floating point exception) is signaled in
this case.

When \code{incf} or \code{decf} is called with the value of
\varname{place} being \clieeeterm{infinity}  and \varname{d} being a
finite \clterm{number} (or vice-versa), then \varname{result} is the
value of \varname{place} (or, vice-versa, \varname{d}).

\DExceptional{}

There are different exceptional situations to be considered.

\begin{enumerate}
\item When \code{incf} or \code{decf} is called with either the value of
  \varname{place} or \varname{d} being a \emph{signaling NaN}, then the
  \clname{cl:floating-point-invalid-operation} error is signaled.

\item When \code{incf} or \code{decf} is called with the value of
  \varname{place} being a \clieeeterm{positive infinity} and \varname{d}
  being a \clieeeterm{negative infinity} (or vice-versa), then the
  \clname{cl:floating-point-invalid-operation} error is signaled.

\item If the value of \varname{place} or \varname{d} is not a \CL{}
  \clterm{number} then the macros \code{incf} and \code{decf}
  might signal a \clname{cl:type-error}.
\end{enumerate}

\DNotes{}

The default for \varname{d} is coerced to the appropriate floating
point \code{1.0}, depending on the floating point format of
\varname{place}.

\DSeeAlso{}

\code{+}, \code{-}, \code{1+}, \code{1-}.

\end{document}