\documentclass[../Arithmetic-Operations.tex]{subfiles}
\begin{document}

\DDictionaryItem{Functions \code{+}, \code{*}}
\index{*!\code{+}}
\index{*!\code{*}}

\DSyntax{}

\code{+} \varname{a} \varname{b} \RArrow \varname{result}\\
\code{*} \varname{a} \varname{b} \RArrow \varname{result}\\
\code{+} \code{\&rest} \varname{ns} \RArrow \varname \code{n}\\
\code{*} \code{\&rest} \varname{ns} \RArrow \varname \code{n}\\

\DArgsNValues{}

\varname{a}, \varname{b}, \varname{result} -- Floating Point numbers.\\
\varname{numbers} -- A, possibly empty, list of numbers.\\
\varname{n} -- A number.



\DDescription{}

The dyadic versions of the functions \code{+}, and \code{*}, when operating on
floating point numbers (or numbers that eventually are converted to
floating point numbers according to \CL{} float contagion rules -- cfr.,
Section~12.1.4 of \cite{1996:ANSIHyperSpec}) assume the behavior of the
underlying \IEEEFPStd{} specification \cite{2008:IEEE-754}.  It is assumed that
the multiple argument versions are eventually built upon the dyadic
ones (zero and one argument versions can be seen as special cases from
the point of view of this specification).

The dyadic versions of the functions \code{+}, and \code{*} behave
according to the \IEEEFPStd{} operations described in Section~5.4.1 of
\cite{2008:IEEE-754}, which is assumed to be the usual behavior specified
for \CL{} by \cite{1996:ANSIHyperSpec} when \varname{a} and \varname{b} are
not \emph{NaN}s or \emph{infinities}.

\noindent
The operations from \cite{2008:IEEE-754} are:

\vspace*{3mm}

\noindent
\textit{formatOf}-\textbf{addition}(\varname{a}, \varname{b})\\
\textit{formatOf}-\textbf{multiplication}(\varname{a}, \varname{b})

\vspace*{3mm}

\noindent
where \textit{formatOf} describes the resulting floating point
format.  As already mentioned, the actual floating point format of
\varname{result} is dictated by the \CL{} standard.

When \code{+}, or \code{*} is called with either \varname{a} or
\varname{b} being a \emph{quiet NaN}, and neither is a
\emph{signaling NaN} then \varname{result} is a (quiet) \code{NAN}.
No error (floating point exception) is signaled in this case.

When \code{+} is called with \varname{a} being an 
\clieeeterm{infinity} and \varname{b} being a finite \clterm{number}
(or vice-versa), then \varname{result} is \varname{a} (or, vice-versa, 
\varname{b}).

When \code{*} is called with \varname{a} being an
\clieeeterm{infinity} and \varname{b} being a non-zero finite
\clterm{number} (or vice-versa), then \varname{result} is \varname{a}
(or, vice-versa, \varname{b}).

\DExceptional{}

There are different exceptional situations to be considered.

\begin{enumerate}
\item When \code{+} is called with either \varname{a} or \varname{b}
  being a \emph{signaling NaN}, then the\\
  \clname{cl:floating-point-invalid-operation} error is signaled.

\item When \code{+} is called with \varname{a} being a
  \clieeeterm{positive infinity} and \varname{b} being a
  \clieeeterm{negative infinity} (or vice-versa), then the
  \clname{cl:floating-point-invalid-operation} error is signaled.

\item When \code{*} is called with \varname{a} being an
  \clieeeterm{infinity} and \varname{b} being a \clieeeterm{zero}
  value (or vice-versa), then the
  \clname{cl:floating-point-invalid-operation} error is signaled.

\item If some of \varname{a}, \varname{b}, or any element of a non-empty
  \varname{numbers} is not a \CL{} \clterm{number} then the function
  might signal a \clname{cl:type-error}.
\end{enumerate}

\end{document}