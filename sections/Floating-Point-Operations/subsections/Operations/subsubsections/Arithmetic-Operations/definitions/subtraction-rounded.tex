\documentclass[../Arithmetic-Operations.tex]{subfiles}
\begin{document}

\DDictionaryItem{Functions \code{-.<}, \code{-.>}}
\index{*!\code{-.<}}
\index{*!\code{-.>}}


\code{-.<} \varname{a} \varname{b} \RArrow \varname{result}\\
\code{-.>} \varname{a} \varname{b} \RArrow \varname{result}\\
\code{-.<} \code{\&rest} \varname{ns} \RArrow \varname \code{n}\\
\code{-.>} \code{\&rest} \varname{ns} \RArrow \varname \code{n}\\

\DArgsNValues{}

\varname{a}, \varname{b}, \varname{result} -- Floating Point numbers.\\
\varname{numbers} -- A, possibly empty, list of numbers.\\
\varname{n} -- A number.

\DDescription{}

The functions  \code{-.<} and \code{-.>} perform subtractions as per \code{-}
but they establish a \emph{downward} and, respectively a
\emph{upward} rounding mode before performing the operation.

\DNotes{}

LIA1 suggests to call these operations \code{<-} and \code{->}, but the symbols \code{<-}
and \code{->} get ``used up'', which may cause some readability issues
as they are surely used more often than not for some specialized
library.

\end{document}