%%%% -*- Mode: LaTeX -*-

%%%% Arithmetic-Operations.tex

\documentclass[../../Operations.tex]{subfiles}
\begin{document}

\label{sect:arith-ops}

\begin{table}[h!]
  \centering
  \begin{tt}
    \begin{tabular}{lll}
      * & 1+ & \ldots \\
      + & 1- & \ldots \\
      - & incf & conjugate\\
      / & decf & \\
    \end{tabular}
  \end{tt}
  \caption{The basic \CL{} arithmetic operations}
  \label{table:cl-arit-ops}
\end{table}

\noindent
Each of the functions in Table~\ref{table:cl-arit-ops} must be further
specified with respect to \cite{1996:ANSIHyperSpec} in order to adhere to
the requirements of \IEEEFPStd{}.  The descriptions and the references for
each function further specify the standard \CL{} behavior with respect
to \emph{NaN}s, \emph{infinities} and floating point exceptions.

\vspace*{3mm}

\noindent
Note that the \code{gcd}, and \code{lcm} functions are not
present in the above table, which corresponds to Figure~12-1 of
\cite{1996:ANSIHyperSpec}.

\subfile{definitions/addition-multiplication.tex}
\subfile{definitions/addition-rounded.tex}
\subfile{definitions/multiplication-rounded.tex}
\subfile{definitions/subtraction.tex}
\subfile{definitions/subtraction-rounded.tex}
\subfile{definitions/division.tex}
\subfile{definitions/division-rounded.tex}
\subfile{definitions/plus-minus-one.tex}
\subfile{definitions/incf-decf.tex}
\subfile{definitions/conjugate.tex}

\end{document}