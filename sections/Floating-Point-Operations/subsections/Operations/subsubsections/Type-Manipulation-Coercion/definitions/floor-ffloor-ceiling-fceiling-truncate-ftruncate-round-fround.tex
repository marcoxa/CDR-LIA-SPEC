%%%% -*- Mode: LaTeX -*-
%%%% floor-ffloor-ceiling-fceiling-truncate-ftruncate-round-fround.tex

\documentclass[../Exponentials-Logarithms-Trigonometry.tex]{subfiles}
\begin{document}

\DDictionaryItem{Functions \code{floor}, \code{ffloor}, \code{ceiling},
  \code{fceiling}, \code{truncate}, \code{ftruncate}, \code{round}, \code{fround}}
\index{A!\code{floor}}
\index{A!\code{ffloor}}
\index{A!\code{ceiling}}
\index{A!\code{fceiling}}
\index{A!\code{truncate}}
\index{A!\code{ftruncate}}
\index{A!\code{round}}
\index{A!\code{fround}}

\DSyntax{}

\code{floor} \varname{n} \code{\&optional} \varname{d} \RArrow
\varname{q}, \varname{r}\\
\code{ffloor} \varname{n} \code{\&optional} \varname{d} \RArrow
\varname{q}, \varname{r}\\
\code{ceiling} \varname{n} \code{\&optional} \varname{d} \RArrow
\varname{q}, \varname{r}\\
\code{fceiling} \varname{n} \code{\&optional} \varname{d} \RArrow
\varname{q}, \varname{r}\\
\code{truncate} \varname{n} \code{\&optional} \varname{d} \RArrow
\varname{q}, \varname{r}\\
\code{ftruncate} \varname{n} \code{\&optional} \varname{d} \RArrow
\varname{q}, \varname{r}\\
\code{round} \varname{n} \code{\&optional} \varname{d} \RArrow
\varname{q}, \varname{r}\\
\code{fround} \varname{n} \code{\&optional} \varname{d} \RArrow
\varname{q}, \varname{r}\\

\DArgsNValues{}

\varname{n},  -- A \clieeeterm{real}\\
\varname{d},  -- A \clieeeterm{real}, default $1$\\
\varname{1},  -- A \clieeeterm{integer} for \code{floor}, \code{ceiling},
\code{truncate}, \code{round}, a \clieeeterm{float} for \code{ffloor},
\code{fceiling}, \code{ftruncate}, \code{fround}\\
\varname{n},  -- A \clieeeterm{real}\\

\DDescription{}

The functions divide \varname{n} by \varname{d}. \varname{q} represents the
quotient as as integer with the rounding defined by
\cite{1996:ANSIHyperSpec}, while \varname{r} the remainder of the division. When
\varname{n}, \varname{d} are not \clieeeterm{NaNs} or \clieeeterm{infinities}
their behavior is the one described in \cite{1996:ANSIHyperSpec}.

\noindent
The \clieeeterm{rounding-modes} defined for each function, e.g. \code{.<}, are
redundant for \varname{q} but are relevant for \varname{r}.

\noindent
When \varname{n} or \varname{d} are
\clieeeterm{quiet NaN} the functions return a \clieeeterm{quiet NaN}.

\noindent
When \varname{n} is a \clieeeterm{positive-real}, including \clieeeterm{zero},
and not an \clieeeterm{infinity} and \varname{d} is a
\code{positive-infinity}\\
(\code{negative-infinity}) the result of the
functions is $0.0$ ($-0.0$).

\noindent
When \varname{n} is a \clieeeterm{negative-real},
and not an \clieeeterm{infinity} and \varname{d} is a
\code{positive-infinity} (\code{negative-infinity}) the result of the
functions is $-0.0$ ($0.0$).

\DExceptional{}

There are different exceptional situations to be considered:
\begin{enumerate}
\item If the functions are called with
  \varname{n} or \varname{d} being a
  \clieeeterm{signaling NaN}, then the\\
  \clname{cl:floating-point-invalid-operation} error is signaled.
\item If the functions are called with \varname{n} being an
  \clieeeterm{infinity}, then the a\\
  \clname{cl:floating-point-invalid-operation}  is signaled.
\item If the functions are called with \varname{d} being $0.0$ or $-0.0$, then
  the \\
  \clname{cl:floating-point-invalid-operation}  is signaled.
\item If \varname{n} or \varname{d} are not \CL{}
  \clterm{number} then the functions signals a \clname{cl:type-error}.
\end{enumerate}

\end{document}

%%% Local Variables:
%%% mode: latex
%%% TeX-master: "../../../../../../../CDR-IEEE-754-support"
%%% End:
