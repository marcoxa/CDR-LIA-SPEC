%%%% -*- Mode: LaTeX -*-
%%%% float.tex

\documentclass[../Type-Manipulation-Coercion.tex]{subfiles}
\begin{document}

\DDictionaryItem{Function \code{float}}
\index{A!\code{float}}

\DSyntax{}

\code{float} \varname{n}, \varname{p} \RArrow \varname{f}

\DArgsNValues{}

\varname{n} -- A \clieeeterm{real}\\
\varname{p} -- A \clieeeterm{float}\\
\varname{f} -- A \clieeeterm{float}\\

\DDescription{}

The function returns the float number \varname{f} with the same magnitude of
\varname{n} and format of \varname{p}. When
\varname{n}, \varname{p} are not \clieeeterm{NaNs} or \clieeeterm{infinities}
their behavior is the one described in \cite{1996:ANSIHyperSpec}.

\noindent
When \varname{n} or \varname{p} are
\clieeeterm{quiet NaN} the functions return a \clieeeterm{quiet NaN}.

\noindent
When \varname{n} is \code{positive-infinity} or
\code{negative-infinity} then the result of \code{float} is a float
with magnitude \code{positive-infinity} or\code{negative-infinity} and
format of \varname{p}.

\noindent
When \varname{p} is \code{positive-infinity} or
\code{negative-infinity} then the result of \code{float} is a float
with magnitude of \varname{n} and the format of the \clieeeterm{infinity}
represented with p.

\DExceptional{}

There are different exceptional situations to be considered:
\begin{enumerate}
\item If the functions are called with
  \varname{n} or \varname{p} being a
  \clieeeterm{signaling NaN}, then the\\
  \clname{cl:floating-point-invalid-operation} error is signaled.
\item If \varname{n} is not a \CL{} \clterm{real number} or \varname{p} is
  not \CL{} \clterm{float} then the functions signals a
  \clname{cl:type-error}.
\end{enumerate}

\end{document}