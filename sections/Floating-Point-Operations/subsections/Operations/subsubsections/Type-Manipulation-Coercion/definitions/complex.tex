%%%% -*- Mode: LaTeX -*-
%%%% complex.tex

\documentclass[../Type-Manipulation-Coercion.tex]{subfiles}
\begin{document}

\DDictionaryItem{Function \code{complex}}
\index{A!\code{complex}}

\DSyntax{}

\code{complex} \varname{r}, \varname{i} \RArrow
\varname{c}

\DArgsNValues{}

\varname{r} -- A \clieeeterm{real}\\
\varname{i} -- A \clieeeterm{real}\\
\varname{c} -- A \clieeeterm{complex}\\

\DDescription{}

The function returns the complex number \varname{c} with real part
\varname{r} and imaginary part \varname{i}. When
\varname{r}, \varname{i} are not \clieeeterm{NaNs} or \clieeeterm{infinities}
their behavior is the one described in \cite{1996:ANSIHyperSpec}.

\noindent
When \varname{r} or \varname{i} are
\clieeeterm{quiet NaN} the functions return a \clieeeterm{quiet NaN}.

\noindent
When \varname{r}, or \varname{i} or both are \code{positive-infinity} or
\code{negative-infinity} then the result of \code{complex} is a complex
number whose real or imaginary parts or both are \code{positive-infinity} or
\code{negative-infinity}.

\DExceptional{}

There are different exceptional situations to be considered:
\begin{enumerate}
\item If the functions are called with
  \varname{r} or \varname{i} being a
  \clieeeterm{signaling NaN}, then the\\
  \clname{cl:floating-point-invalid-operation} error is signaled.
\item If \varname{r} or \varname{i} are not \CL{}
  \clterm{real numbers} then the functions signals a \clname{cl:type-error}.
\end{enumerate}

\end{document}
