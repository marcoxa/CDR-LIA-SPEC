%%%% -*- Mode: LaTeX -*-
%%%% realpart-imagpart.tex

\documentclass[../Type-Manipulation-Coercion.tex]{subfiles}
\begin{document}

\DDictionaryItem{Functions \code{realpart}, \code{imagpart}}
\index{A!\code{realpart}}
\index{A!\code{imagpart}}

\DSyntax{}

\code{realpart} \varname{n} \RArrow \varname{r}\\
\code{imagpart} \varname{n} \RArrow \varname{r}

\DArgsNValues{}

\varname{n} -- A \clieeeterm{number}\\
\varname{r} -- A \clieeeterm{real}

\DDescription{}

The functions returns the real or the imaginary part of a number \varname{n}.
When \varname{n} is not \clieeeterm{NaN} or \clieeeterm{infinity}
their behavior is the one described in \cite{1996:ANSIHyperSpec}.

\noindent
When \varname{n} is a \clieeeterm{quiet NaN} the functions return a
\clieeeterm{quiet NaN}.

\noindent
When \varname{n} is \code{positive-infinity} or
\code{negative-infinity} then the result of
\code{realpart} is\\
\code{positive-infinity} or \code{negative-infinity}, the
result of \code{imagpart} is $0.0$ or $-0.0$.

\noindent
When \varname{n}, is a complex number with a \code{positive-infinity} or
\code{negative-infinity} as real part, then \code{realpart} returns
\code{positive-infinity} or \code{negative-infinity}.

\noindent
When \varname{n}, is a complex number with a \code{positive-infinity} or
\code{negative-infinity} as imaginary part, then \code{imagpart} returns
\code{positive-infinity} or \code{negative-infinity}.

\DExceptional{}

There are different exceptional situations to be considered:
\begin{enumerate}
\item If the functions are called with
  \varname{n} being a
  \clieeeterm{signaling NaN}, then the\\
  \clname{cl:floating-point-invalid-operation} error is signaled.
\item If \varname{n} is not a \CL{} \clterm{number} then the functions
  signals a \clname{cl:type-error}.
\end{enumerate}

\end{document}