%%%% -*- Mode: LaTeX -*-

%%%% Type-Manipulation-Coercion.tex

\documentclass[../../Operations.tex]{subfiles}
\begin{document}

\label{sect:transc-ops}

\begin{tt}
  \begin{tabular}{lll}
    ceiling & \# float-radix \# & \# rational \#\\
    complex &  \# float-sign \# &  \# rationalize \#\\
    \# decode-float \# & floor  &  realpart\\
    \# denominator \# &  fround &  \# rem \#\\
    fceiling & ftruncate &  round\\
    ffloor &  imagepart & \# scale-float \#\\
    float & \# integer-decode-float \# & truncate\\
    \# float-digits \# & \# mod \# & \\
    \# float-precision \# & \# numerator \# & \\
  \end{tabular}
\end{tt}

\vspace*{3mm}

\noindent
The above table corresponds to Figure~12-4 of \cite{1996:ANSIHyperSpec}.
The ``commented'' entries will not be described as they either do not
apply to float numbers or they have a trivial extension
compatible with \cite{2012:LIA1,2001:LIA2,2004:LIA3}, e.g. \code{float-sign}
has a natural extension to \code{positive-infinity},
\code{negative-infinity} and $-0.0$. As a general rule, if any argument of a
function is a \clieeeterm{quiet-Nan} then the function returns a
\clieeeterm{quiet-Nan}, if any argument of a function i a
\clieeeterm{signaling-Nan} then the
\clname{cl:floating-point-invalid-operation} error is signaled.

\noindent
For each of the functions in the table above, even if for some of them is
redundant, the following set of functions is defined:
\begin{itemize}
\item \code{function.<} compute the function with rounding downward
\item \code{function.>} compute the function with rounding upward
\item \code{function.<>} compute the function with rounding to nearest
\item \code{function} compute the function with the current rounding mode
\end{itemize}
\vspace*{3mm}

\subfile{definitions/floor-ffloor-ceiling-fceiling-truncate-ftruncate-round-fround.tex}
\subfile{definitions/complex.tex}
\subfile{definitions/float.tex}
\subfile{definitions/realpart-imagpart.tex}

\end{document}