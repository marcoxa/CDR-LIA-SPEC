%%%% -*- Mode: LaTeX -*-

%%%% Exponentials-Logarithms-Trigonometry.tex

\documentclass[../../Operations.tex]{subfiles}
\begin{document}

\label{sect:transc-ops}

\begin{tt}
  \begin{tabular}{lll}
    \#| abs |\# & cos & signum\\
    acos &  cosh &  sin\\
    acosh & exp  &  sinh\\
    asin &  expt &  sqrt\\
    asinh & isqrt &  tan\\
    atan &  log &   tanh\\
    atanh & phase & \\
    cis & \#| pi |\# & \\
  \end{tabular}
\end{tt}

\vspace*{3mm}

\noindent
The above table corresponds to Figure~12-2 of \cite{1996:ANSIHyperSpec}.
The ``commented'' entries will not be described as they either don't
operate on floating point numbers or do not have a semantic different
from the \CL{} standard.

\noindent
The listed \CL{} functions have correspondences in the \cite{2008:IEEE-754}
specification, but with some key differences, e.g., \code{cl:log} returns
a \clterm{complex number} for negative values.  In the following each
function will be further specified to comply with the \cite{2008:IEEE-754}
standard.


\noindent
For each of the functions in the table above the following set of functions is
defined:
\begin{itemize}
\item \code{function.<} compute the function with rounding downward
\item \code{function.>} compute the function with rounding upward
\item \code{function.<>} compute the function with rounding to nearest
\item \code{function} compute the function with the current rounding mode
\end{itemize}
\vspace*{3mm}

The following functions will also be specified for completeness,
according to Section~9 of \cite{2008:IEEE-754}.

\subfile{definitions/asin-acos-atan.tex}
\subfile{definitions/sin-cos-tan.tex}
\subfile{definitions/sinh-cosh-tanh-asinh-acosh-atanh.tex}
\subfile{definitions/phase.tex}
\subfile{definitions/exp-expt-cis.tex}
\subfile{definitions/sqrt-isqrt.tex}
\subfile{definitions/signum.tex}
\subfile{definitions/log.tex}

\end{document}