%%%% -*- Mode: LaTeX -*-

%%%% exp-expt-cis.tex

\documentclass[../Exponentials-Logarithms-Trigonometry.tex]{subfiles}
\begin{document}

    \DDictionaryItem{Functions \code{exp}, \code{expt}, \code{cis}}
    \index{A!\code{exp}}
    \index{A!\code{expt}}
    \index{A!\code{cis}}

    \DSyntax{}

    \code{exp} \varname{n} \RArrow \varname{result}\\
    \code{expt} \varname{base} \varname{power} \RArrow \varname{result}\\
    \code{cis} \varname{radians} \RArrow \varname{result}

    \DArgsNValues{}

    \varname{n} -- A \clieeeterm{number}\\
    \varname{base} -- A \clieeeterm{number}\\
    \varname{power} -- A \clieeeterm{number}\\
    \varname{radians} -- A \clieeeterm{real number}

    \DDescription{}

    The function \code{exp} compute the natural exponentiation of a number. In
    this section we describe only \code{exp}, the description and the properties
    of \code{expt} and \code{cis} can be derived with the relations:
    \begin{itemize}
        \item \code{(expt} \varname{a} \varname{b}\code{)} $\equiv$ \code{(* (exp}
        \varname{b}\code{)} \code{(log} \varname{a}\code{))}
        \item \code{(cis} \varname{x}\code{)} $\equiv$ \code{(exp} \code{\#c(0}
        \varname{x}\code{))}
    \end{itemize}


    When \varname{n} is not a \clieeeterm{NaNs} or an
    \clieeeterm{infinity} their behavior is the one described in
    \cite{1996:ANSIHyperSpec} for regular floating point numbers and complex
    numbers.

    \noindent
    When \varname{n} is a \clieeeterm{quiet NaN} the function \code{exp} returns a
    \clieeeterm{quiet NaN}.

    \noindent
    The function \code{exp}, when \varname{n} is a complex number, \varname{x}
    and \varname{y} are its \clieeeterm{real} and \clieeeterm{imaginary} parts
    \begin{itemize}
        \item If \varname{x} is a \code{negative-zero} returns \code{(exp \#c(0 }
        \varname{y}\code{))}
        \item If \varname{x} is not a \code{negative-zero} and \varname{y} is a
        \code{negative-zero} returns \code{(conjugate (exp \#c(}\varname{x}
        \code{0)))}
        \item If \varname{x} is \code{negative-infinity} and
        \varname{y} is not \code{positive-infinity} or \code{negative-infinity}
        returns \code{\#c((* 0 (cos }\varname{y}\code{)) (* 0 (sin
        }\varname{y}\code{)))}
        \item If \varname{x} is \code{positive-infinity} and
        \varname{y} is not \code{positive-infinity} or \code{negative-infinity} or $0$
        returns \code{\#c((* positive-infinity (cos }\varname{y}\code{)) (*
        positive-infinity (sin
        }\varname{y}\code{)))}
        \item If \varname{x} is \code{positive-infinity} and \varname{y} is $0$
        returns \code{\#c(positive-infinity 0)}
    \end{itemize}

    \DExceptional{}

    There are different exceptional situations to be considered:
    \begin{enumerate}
        \item If \code{exp} is called with
        \varname{n} being a \clieeeterm{signaling NaN}, then the
        \clname{cl:floating-point-invalid-operation} error is signaled.
        \item If \code{exp} is called with the imaginary
        part of \varname{n} being a \code{infinity}, then the
        \clname{cl:floating-point-invalid-operation} error is signaled.
        \item If \varname{n} is not \CL{}
        \clterm{number} then the function \code{exp} signals a
        \clname{cl:type-error}.
    \end{enumerate}

    \DSeeAlso{}

    \code{log}.

\end{document}