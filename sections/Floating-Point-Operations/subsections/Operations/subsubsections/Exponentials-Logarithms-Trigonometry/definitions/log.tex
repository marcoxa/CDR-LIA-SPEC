%%%% -*- Mode: LaTeX -*-

%%%% log.tex

\documentclass[../Comparisons-Predicates.tex]{subfiles}
\begin{document}

    \DDictionaryItem{Function \code{log}}
    \index{S!\code{log}}

    \DSyntax{}

    \code{log} \varname{n} \varname{base} \RArrow{} \varname{number}

    \DArgsNValues{}

    \varname{n} -- A \clterm{number}.\\
    \varname{base} -- A \clterm{number}.\\
    \varname{number} -- A \clterm{number}.

    \DDescription{}

    The function \code{log} compute the \emph{natural logarithm} of
    \varname{n} if \varname{base} is not supplied. If \varname{base} is
    supplied \code{(log} \varname{n} \varname{base}\code{)} $\equiv$ \code{
    (/ (log} \varname{n}\code{) (log} \varname{base}\code{))}

    \noindent
    When \varname{n} is not a \clieeeterm{NaNs} or an
    \clieeeterm{infinity} its behavior is the one described in
    \cite{1996:ANSIHyperSpec} for regular floating point numbers and complex
    numbers.

    \noindent
    When \varname{n} is a \clieeeterm{quiet NaN} the function \code{log}
    returns a \clieeeterm{quiet NaN}.

    \noindent
    When \varname{n} is a negative \clterm{real}
    number the function \code{log} does not signal a
    \code{cl:floating-point-invalid-operation}, it quietly returns a
    \clterm{complex} number.

    \noindent
    The function \code{log}
    \begin{itemize}
        \item When \varname{n} is $0$ or \code{negative-zero} returns
        \code{negative-infinity}
        \item When \varname{n} is \code{positive-infinity} returns
        \code{positive-infinity}
        \item When \varname{n} is a complex number and \varname{x} and
        \varname{y} are its real and imaginary parts
        \begin{itemize}
            \item If \varname{x} is $0$ or \code{negative-zero} and
            \varname{y} is $0$ returns \code{\#c(negative-infinity (phase
            \#c(}\varname{x} \varname{y}\code{)))}
            \item If \varname{y} is \code{negative-zero} returns \code
            {(conjugate (log \#c(}\varname{x} \code{0)))}
            \item If \varname{x} is \code{negative-zero} and \varname{y} is a
            positive \clterm{number} or \code{positive-infinity} returns
            \code{\#c((log} \varname{y}\code{) (/ $\pi$ 2)}
            \item If \varname{x} is \code{negative-zero} and \varname{y} is a
            negative \clterm{number} or \code{negative-infinity} returns
            \code{\#c((log (-} \varname{y}\code{)) (- (/ $\pi$ 2))}
            \item If \varname{x} is an \clieeeterm{infinity} and \varname{y}
            is a finite number or an \clieeeterm{infinity} returns
            \code{\#c(positive-infinity (phase \#c(}\varname{x}
            \varname{y}\code{)))}
            \item If \varname{x} is a \clterm{finite} number and \varname{y}
            is an \clieeeterm{infinity} returns
            \code{\#c(positive-infinity (phase
            \#c(}\varname{x} \varname{y}\code{)))}
        \end{itemize}
    \end{itemize}


    \DExceptional{}

    There are different exceptional situations to be considered:
    \begin{enumerate}
        \item If \code{log} is called with
        \varname{n} being a \clieeeterm{signaling NaN}, then the
        \clname{cl:floating-point-invalid-operation} error is signaled.
        \item If \varname{n} is not \CL{}
        \clterm{number} then the function \code{log} signals a
        \clname{cl:type-error}.
    \end{enumerate}

\end{document}