%%%% -*- Mode: LaTeX -*-

%%%% sin-cos-tan.tex

\documentclass[../Exponentials-Logarithms-Trigonometry.tex]{subfiles}
\begin{document}

\DDictionaryItem{Functions \code{sin}, \code{cos}, \code{tan}}
\index{A!\code{sin}}
\index{A!\code{cos}}
\index{A!\code{tan}}

\DSyntax{}

\code{sin} \varname{radians} \RArrow \varname{result}\\
\code{cos} \varname{radians} \RArrow \varname{result}\\
\code{tan} \varname{radians} \RArrow \varname{result}

\DArgsNValues{}

\varname{radians} -- A \clieeeterm{floating
  point number}\\
\varname{result} -- A \clieeeterm{floating point number}

\DDescription{}

The functions \code{sin}, \code{cos}, \code{tan} compute the sine, cosine
and tangent of a number. When \varname{radians} is not a \clieeeterm{NaNs} or an
\clieeeterm{infinity} their behavior is the one described in
\cite{1996:ANSIHyperSpec} for regular floating point numbers and complex
numbers.

\noindent
When \varname{radians} is a \clieeeterm{quiet NaN} the functions \code{sin},
\code{cos}, \code{tan} return a \clieeeterm{quiet NaN}.

\noindent
The function \code{sin}, when \varname{radians} is a complex number, \varname{x}
and \varname{y} are its \clieeeterm{real} and \clieeeterm{imaginary} parts
\begin{itemize}
  \item If \varname{y} is $-0.0$ returns \code{(conjugate
  (sin (complex }\varname{x}\code{ 0)))}
  \item If \varname{x} is $-0.0$ and \varname{y} is not
$-0.0$ returns \code{(- (sin (complex 0 (-
  }\varname{y}\code{))))}
  \item If \varname{x} is not $-0.0$ or $0.0$ and
  \varname{y} is \code{positive-infinity} or \code{negative-infinity} returns
  \code{(complex (* (sin }\varname{x}\code{) positive-infinity) (*
  (cos} \varname{x}\code{)} \varname{y}\code{))}
  \item If \varname{x} is $0.0$ and \varname{y} is
  \code{positive-infinity} or \code{negative-infinity} returns \code{(complex 0
  }\varname{y}\code{)}
\end{itemize}

\noindent
The function \code{cos}, when \varname{radians} is a complex number, \varname{x}
and \varname{y} are its \clieeeterm{real} and \clieeeterm{imaginary} parts
\begin{itemize}
  \item If \varname{y} is $-0.0$ returns \code{(conjugate
  (cos (complex }\varname{x}\code{ 0)))}
  \item If \varname{x} is $-0.0$ and \varname{y} is not
$-0.0$ returns \code{(cos (complex 0 (-
  }\varname{y}\code{)))}
  \item If \varname{x} is not $-0.0$ or $0.0$ and
  \varname{y} is \code{positive-infinity} or \code{negative-infinity} returns
  \code{(complex (* (cos }\varname{x}\code{) positive-infinity) (*
  (sin} \varname{x}\code{)} (- \varname{y})\code{))}
  \item If \varname{x} is $0.0$ and \varname{y} is
  \code{positive-infinity} or \code{negative-infinity} returns
  \item \code{(complex positive-infinity }(/ (-1) \varname{y})\code{)}
\end{itemize}

\noindent
The function \code{tan}, when \varname{radians} is a complex number, \varname{x}
and \varname{y} are its \clieeeterm{real} and \clieeeterm{imaginary} parts
\begin{itemize}
  \item If \varname{y} is $-0.0$ returns \code{(conjugate
  (tan (complex }\varname{x}\code{ 0)))}
  \item If \varname{x} is $-0.0$ and \varname{y} is not
$-0.0$ returns \code{(- (tan (complex 0 (-
  }\varname{y}\code{))))}
  \item If \varname{x} is not $-0.0$ and \varname{y} is a
  \code{positive-infinity} or a \code{negative-infinity} returns \code{(complex (*
  (tan }\varname{x}\code{) 0) (signum }\varname{y}\code{))}
\end{itemize}

\DExceptional{}

There are different exceptional situations to be considered:
\begin{enumerate}
  \item If \code{sin}, \code{cos} and \code{tan} are called with
    \varname{radians} being a \clieeeterm{signaling NaN}, then the
    \clname{cl:floating-point-invalid-operation} error is signaled.
  \item If \code{sin}, \code{cos} and \code{tan} are called with the real
    part of \varname{radians} being a \code{infinity}, then the
    \clname{cl:floating-point-invalid-operation} error is signaled.
  \item If \code{tan} is called with the real part of
    \varname{radians} being an $\pi/2 + k\pi$, then the
    \clname{cl:floating-point-invalid-operation} error is signaled.
  \item If \varname{radians} is not \CL{}
    \clterm{number} then the functions \code{asin}, \code{acos} and
    \code{atan} signal a \clname{cl:type-error}.
\end{enumerate}

\DSeeAlso{}

\code{*}, \code{-}, \code{signum}, \code{abs}, \code{conjugate}.

\end{document}