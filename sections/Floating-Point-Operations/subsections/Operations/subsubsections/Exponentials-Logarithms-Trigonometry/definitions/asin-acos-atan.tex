\documentclass[../Exponentials-Logarithms-Trigonometry.tex]{subfiles}
\begin{document}

\DDictionaryItem{Functions \code{asin}, \code{acos}, \code{atan}}
\index{A!\code{asin}}
\index{A!\code{acos}}
\index{A!\code{atan}}

\DSyntax{}

\code{asin} \varname{n} \RArrow \varname{radians}\\
\code{acos} \varname{n} \RArrow \varname{radians}\\
\code{atan} \varname{n1} \code{\&optional} \varname{n2} \RArrow
\varname{radians}

\DArgsNValues{}

\varname{n}, \varname{n1}, \varname{n2} -- A \clieeeterm{floating
  point number}\\
\varname{radians} -- A \clieeeterm{floating point number}

\DDescription{}

The functions \code{asin}, \code{acos}, \code{atan} compute the the
arc sine, arc cosine and arc tangent of a number.  Their behavior is
the one described in \cite{1996:ANSIHyperSpec} for regular floating point
numbers.

The functions return a \clieeeterm{quiet NaN} if \varname{n},
\varname{n1}, or \varname{n2} is a \clieeeterm{quiet NaN}.

The behavior of \code{asin}, \code{acos}, and \code{atan} in case of
\clterm{complex} arguments is also the one described in
\cite{1996:ANSIHyperSpec}.

\DNotes{}

The functions \code{asin} and \code{acos} do not signal a
\code{cl:floating-point-invalid-operation} when \varname{n} is a
\clterm{real} outside the $[-1, 1]$ interval; they quietly return a
\clterm{complex} number.

\DExceptional{}

If \varname{n}, \varname{n1}, or \varname{n2} is a
\clieeeterm{signaling NaN}, then \code{asin}, \code{acos} and
\code{atan} signal the\\
\clname{cl:floating-point-invalid-operation} error.

According to \cite{2008:IEEE-754} there are several issues to be
considered.

If \varname{n}, \varname{n1}, or \varname{n2} are not \clterm{number}s
a \code{type-error} is signaled by \code{acos}, \code{asin}, and
\code{atan}.  If \varname{n1} and \varname{n2} are both supplied to
\code{atan}, and they are not both \clterm{real} numbers, then a
\code{type-error} is signaled.

\end{document}