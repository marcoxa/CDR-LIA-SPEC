\documentclass[../Exponentials-Logarithms-Trigonometry.tex]{subfiles}
\begin{document}

\DDictionaryItem{Functions \code{asin}, \code{acos}, \code{atan}}
\index{A!\code{asin}}
\index{A!\code{acos}}
\index{A!\code{atan}}

\DSyntax{}

\code{asin} \varname{n} \RArrow \varname{radians}\\
\code{acos} \varname{n} \RArrow \varname{radians}\\
\code{atan} \varname{n1} \code{\&optional} \varname{n2} \RArrow
\varname{radians}

\DArgsNValues{}

\varname{n}, \varname{n1}, \varname{n2} -- A \clieeeterm{floating
  point number}\\
\varname{radians} -- A \clieeeterm{floating point number}

\DDescription{}

The functions \code{asin}, \code{acos}, \code{atan} compute the
arc sine, arc cosine and arc tangent of a number. When \varname{n},
\varname{n1} and \varname{n2} are not
\clieeeterm{NaNs} or \clieeeterm{infinities} their behavior is the
one described in \cite{1996:ANSIHyperSpec} for regular floating
point numbers and complex numbers.

\noindent
When \varname{n} is a \clterm{real} number outside the $[-1; 1]$
interval the functions \code{asin} and \code{acos} do no signal a
\code{cl:floating-point-invalid-operation}, they quietly return a
\clterm{complex} number.

\noindent
When \varname{n} or \varname{n1} or \varname{n2} are
\clieeeterm{quiet NaN} the functions \code{asin}, \code{acos},
\code{atan} return a \clieeeterm{quiet NaN}.

\noindent
The function \code{asin}
\begin{itemize}
\item When \varname{n} is \code{negative-infinity} returns
  \code{\#c((- (\ pi 2)) positive-infinity)}

\item When \varname{n} is \code{positive-infinity} returns
  \code{\#c((\ pi 2) negative-infinity)}

\item When \varname{n} is a complex number, \varname{x} and \varname{y}
  are its \clieeeterm{real} and \clieeeterm{imaginary} parts
  \begin{itemize}
  \item If \varname{x} is a \code{negative-zero} returns \code{(- (asin
      \#c(0 (- }\varname{y}\code{))))}
  \item If \varname{y} is a \code{negative-zero} and \varname{x} is not a
    \code{negative-zero} returns \code{(conjugate (asin
      \#c(}\varname{x}\code{ (- 0)))}
  \item If \varname{x} is a \code{positive-infinity} or
    \code{negative-infinity} and \varname{y} a real number OR if \varname{y} is
    a \code{positive-infinity} or a \code{negative-infinity} and \varname{x} is
    a \clieeeterm{finite} \clieeeterm{real} number returns \code{\#c((phase
      \#c((abs }\varname{y}\code{) }\varname{x}\code{)) (* (signum}
    \varname{y}\code{) positive-infinity))}
  \end{itemize}
\end{itemize}

\noindent
The function \code{acos}
\begin{itemize}
\item When \varname{n} is \code{negative-infinity} returns \code{\#c(pi
    negative-infinity)}
\item When \varname{n} is \code{positive-infinity} returns \code{\#c(0
    positive-infinity)}
\item When \varname{n} is a complex number, \varname{x} and \varname{y}
  are its \clieeeterm{real} and \clieeeterm{imaginary} parts
  \begin{itemize}
  \item If \varname{x} is a \code{negative-zero} returns \code{\#c((\ pi 2)
      (- (asinh }\varname{y}\code{)))}
  \item If \varname{y} is a \code{negative-zero} and \varname{x} is not a
    \code{negative-zero} returns \code{(conjugate (acos \#c
      (}\varname{x}\code{ 0))}
  \item If \varname{x} is a \code{positive-infinity} or
    \code{negative-infinity} and \varname{y} a positive \clieeeterm{real}
    number or zero or a \code{positive-infinity} returns \code{\#c((phase n)
      negative-infinity))}
  \item If \varname{x} is a \code{positive-infinity} or
    \code{negative-infinity} and \varname{y} a negative \clieeeterm{real}
    number or a \code{negative-infinity} returns
    \code{\#c((phase \#c(x (- y)) positive-infinity))}
  \end{itemize}
\end{itemize}

\noindent
The function \code{atan}
\begin{itemize}
\item When \varname{n1}, or the ratio \varname{n1}$/$\varname{n2}, is a
  \code{positive-infinity} (or \code{negative-infinity}) it returns $+\pi/2$
  (or $-\pi/2$)
\item When \varname{n1} is a complex
  number, \varname{x} and \varname{y} are its \clieeeterm{real} and
  \clieeeterm{imaginary} parts,
  \begin{itemize}
  \item If $x = 0$ and $y \in \{-1, 1\}$ then the function
    \code{(atan }\varname{n1}\code{)}
    returns \code{\#c(0 (* }\varname{y} \code{positive-infinity))}
  \item If \varname{x} is a \code{negative-zero} return \code{(- (atan \#c(0 (-
    }\varname{y}\code{))))}
  \item If \varname{y} is a \code{negative-zero} and \varname{x} is not a
    \code{negative-zero} return \code{(conjugate (atan
      \#c(}\varname{x}\code{ 0))}
  \item If \varname{x} is a \code{positive-infinity} or
    \code{negative-infinity} and \varname{y} a real number OR if \varname{y} is
    a \code{positive-infinity} or a \code{negative-infinity} and \varname{x} is
    a \clieeeterm{finite} \clieeeterm{real} number return \code{\#c((* (signum
    }\varname{x}\code{) (\ pi 2)) (* (signum
    }\varname{y}\code{) 0))}
  \end{itemize}
\end{itemize}


\DExceptional{}

There are different exceptional situations to be considered:
\begin{enumerate}
\item If \code{asin}, \code{acos} and \code{atan} are called with
  \varname{n}, \varname{n1}, or \varname{n2} being a
  \clieeeterm{signaling NaN}, then the
  \clname{cl:floating-point-invalid-operation} error is signaled.
\item If \code{atan} is called with both \varname{n1} and
  \varname{n2} and at least one of them is not a \clterm{real} number,
  then the a \code{type-error} is signaled.
\item If \varname{n} or \varname{n1} or \varname{n2} are not \CL{}
  \clterm{number} then the functions \code{asin}, \code{acos} and
  \code{atan} signals a \clname{cl:type-error}.
\end{enumerate}

\DSeeAlso{}

\code{*}, \code{-}, \code{signum}, \code{abs}, \code{conjugate}.

\end{document}