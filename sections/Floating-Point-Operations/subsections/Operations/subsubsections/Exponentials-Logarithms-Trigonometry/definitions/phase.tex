%%%% -*- Mode: LaTeX -*-

%%%% phase.tex

\documentclass[../Exponentials-Logarithms-Trigonometry.tex]{subfiles}
\begin{document}

\DDictionaryItem{Function \code{phase}}
\index{A!\code{phase}}

\DSyntax{}

\code{phase} \varname{n} \RArrow \varname{result}

\DArgsNValues{}

\varname{n} -- A \clieeeterm{floating
  point number}\\
\varname{result} -- A \clieeeterm{floating point number}

\DDescription{}

The function \code{phase}, \code{cos}, \code{tan} compute the phase of a 
number in radians. When \varname{n} is not a \clieeeterm{NaNs} or an
\clieeeterm{infinity} its behavior is the one described in
\cite{1996:ANSIHyperSpec} for regular floating point numbers and complex
numbers.

\noindent
When \varname{n} is a \clieeeterm{quiet NaN} the functions \code{sin},
\code{cos}, \code{tan} return a \clieeeterm{quiet NaN}.

When \varname{n} is a complex number and \varname{x} and \varname{y} are its
real and imaginary parts
\begin{itemize}
\item If \varname{x} is \code{positive-infinity} and \varname{y} is not an
  infinity \code{phase} returns $0$
\item If \varname{x} is \code{negative-infinity} and \varname{y} is not an
  infinity \code{phase} returns $\pi$
\item If \varname{y} is \code{positive-infinity} and \varname{x} is not an
  infinity \code{phase} returns $\pi/2$
\item If \varname{y} is \code{negative-infinity} and \varname{x} is not an
  infinity \code{phase} returns $-\pi/2$
\end{itemize}

\DExceptional{}

There are different exceptional situations to be considered:
\begin{enumerate}
\item If \code{phase} is called with
  \varname{n} being a \clieeeterm{signaling NaN}, then the
  \clname{cl:floating-point-invalid-operation} error is signaled.
\item If \code{phase} is called with a number with both the real and
  imaginary parts of \varname{n} being a \code{infinity}, then the
  \clname{cl:floating-point-invalid-operation} error is signaled.
\item If \varname{n} is not \CL{}
  \clterm{number} then the functions \code{phase} signals a
  \clname{cl:type-error}.
\end{enumerate}

\end{document}