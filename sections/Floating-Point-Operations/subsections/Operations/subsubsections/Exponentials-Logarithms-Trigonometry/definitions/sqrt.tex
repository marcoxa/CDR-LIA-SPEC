\documentclass[../Comparisons-Predicates.tex]{subfiles}
\begin{document}

    \DDictionaryItem{Function \code{sqrt}}
    \index{S!\code{sqrt}}

    \DSyntax{}

    \code{sqrt} \varname{n} \RArrow{} \varname{root}

    \DArgsNValues{}

    \varname{n} -- A \clterm{number}.\\
    \varname{root} -- A \clterm{number}.

    \DDescription{}

    The function \code{sqrt} compute the \emph{square root} of \varname{n}.

    \noindent
    When \varname{n} is not a \clieeeterm{NaNs} or an
    \clieeeterm{infinity} its behavior is the one described in
    \cite{1996:ANSIHyperSpec} for regular floating point numbers and complex
    numbers.

    \noindent
    When \varname{n} is a \clieeeterm{quiet NaN} the function \code{sqrt}
    returns a
    \clieeeterm{quiet NaN}.

    \noindent
    When \varname{n} is a negative \clterm{real}
    number the function \code{sqrt} does not signal a
    \code{cl:floating-point-invalid-operation}, it quietly returns a
    \clterm{complex} number.

    \noindent
    The function \code{sqrt}, when \varname{n} is a complex number, \varname{x}
    and \varname{y} are its \clieeeterm{real} and \clieeeterm{imaginary} parts
    \begin{itemize}
        \item If \varname{x} is a \code{negative-zero} returns \code{(sqrt \#c
        (0 }
        \varname{y}\code{))}
        \item If \varname{x} is a \clieeeterm{number} or an
        \clieeeterm{infinity} and
        \varname{y} is a
        \code{negative-zero} returns \code{(conjugate (sqrt \#c(}\varname{x}
        \code{0)))}
        \item If \varname{x} is a \clieeeterm{number} or an
        \clieeeterm{infinity} and
        \varname{y} is a
        \code{positive-infinity} returns \code{\#c(positive-infinity
        positive-infinity)}
        \item If \varname{x} is \code{positive-infinity} and
        \varname{y} is a greater than or equal to $0$ returns \code{\#c
        (positive-infinity
        0)}
        \item If \varname{x} is \code{positive-infinity} and
        \varname{y} is a less than $0$ returns \code{\#c
        (positive-infinity
        negative-zero)}
        \item If \varname{x} is a \clieeeterm{number} or an
        \clieeeterm{infinity} and
        \varname{y} is a
        \code{negative-infinity} returns \code{\#c(positive-infinity
        negative-infinity)}
        \item If \varname{x} is \code{negative-infinity} and
        \varname{y} is a greater than or equal to $0$ returns \code{\#c
        (0 positive-infinity)}
        \item If \varname{x} is \code{negative-infinity} and
        \varname{y} is a less than $0$ returns \code{\#c
        (0 negative-infinity)}
    \end{itemize}


    \DExceptional{}

    There are different exceptional situations to be considered:
    \begin{enumerate}
        \item If \code{sqrt} is called with
        \varname{n} being a \clieeeterm{signaling NaN}, then the
        \clname{cl:floating-point-invalid-operation} error is signaled.
        \item If \varname{n} is not \CL{}
        \clterm{number} then the function \code{sqrt} signals a
        \clname{cl:type-error}.
    \end{enumerate}

%    If \varname{} is a \clliaterm{signaling NaN} then if the notification
%    style is NACF then a\\
%    \code{floating-point-invalid-operation} is
%    signaled, with a \clliaterm{quiet NaN} as a continuation value. If
%    the notification style is NRI then the \code{:invalid} indicator is
%    recorded and a \clliaterm{quiet NaN} is returned as continuation
%    value.

    \DNotes{}

    The LIA specification suggests to call \code{sqrt.<} and \code{sqrt.>}
    as \code{sqrtDwn} and \code{sqrtUp} (cfr., LIA1
    \cite{2012:LIA1}). This suggestion goes against the traditional \CL{}
    naming verbosity ``feature'', plus it assumes case-sensitivity, which
    \CL{} does not have in default mode. Therefore the more evocative
    names \code{sqrt.<} and \code{sqrt.>} (and \code{sqrt.<>}) are
    introduced.

\end{document}