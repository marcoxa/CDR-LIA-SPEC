%%%% -*- Mode: LaTeX -*-

%%%% signum.tex

\documentclass[../Comparisons-Predicates.tex]{subfiles}
\begin{document}

    \DDictionaryItem{Functions \code{signum}, \code{float-sign}}
    \index{S!\code{signum}}
    \index{F!\code{float-sign}}


    \DSyntax{}

    \code{signum} \varname{n} \RArrow \varname{signed-prototype}\\
    \code{float-sign} \varname{f1} \code{\&optional} \varname{f2}
    \RArrow \varname{signed-float}


    \DArgsNValues{}

    \varname{n} -- a \clterm{number}\\
    \varname{signed-prototype} -- a \clterm{number}\\
    \varname{f1} -- a \clterm{float}\\
    \varname{f2} -- a \clterm{float}\\
    \varname{signed-float} -- a \clterm{float}


    \DDescription{}

    The function \code{signum} computes the sign of \varname{n}. The function
    \code{float-sign} computes the sign of \varname{f1} is \varname{f2} is not
    provided, otherwise it returns a number with the same absolute value of
    \varname{f2} and the sign of \varname{f1}.

    \noindent
    When \varname{n} or \varname{f1} or \varname{f2} are not a
    \clieeeterm{NaNs} or an
    \clieeeterm{infinity} their behavior is the one described in
    \cite{1996:ANSIHyperSpec} for regular floating point numbers and complex
    numbers.

    \noindent
    When \varname{n} or \varname{f1} or \varname{f2} are a \clieeeterm{quiet
    NaN} the functions \code{signum} or \code{float-sign} return a
    \clieeeterm{quiet NaN}.

    \noindent
    The function \code{signum}
    \begin{itemize}
        \item When \varname{n} is \code{positive-infinity} returns $1$
        \item When \varname{n} is \code{negative-infinity} returns $-1$
        \item When \varname{n} is a complex number and \varname{x} and
        \varname{y} are its real and imaginary parts
        \begin{itemize}
            \item If \varname{y} is \code{negative-infinity} or
            code{negative-zero} returns \code{(conjugate (signum (conjugate
            (complex }\varname{x} \varname{y}\code{))))}
            \item If \varname{x} is \code{negative-infinity} or
            code{negative-zero} and \varname{y} is not \code{negative-infinity} or
            code{negative-zero} returns \code{(- (signum (-
            (complex }\varname{x} \varname{y}\code{))))}
            \item Else is equivalent to \code{(complex (sin (phase (complex }\varname{y} \varname{x}\code{))) (sin
            (phase (complex }\varname{y} \varname{x}\code{))))}
        \end{itemize}
    \end{itemize}


    The function \code{float-sign}
    \begin{itemize}
        \item If \varname{f2} is not provided has the same behavior of
        \code{signum}
        \item If \varname{f2} is a \code{positive-infinity} or a
        \code{negative-infinity} returns
        \code{(* (signum} \varname{f1}\code{) positive-infinity)}
        \item If \varname{f2} is a complex number number and its real or
        imaginary parts are \code{positive-infinity} or
        \code{negative-infinity} returns
        \code{(* (signum} \varname{f1}\code{) positive-infinity)}
        \item If \varname{f2} is a \emph{finite} real or complex number is
        equivalent to \code{(* (signum} \varname{f1}\code{) (abs}
        \varname{f2}\code{)}
    \end{itemize}

    \DExceptional{}

    There are different exceptional situations to be considered:
    \begin{enumerate}
        \item If \code{signum} or \code{float-sign} are called with
        \varname{n} or \varname{f1} or \varname{f2} being a
        \clieeeterm{signaling NaN}, then the
        \clname{cl:floating-point-invalid-operation} error is signaled.
        \item If \varname{n} or \varname{f1} or \varname{f2} are not \CL{}
        \clterm{number} then the functions \code{signum} and \code{float-sign}
        signal a \clname{cl:type-error}.
    \end{enumerate}

%    If \varname{n} is a floating point \clliaterm{signaling NaN} and the
%    notification style is NACF, then a\\
%    \code{floating-point-invalid} condition is signaled with a
%    \clliaterm{quiet NaN} as a \emph{continuation value}; if the
%    notification style is NRI then the indicator \code{:invalid} is
%    recorder and a \clliaterm{quiet NaN} is returned.
%
%    A \code{type-error} is signaled if \varname{n} is not a number or
%    either \varname{f1}, or \varname{f2} is not a \clterm{float} number.


    \DNotes{}

    The specification of \code{signum} is substantially different from the
    one present in \cite{1994:ANSICL}.

\end{document}