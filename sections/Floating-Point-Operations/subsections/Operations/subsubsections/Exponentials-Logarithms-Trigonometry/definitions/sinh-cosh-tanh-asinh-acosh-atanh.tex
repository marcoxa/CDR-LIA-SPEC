%%%% -*- Mode: LaTeX -*-

%%%% sinh-cosh-tanh-asinh-acosh-atanh.tex

\documentclass[../Exponentials-Logarithms-Trigonometry.tex]{subfiles}
\begin{document}

\DDictionaryItem{Functions \code{sinh}, \code{cosh}, \code{tanh},
\code{asinh}, \code{acosh}, \code{atanh}}
\index{A!\code{sinh}}
\index{A!\code{cosh}}
\index{A!\code{tanh}}
\index{A!\code{asinh}}
\index{A!\code{acosh}}
\index{A!\code{atanh}}

\DSyntax{}

\code{sinh} \varname{n} \RArrow \varname{result}\\
\code{cosh} \varname{n} \RArrow \varname{result}\\
\code{tanh} \varname{n} \RArrow \varname{result}\\
\code{asinh} \varname{n} \RArrow \varname{result}\\
\code{acosh} \varname{n} \RArrow \varname{result}\\
\code{atanh} \varname{n} \RArrow \varname{result}

\DArgsNValues{}

\varname{n} -- A \clieeeterm{floating
  point number}\\
\varname{result} -- A \clieeeterm{floating point number}

\DDescription{}

The functions \code{sinh}, \code{cosh}, \code{tanh}, \code{asinh},
\code{acosh} and \code{atanh} compute the hyperbolic
sine, hyberbolic cosine, hyperbolic tangent, hyperbolic
arc sine, hyberbolic arc cosine and hyperbolic arc tangent of a number. When
\varname{n} is not a \clieeeterm{NaNs} or an
\clieeeterm{infinity} their behavior is the one described in
\cite{1996:ANSIHyperSpec} for regular floating point numbers and complex
numbers.

\noindent
When \varname{n} is a \clterm{real} number smaller than $+1$ the function
\code{acosh} does not signal a \\\code{cl:floating-point-invalid-operation}, it
quietly returns a \clterm{complex} number.

\noindent
When \varname{n} is a \clterm{real} number outside the interval $[-1, +1]$
the function \code{atanh} does not signal a \\
\code{cl:floating-point-invalid-operation}, it quietly
returns a \clterm{complex} number.

\noindent
When \varname{n} is a \clieeeterm{quiet NaN} the functions \code{sinh},
\code{cosh}, \code{tanh}, \code{asinh}, \code{acosh} and \code{atanh} return
a \clieeeterm{quiet NaN}.

\noindent
The function \code{sinh}
\begin{itemize}
  \item When \varname{n} $\in \{$\code{negative-infinity}$,$
  $-0.0$$,$ \code{positive-infinity}$\}$ returns \varname{n}
  \item When \varname{n} is a complex number and  \varname{x} and \varname{y}
  are its real and imaginary parts has the behaviour and the exceptional
  situations derived from the relation: \code{(sinh }\varname{n}\code{)}
  $\equiv$ \code{(* (complex 0 1) (sin (complex }\varname{y}\code{ (-
  }\varname{x}\code{)))}
\end{itemize}

\noindent
The function \code{cosh}
\begin{itemize}
  \item When \varname{n} $\in \{$\code{negative-infinity}$,$
  \code{positive-infinity}$\}$ returns \code{positive-infinity}
  \item When \varname{n} is $-0.0$ returns $+1$
  \item When \varname{n} is a complex number and  \varname{x} and \varname{y}
  are its real and imaginary parts has the behaviour and the exceptional
  situations derived from the relation: \code{(cosh }\varname{n}\code{)}
  $\equiv$ \code{(cos (complex }\varname{y}\code{ (-
  }\varname{x}\code{)))}
\end{itemize}

\noindent
The function \code{tanh}
\begin{itemize}
  \item When \varname{n} is \code{negative-infinity} returns
  $-1$
  \item When \varname{n} is $-0.0$ returns $-0.0$
  \item When \varname{n} is \code{positive-infinity} returns
  $1$
  \item When \varname{n} is a complex number and  \varname{x} and \varname{y}
  are its real and imaginary parts has the behaviour and the exceptional
  situations derived from the relation: \code{(tanh }\varname{n}\code{)}
  $\equiv$ \code{(* (complex 0 1) (tan (complex }\varname{y}\code{ (-
  }\varname{x}\code{)))}
\end{itemize}

\noindent
The function \code{asinh}
\begin{itemize}
  \item When \varname{n} $\in \{$\code{negative-infinity}$,$
  $-0.0$$,$ \code{positive-infinity}$\}$ returns \varname{n}
  \item When \varname{n} is a complex number and  \varname{x} and \varname{y}
  are its real and imaginary parts has the behaviour and the exceptional
  situations derived from the relation: \code{(asinh }\varname{n}\code{)}
  $\equiv$ \code{(* (complex 0 1) (asin (complex }\varname{y}\code{ (-
  }\varname{x}\code{)))}
\end{itemize}

\noindent
The function \code{acosh}
\begin{itemize}
  \item When \varname{n} is \code{positive-infinity} returns
  \code{positive-infinity}
  \item When \varname{n} is a complex number and  \varname{x} and \varname{y}
  are its real and imaginary parts has the behaviour and the exceptional
  situations derived from the following relations:
  \begin{itemize}
    \item If \varname{y} is greater than or equal to zero or
    \code{positive-infinity}, \code{(acosh
    }\varname{n}\code{)}
    $\equiv$ \code{(* (complex 0 1) (acos (complex }\varname{x}\code{
    }\varname{y}\code{)))}
    \item Else  \code{(acosh
    }\varname{n}\code{)}
    $\equiv$ \code{(- (* (complex 0 1) (acos (complex }\varname{x}\code{
    }\varname{y}\code{))))}
  \end{itemize}
\end{itemize}

\noindent
The function \code{atanh}
\begin{itemize}
  \item When \varname{n} is $-1$ returns \code{negative-infinity}
  \item When \varname{n} is $-0.0$ returns $-0.0$
  \item When \varname{n} is $1$ returns \code{positive-infinity}
  \item When \varname{n} is a complex number and  \varname{x} and \varname{y}
  are its real and imaginary parts has the behaviour and the exceptional
  situations derived from the relation: \code{(atanh }\varname{n}\code{)}
  $\equiv$ \code{(* (complex 0 1) (atan (complex }\varname{y}\code{ (-
  }\varname{x}\code{)))}
\end{itemize}

\DExceptional{}

There are different exceptional situations to be considered, in addition to
the ones that can be derived from the relations stated above:
\begin{enumerate}
  \item If \code{sinh}, \code{cosh}, \code{tanh}, \code{asinh},
    \code{acosh} and \code{atanh} are called with
    \varname{n} being a \clieeeterm{signaling NaN}, then the
    \clname{cl:floating-point-invalid-operation} error is signaled.
  \item If \varname{n} is not \CL{}
    \clterm{number} then the functions \code{sinh}, \code{cosh}, \code{tanh},
    \code{asinh}, \code{acosh} and \code{atanh} signals a
    \clname{cl:type-error}.
\end{enumerate}

\DSeeAlso{}

\code{*}, \code{-}, \code{sin}, \code{cos}, \code{tan}, \code{asin},
\code{acos}, \code{atan}.

\end{document}