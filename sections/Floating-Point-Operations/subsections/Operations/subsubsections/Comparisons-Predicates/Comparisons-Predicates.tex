\documentclass[../../Operations.tex]{subfiles}
\begin{document}

\begin{tt}
    \begin{tabular}{lll}
      /= &  >=     & oddp\\
      <  &  evenp  & plusp\\
      <= &  max    & zerop\\
      =  &  min    & \\
      >  &  minusp & \\
    \end{tabular}
  \end{tt}
  
  \noindent
  The table above corresponds to Figure~12-3 of
  \cite{1996:ANSIHyperSpec}.  The LIA specifications require more
  operators than those listed above.
  
  
  \subfile{definitions/equals.tex}
  \subfile{definitions/great-less.tex}
  \subfile{definitions/is-neg-zero.tex}
  \subfile{definitions/is-tiny.tex}
  \subfile{definitions/signum.tex}
  \subfile{definitions/residue.tex}
  \subfile{definitions/sqrt.tex}
  \subfile{definitions/exponent-fraction.tex}
  \subfile{definitions/scale-float.tex}
  \subfile{definitions/ieee-754-not-implemented-item.tex}
  
  
  % %%%%%%%%%%%%%%%%%%%%%%%%%%%%%%%%%%%%%%%%%%%%%%%%%%%%%%%%%%%%%%%%%%%%%%%%%%%
  % \DDictionaryItem{Function \code{=}}
  % \index{*!\code{=}}
  
  % \DSyntax{}
  
  % \code{=} \varname{a}, \varname{b} \RArrow \varname{boolean}\\
  % \code{=} \varname{a} \code{\&rest} \varname{bs} \RArrow \varname{boolean}
  
  % \DArgsNValues{}
  
  % \varname{a} \varname{b} -- Numbers.\\
  % \varname{bs} -- A list of numbers.\\
  % \varname{boolean} -- a \clterm{generalized boolean}.
  
  % \DDescription{}
  
  % The dyadic version of \code{=} performs an arthimetic equality test
  % between \varname{a} and \varname{b}.  The monadic and n-adic versions are built upon
  % the dyadic one as per the regular \CL{} description in
  % \cite{1996:ANSIHyperSpec}.
  
  % It is assumed that \varname{a} and \varname{b} are converted (as per
  % the \emph{contagion rules} of \CL{}) to be of the same type.
  % Therefore the following cases can be be considered as per the LIA
  % specifications.
  
  % \begin{description}
  % \item If \varname{a} and \varname{b} are either finite integers, finite
  % floating point numbers, or finite complex numbers then the result is
  % \varname{true} if the two numbers are equal in the mathematical sense.  In the
  % LIA spec this is the result of $\mathit{eq}_T(a, b) \equiv a = b$ for an
  % appropriate $T$.  This is the standard \CL{} case.
  
  % \item If \varname {a} and \varname {b} are \clieeeterm{infinities} then
  % \code{=} returns \varname{true} if they are both positive or both
  % negative; otherwise it returns \varname{false}.
  
  % \item If either \varname {a} or \varname {b} is a \clieeeterm{quiet NaN},
  % and, respectively, \varname {b} and \varname {a} is not a
  % \clieeeterm{signaling NaN}, then the result is \varname{false}.
  
  % \item Complex numbers are checked recursively on the real and imaginary
  % parts.
  % \end{description}
  
  % \DExceptional{}
  
  % If either \varname {a} or \varname {b} is a \clieeeterm{signaling
  %   NaN}, then, under the notification NACF regime, the indicator
  % \code{:invalid} is recorded and the
  % \code{floating-point-invalid-operation} is signalled (with
  % \emph{continuation value} \code{NIL} recorded); otherwise, under the
  % NRI notification regine, the indicator \code{invalid} is recorded and
  % \code{NIL} (\varname{false}) is returned as \emph{continuation value}.
  
  % For complex numbers, the recording and signaling operations (under NRI
  % and NACF) happens if the condition above applied to either of the real
  % or the imaginary parts of \varname{a} and \varname{b}.
  
  
  % %%%%%%%%%%%%%%%%%%%%%%%%%%%%%%%%%%%%%%%%%%%%%%%%%%%%%%%%%%%%%%%%%%%%%%%%%%%
  % \DDictionaryItem{Function \code{/=}}
  % \index{*!\code{/=}}
  
  % \DSyntax{}
  
  % \code{/=} \varname{a}, \varname{b} \RArrow \varname{boolean}\\
  % \code{/=} \varname{a} \code{\&rest} \varname{bs} \RArrow \varname{boolean}
  
  % \DArgsNValues{}
  
  % \varname{a} \varname{b} -- Numbers.\\
  % \varname{bs} -- A list of numbers.\\
  % \varname{boolean} -- a \clterm{generalized boolean}.
  
  % \DDescription{}
  
  % The dyadic version of \code{/=} performs an arthimetic equality test
  % between \varname{a} and \varname{b}.  The monadic and n-adic versions are built upon
  % the dyadic one as per the regular \CL{} description in
  % \cite{1996:ANSIHyperSpec}.
  
  % It is assumed that \varname{a} and \varname{b} are converted (as per
  % the \emph{contagion rules} of \CL{}) to be of the same type.
  % Therefore the following cases can be be considered as per the LIA
  % specifications.
  
  % If \varname{a} and \varname{b} are either finite integers, finite
  % floating point numbers, or finite complex numbers then the result is
  % \varname{true} if the two numbers are equal in the mathematical sense.  In the
  % LIA spec this is the result of $\mathit{neq}_T(a, b) \equiv a \neq b$ for an
  % appropriate $T$.  This is the standard \CL{} case.
  
  % If \varname {a} and \varname {b} are \clieeeterm{infinities} then
  % \code{/=} returns \varname{false} if they are both positive or both
  % negative; otherwise it returns \varname{true}.
  
  % If either \varname {a} or \varname {b} is a \clieeeterm{quiet NaN},
  % and, respectively, \varname {b} and \varname {a} is not a
  % \clieeeterm{signaling NaN}, then the result is \varname{false}.
  
  % Complex numbers are checked recursively on the real and imaginary parts.
  
  % \DExceptional{}
  
  % If either \varname {a} or \varname {b} is a \clieeeterm{signaling
  %   NaN}, then, under the notification NACF regime, the indicator
  % \code{:invalid} is recorded and the
  % \code{floating-point-invalid-operation} is signalled (with
  % \emph{continuation value} \code{NIL} recorded); otherwise, under the
  % NRI notification regine, the indicator \code{invalid} is recorded and
  % \code{NIL} (\varname{false}) is returned as \emph{continuation value}.
  
  % For complex numbers, the recording and signaling operations (under NRI
  % and NACF) happens if the condition above applied to either of the real
  % or the imaginary parts of \varname{a} and \varname{b}.
  
  
  %%%%%%%%%%%%%%%%%%%%%%%%%%%%%%%%%%%%%%%%%%%%%%%%%%%%%%%%%%%%%%%%%%%%%%%%%%%
  

  
  
  %%%%%%%%%%%%%%%%%%%%%%%%%%%%%%%%%%%%%%%%%%%%%%%%%%%%%%%%%%%%%%%%%%%%%%%%%%%
  
  
  %%%%%%%%%%%%%%%%%%%%%%%%%%%%%%%%%%%%%%%%%%%%%%%%%%%%%%%%%%%%%%%%%%%%%%%%%%%
  
  
  
  
  
  % \vspace*{3mm}
  
  % \noindent
  % Note that the \code{evenp} and \code{oddp} functions are not present
  % in the above table,  which corresponds to Figure~12-3 of
  % \cite{1996:ANSIHyperSpec}.
  
  % \paragraph{Correspondances with \IEEEFPStd{}.} The list of operations
  % recommended by \cite{2008:IEEE-754} is more extensive than the list of
  % operations provided by \CL{} (cfr., Table~9.1 in \cite{2008:IEEE-754}).
  % The tables referenced below provide correspondances for the \CL{} functions,
  % especially regarding the exceptions (conditions) that must be
  % signalled.
  
  % \begin{table}[h]
  %   \begin{tabulary}{\textwidth}{|L|L|L|}
  %     \hline
  %     \CL{} Function & IEEE-745 Function & Exceptional Situations\\
  %     \hline\hline
  %     \multicolumn{3}{|l|}{Arithmetic}\\\hline
  %     \code{+}
  %     & \textit{F\_\textbf{addition}}(\varname{a}, \varname{b})
  %     & Long list of cases
  %     \\\hline
      
  %   \end{tabulary}
  % \end{table}
  
  
  %%%%%%%%%%%%%%%%%%%%%%%%%%%%%%%%%%%%%%%%%%%%%%%%%%%%%%%%%%%%%%%%%%%%%%%%%%%
  
  
\end{document}