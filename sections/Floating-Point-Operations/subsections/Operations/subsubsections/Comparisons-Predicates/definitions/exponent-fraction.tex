%%%% -*- Mode: LaTeX -*-

%%%% exponent-fraction.tex

\documentclass[../Comparisons-Predicates.tex]{subfiles}
\begin{document}

\DDictionaryItem{Functions \code{exponent}, \code{fraction}}
\index{E!\code{exponent}}
\index{F!\code{fraction}}

\DSyntax{}

\code{exponent} \varname{f} \RArrow{} \varname{exp}\\
\code{fraction} \varname{f} \RArrow{} \varname{frac}\\


\DArgsNValues{}

\varname{f} -- A \clterm{float}.\\
\varname{exp} -- An \clterm{integer}.\\
\varname{frac} -- A \clterm{float}.

\DDescription{}

The \code{exponent} and \code{fraction} functions extract parts of the
representation of a floating point number.  The two functions behave
as if defined (for finite floating point numbers) as follows:

\begin{alltt}
  (defun \codelia{fraction} (f) (nth-value 1 (decode-float f)))
  
  (defun \codelia{exponent} (f) (nth-value 0 (decode-float f)))
  
  \textcolor{red}{;;; NTH-VALUE is used in both functions just for symmetry.}
\end{alltt}

If \varname{f} is an \clliaterm{infinity} or $0$, then \code{fraction}
returns \varname{f} as result \varname{frac}; \code{exponent}
returns $\infty$ if \varname{f} is an \clliaterm{infinity}.

If \varname{f} is a \clliaterm{quiet NaN} then, \code{exponent}
returns a \clliaterm{quiet NaN}.  If \varname{f} is a \clliaterm{quiet
  NaN} then, \code{exponentfraction} returns a \clliaterm{quiet NaN}.

\DExceptional{}

If \varname{f} is a \clliaterm{signaling NaN} then, if the notification
style is NACF, the condition\\
\code{floating-point-invalid-operation} is signaled with a
\clliaterm{quiet NaN} as continuation value; if the notification style
is NRI, the indicator \code{:invalid} is recorded and a
\clliaterm{quiet NaN} is returned as continuation value.

If \varname{f} is $0$ then \code{exponent} signals a
\code{floating-point-infinitary-operation} with a negative
\clliaterm{infinite} ($-\infty$) as continuation value If the
notification style is NACF; otherwise, if the notification style is
NRI, the \code{:infinitary} indicator is recorded and a negative
\clliaterm{infinite} ($-\infty$) is returned as continuation value.

\end{document}