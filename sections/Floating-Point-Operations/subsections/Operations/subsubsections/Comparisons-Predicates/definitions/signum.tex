\documentclass[../Comparisons-Predicates.tex]{subfiles}
\begin{document}

\DDictionaryItem{Functions \code{signum}, \code{float-sign}}
  \index{S!\code{signum}}
  \index{F!\code{float-sign}}
  
  
  \DSyntax{}
  
  \code{signum} \varname{x} \RArrow \varname{signed-prototype}\\
  \code{float-sign} \varname{f1} \code{\&optional} \varname{f2}
  \RArrow \varname{signed-float}
  
  
  \DArgsNValues{}
  
  \varname{x} -- a \clterm{number}\\
  \varname{signed-prototype} -- a \clterm{number}\\
  \varname{f1} -- a \clterm{float}\\
  \varname{f2} -- a \clterm{float}\\
  \varname{signed-float} -- a \clterm{float}
  
  
  \DDescription{}
  
  The functions \code{signum} and \code{float-sign} behave as per the
  the \CL{} Standard \cite{1994:ANSICL} in the cases where the arguments
  \varname{x}, \varname{f1} and \varname{f2} are as in the standard case.
  
  The \code{signum} function returns $1$ if \varname{x} is a positive
  \clliaterm{infinity}, and  $-1$ if \varname{x} is a negative
  \clliaterm{infinity}.
  
  If \varname{x} is a \clterm{complex} number, then the result
  \varname{signed-prototype} is also a \clterm{complex number}, and its
  value is determined according to the specification set forth in LIA3
  \cite{2004:LIA3}.
  
  If \varname{x} is a floating point \clliaterm{quiet NaN} then
  \code{signum} returns a \clliaterm{quiet NaN}.
  
  The function \code{float-sign} returns an \clliaterm{infinity} of the
  appropriate sign if \varname{f2} is an \clliaterm{infinity}.  If
  either \varname{f1} or \varname{f2} is a \clliaterm{quiet NaN} then
  \code{float-sign} returns a \clliaterm{quiet NaN}.
  
  \DExceptional{}
  
  If \varname{x} is a floating point \clliaterm{signaling NaN} and the
  notification style is NACF, then a\\
  \code{floating-point-invalid} condition is signaled with a
  \clliaterm{quiet NaN} as a \emph{continuation value}; if the
  notification style is NRI then the indicator \code{:invalid} is
  recorder and a \clliaterm{quiet NaN} is returned.
  
  A \code{type-error} is signaled if \varname{x} is not a number or
  either \varname{f1}, or \varname{f2} is not a \clterm{float} number.
  
  
  \DNotes{}
  
  The specification of \code{signum} is substantially different from the
  one present in \cite{1994:ANSICL}.
  
\end{document}