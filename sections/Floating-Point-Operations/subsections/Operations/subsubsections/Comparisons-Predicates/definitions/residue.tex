%%%% -*- Mode: LaTeX -*-

%%%% residue.tex

\documentclass[../Comparisons-Predicates.tex]{subfiles}
\begin{document}

\DDictionaryItem{Function \code{residue}}
  \index{R!\code{residue}}
  
  \DSyntax{}
  
  \code{residue} \varname{x} \varname {y} \RArrow \varname{rm}
  
  \DArgsNValues{}
  
  \varname{x} -- a \clterm{number}\\
  \varname{y} -- a \clterm{number}\\
  \varname{rm} -- a \clterm{number}
  
  
  \DDescription{}
  
  The \code{residue} function returns the \clliaterm{remainder}
  \varname{rm} of a the division between \varname{x} and \varname{y}.
  
  When \varname{y} is an \clliaterm{infinite} number, than \varname{rm}
  is \varname{y} itself.  If either \varname{x} or \varname{y} is a
  \clliaterm{quiet NaN} then \varname{rf} is a \clliaterm{quite NaN}.
  
  If either \varname{x} or \varname{y} is a \clterm{complex number}
  then, as per LIA3 \cite{2004:LIA3}, the real part and the imaginary
  part must be \clterm{integer numbers} and the result \varname{rm} is
  computed as per LIA3.
  
  
  \DExceptional{}
  
  If either \varname{x} or \varname{y} is a \clliaterm{signaling NaN}
  then if the notification style is NACF then a\\
  \code{floating-point-invalid-operation} is signaled with a
  \clliaterm{quiet NaN} as continuation value.  If the notification
  style is NRI then the \code{:invalid} indicator is recorded and a
  \clliaterm{quiet NaN} is returned as continuation value.
  
  A \code{type-error} is signaled if either \varname{x} or \varname{y} is
  not a \clliaterm{number}.
  
\end{document}