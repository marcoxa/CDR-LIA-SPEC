%%%% -*- Mode: LaTeX -*-

%%%% is-tiny.tex

\documentclass[../Comparisons-Predicates.tex]{subfiles}
\begin{document}

\DDictionaryItem{Functions \code{is-tiny}, \code{tiny-p}}
  \index{I!\code{is-tiny}}
  \index{T!\code{tiny-p}}
  
  
  \DSyntax{}
  
  \code{is-tiny} \varname{x} \RArrow \varname{boolean}\\
  \code{tiny-p} \varname{x} \RArrow \varname{boolean}
  
  \DArgsNValues{}
  
  \varname{x} -- A \clterm{real}.\\
  \varname{boolean} -- A \clterm{generalized boolean}.
  
  
  \DDescription{}
  
  These functions check whether \varname{x} is a \emph{tiny}
  \clterm{real} close to  $0$.
  
  \DExceptional{}
  
  If \varname{x} is a \clieeeterm{NaN} then, under NCAF regime, the
  function signals a\\
  \code{floating-point-invalid-operation} with a
  \code{NIL} \clieeeterm{continuation value}; under NRI, the
  \code{:invalid} indicator is recorded and the \clieeeterm{continuation
    value} \code{NIL} is returned.
  
  The function signals a \code{type-error} if the argument \varname{x}
  is not a \clterm{real}.
  
  \DNotes{}
  
  It is very difficult to correctly specify the behavior of this function
  in terms of \CL{} usual setup and assumptions.  Most implementations
  may simply leave this \clliaterm{not-implemented}.
  
\end{document}