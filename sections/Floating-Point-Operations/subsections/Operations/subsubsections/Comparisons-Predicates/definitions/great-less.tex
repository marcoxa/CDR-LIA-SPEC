\documentclass[../Comparisons-Predicates.tex]{subfiles}
\begin{document}

\DDictionaryItem{Functions \code{<}, \code{<=}, \code{>}, \code{>=}}
  \index{*!\code{<}}
  \index{*!\code{<=}}
  \index{*!\code{>}}
  \index{*!\code{>=}}
  
  \DSyntax{}
  
  \code{<} \varname{a}, \varname{b} \RArrow \varname{boolean}\\
  \code{<} \varname{a} \code{\&rest} \varname{bs} \RArrow \varname{boolean}\\
  \code{<=} \varname{a}, \varname{b} \RArrow \varname{boolean}\\
  \code{<=} \varname{a} \code{\&rest} \varname{bs} \RArrow \varname{boolean}\\
  \code{>} \varname{a}, \varname{b} \RArrow \varname{boolean}\\
  \code{>} \varname{a} \code{\&rest} \varname{bs} \RArrow \varname{boolean}\\
  \code{>=} \varname{a}, \varname{b} \RArrow \varname{boolean}\\
  \code{>=} \varname{a} \code{\&rest} \varname{bs} \RArrow \varname{boolean}
  
  \DArgsNValues{}
  
  \varname{a} \varname{b} -- Numbers.\\
  \varname{bs} -- A list of numbers.\\
  \varname{boolean} -- a \clterm{generalized boolean}.
  
  \DDescription{}
  
  The dyadic version of \code{<}, \code{<=}, \code{>} and \code{>=}
  perform arithmetic ordering tests between \varname{a} and
  \varname{b}.  The monadic and n-adic versions are built upon the
  dyadic one as per the regular \CL{} description in
  \cite{1996:ANSIHyperSpec}.
  
  It is assumed that \varname{a} and \varname{b} are converted (as per
  the \emph{contagion rules} of \CL{}) to be of the same type.
  Therefore the following cases can be be considered as per the LIA
  specifications.
  
  \begin{description}
  \item If \varname{a} and \varname{b} are either finite reals, then the result is
  \varname{true} (respectively, \varname{false}) if \varname{a} is
  less or less than or equal (respectively, greater or greater or equal)
  than \varname{b} in the mathematical sense.  In the
  LIA specification this is the result of
  $\mathit{lss}_T(a, b) \equiv a < b$,
  $\mathit{leq}_T(a, b) \equiv a \leq b$
  $\mathit{gtr}_T(a, b) \equiv a > b$, or
  $\mathit{geq}_T(a, b) \equiv a \geq b$
  for an
  appropriate $T$.  This is the standard \CL{} case.
  
  \item If \varname {a} and \varname {b} are \clieeeterm{infinities} then
  \code{=} returns \varname{true} (respectively \varname{false}) if they
  are both positive or both negative; otherwise it returns
  \varname{false} (respectively \varname{true}).
  
  \item If either \varname {a} or \varname {b} is a \clieeeterm{quiet NaN},
  and, respectively, \varname {b} and \varname {a} is not a
  \clieeeterm{signaling NaN}, then the result is \varname{false}.
  
  \item Complex numbers cannot be compared as per the \CL{}
    specification \cite{1996:ANSIHyperSpec}.
  \end{description}
  
  \DExceptional{}
  
  If either \varname {a} or \varname {b} is a \clieeeterm{signaling
    NaN}, then, under the notification NACF regime, the indicator
  \code{:invalid} is recorded and the
  \code{floating-point-invalid-operation} is signaled (with
  \emph{continuation value} \code{NIL} recorded); otherwise, under the
  NRI notification regime, the indicator \code{invalid} is recorded and
  \code{NIL} (\varname{false}) is returned as \emph{continuation value}.
  
  If either \varname{a}, or \varname{b} is not a \clterm{real} then a
  \code{type-error} is signaled.
  
\end{document}