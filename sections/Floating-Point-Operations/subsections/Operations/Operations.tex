%%%% -*- Mode: LaTeX -*-

%%%% Operations.tex

\documentclass[../Floating-Point-Operations.tex]{subfiles}
\begin{document}

The following \CL{} operations, broken down according to the
classification in Section~12.1.1 of \cite{1996:ANSIHyperSpec} are exported
form the \code{CL-MATH-IEEE-2019} package
(cfr.,~\ref{sect:package}).

If \code{is-cdr-ieee-754-operation-providing} returns non-\code{NIL}
(and the\\
\code{:cdr-ieee-754-operation-providing} is in
\code{*features*}, then the operations are also implemented, otherwise
each of them signal an ``not implemented'' error;\\
see \code{ieee-754-not-implemented-item}.

The list of operations
recommended by \cite{2008:IEEE-754} is more extensive than the list of
operations provided by \CL{} (cfr., Table~9.1 in \cite{2008:IEEE-754}).
%
The tables referenced below provide correspondences for the \CL{} functions,
especially regarding the exceptions (conditions) that must be
signaled.

\paragraph{Rounding Modes.} In particular, all the operations listed
in Sections~\ref{sect:arith-ops} and~\ref{sect:transc-ops} respect the
\clieeeterm{rounding mode} set in the floating point
environment in effect when the operation is executed.

\paragraph{Underflow and Overflow.}  The operations listed signal
\code{cl:floating-point-overflow}\\ and
\code{cl:floating-point-underflow} according to the rules established
in \cite{2008:IEEE-754} and Section~12.1.4.3 of \cite{1996:ANSIHyperSpec}.


\subsubsection{Arithmetic Operations}
\subfile{subsubsections/Arithmetic-Operations/Arithmetic-Operations.tex}
\newpage

\subsubsection{Exponential, Logarithms and Trigonometry Operations}
\subfile{subsubsections/Exponentials-Logarithms-Trigonometry/Exponentials-Logarithms-Trigonometry.tex}
\newpage

\subsubsection{Numeric Comparison and Predicates}
\subfile{subsubsections/Comparisons-Predicates/Comparisons-Predicates.tex}


\end{document}