%%%% -*- Mode: LaTeX -*-

%%%% min-max-fixnum.tex

\documentclass[../Integers.tex]{subfiles}
\begin{document}

\DDictionaryItem{Constants \code{minimum-fixnum}, \code{maximum-fixnum}}
\index{M!\code{minimum-fixnum}}
\index{M!\code{maximum-fixnum}}

\DValues{}

The value of \code{minimum-fixnum} is \code{cl:most-negative-fixnum}.\\
The value of \code{maximum-fixnum} is \code{cl:most-positive-fixnum}.

\DDescription{}

These constants represent the minimum and maximum \code{fixnum}s
available in an implementation.

\DNotes{}

These are the parameters \textit{minint}$_I$ and \textit{maxint}$_I$,
with $I$ equal to \code{fixnum}, required in LIA1 \cite{2012:LIA1}.
Their values are the obvious \CL{} counterparts.

\DSeeAlso{}

\code{bounded-fixnum},
\code{bounded-integer},\\
%
\code{most-negative-fixnum},\\
\code{most-positive-fixnum},\\
%
\code{minimum-integer},
\code{maximum-integer}.

\end{document}