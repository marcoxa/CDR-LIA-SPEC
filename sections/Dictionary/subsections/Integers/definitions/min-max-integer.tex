\documentclass[../Integers.tex]{subfiles}
\begin{document}


\DDictionaryItem{Constants \code{minimum-integer}, \code{maximum-integer}}
\index{M!\code{minimum-integer}}
\index{M!\code{maximum-integer}}

\DValues{}

The value of \code{minimum-integer} is $-\infty$.\\
The value of \code{maximum-integer} is $+\infty$\marginnote{The values
  are, for the time being, arbitrary.  They could be made equal to the
  floating point infinities, with all the necessary consequences.}

\DDescription{}

These constants represent the minimum and maximum \code{integers},
which are usually limited by actual memory constraints.

\DNotes{}

These are the parameters \textit{minint}$_I$ and \textit{maxint}$_I$,
with $I$ equal to \code{integer}, required in LIA1 \cite{2012:LIA1}.

\DSeeAlso{}

\code{bounded-fixnum},
\code{bounded-integer},\\
%
\code{minimum-fixnum},
\code{maximum-fixnum}.

\end{document}