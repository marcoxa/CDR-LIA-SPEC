%%%% -*- Mode: LaTeX -*-

%%%% has-infinity-integer.tex

\documentclass[../Integers.tex]{subfiles}
\begin{document}

\DDictionaryItem{Constant \code{has-infinity-integer}}
\index{H!\code{has-infinity-integer}}

\DValues{}

The value is \code{NIL}.\marginnote{Or should it be \code{T}? In this
  case, there will be several consequences; how is an integer infinity
represented? What is the ``continuation value'' for a division by
zero?}

\DDescription{}

This constant indicates that the \CL{} datatype \code{integer} does not
include infinities.

\DNotes{}

This is the parameter \textit{hasinf}$_I$ with $I$ equal to
\code{integer}, required in LIA1 \cite{2012:LIA1}.

Note that this parameter does not agree with the LIA1
specification; it is \code{NIL} and not \code{T}.  The reason is that at the time
of this writing there appear to be no portable way to express an
integer infinity in most (all?) \CL{} implementations.


\DSeeAlso{}

\code{bounded-fixnum}, \code{bounded-integer}, \code{has-infinity-fixnum}.

\end{document}