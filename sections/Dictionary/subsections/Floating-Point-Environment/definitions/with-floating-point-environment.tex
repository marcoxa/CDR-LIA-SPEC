%%%% -*- Mode: LaTeX -*-

%%%% with-floating-point-environment.tex

\documentclass[../Floating-Point-Environment.tex]{subfiles}
\begin{document}

\DDictionaryItem{Macro \code{with-floating-point-environment}}
\index{W!\code{with-floating-point-environment}}

\DSyntax{}

% \code{with-floating-point-environment} (\textit{\code{\&key}}
% \varname{traps}
% \varname{rounding-mode}
% \varname{current-notifications}
% \varname{precision}
% \varname{\code{\&allow-other-keys}}) \code{\&body} \varname{body}
% $\Rightarrow$ \varname{results}

\begin{tabbing}
\code{with-floating-point-environment} \=(\=\code{\&key}\\
\>\>                                \varname{traps}\\
\>\>                                \varname{rounding-mode}\\
\>\>                                \varname{current-notifications}\\
\>\>                                \varname{precision}\\
\>\>                                \code{\&allow-other-keys})\\
\> \code{\&body} \varname{body}\\
$\Rightarrow$ \varname{results}
\end{tabbing}




\DArgsNValues{}

\varname{traps} -- A list of the exception conditions that should cause
traps.\\
\varname{rounding-mode} -- The rounding mode to use when the result is
not exact.\\
\varname{current-notifications} -- The argument is used to set the current
set of exceptions.\\
\varname{precision} -- An integer.\\
\varname{results} -- One or more \CL{} objects.


\DDescription{}

The \code{with-floating-point-environment} macro executes \varname{body} in a
an environment where the floating point modes are determined by the
values passed as arguments to the macro.  Upon termination (either
normal or exceptional) of the code in \varname{body} the floating
point modes are restored to those in effect before the execution of\\
\code{with-floating-point-environment}.

As for \code{set-floating-point-environment} the values that the arguments
can take are the following:

\begin{itemize}
\item \varname{traps} is a list that can contain the keywords
  \code{:underflow}, \code{:overflow}, \code{:inexact}, \code{:invalid},
  \code{:divide-by-zero}, and \code{:denormalized-operand}.

\item \varname{rounding-mode} is the rounding mode to use when the result is
  not exact; it can assume the values \code{:nearest},
  \code{:positive-infinity}, \code{:negative-infinity} and
  \code{:zero}.

\item \varname{current-notifications} is used to set the exception flags. The
  main use is setting the accrued exceptions to \code{NIL} to clear
  them.

\item \varname{precision} can be one of the integers 24, 53 and 64,
  standing for the internal precision of the mantissa.
\end{itemize}



\noindent
\varname{results} is the value (or values) returned by \varname{body}.


\DNotes{}

When called with an empty arguments list,
\code{with-floating-point-environment} is a no-op and \varname{body}
is executed as-is.

\DExceptional{}

The macro \code{with-floating-point-environment} performs a minimal
code-walk of \varname{body} and if it finds some floating point
operation which potentially may not respect the changed environment as
described by the specified arguments, it then issues a warning.


\DSeeAlso{}

\code{get-floating-point-environment},
\code{set-floating-point-environment}\\
\code{with-rounding-mode}.

\end{document}