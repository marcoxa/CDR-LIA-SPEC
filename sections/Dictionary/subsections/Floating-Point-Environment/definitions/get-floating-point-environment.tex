\documentclass[../Floating-Point-Environment.tex]{subfiles}
\begin{document}

\DDictionaryItem{Function \code{get-floating-point-environment}}
\index{G!\code{get-floating-point-environment}}

\DSyntax{}

\code{get-floating-point-environment} \textit{$<$no arguments$>$}
$\Rightarrow$ \varname{modes}

\DArgsNValues{}

\varname{modes} --  An object of type \code{floating-point-environment}.


\DDescription{}

The function returns an a-list that represents the current state of
the floating point modes in use at the time.  The format of the
returned \varname{modes} a-list is such to be
usable as an \code{apply} last argument for
\code{set-floating-point-environment}.


\DExamples{}

\begin{alltt}
SBCL> \codeprompt{(get-floating-point-environment)}
\textit{(:TRAPS (:OVERFLOW :INVALID :DIVIDE-BY-ZERO)
 :ROUNDING-MODE :NEAREST
 :CURRENT-NOTIFICATIONS (:INEXACT)
 :FAST-MODE NIL
 :PRECISION :53-BIT)}
\end{alltt}


\DSeeAlso{}

\code{set-floating-point-environment}.

\end{document}