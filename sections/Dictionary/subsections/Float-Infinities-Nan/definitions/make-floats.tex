%%%% -*- Mode: LaTeX -*-

%%%% make-floats.tex

\documentclass[../Float-Infinities-Nan.tex]{subfiles}
\begin{document}

\DDictionaryItem{Functions
  \code{make-short-float},
  \code{make-single-float},\\
  \code{make-double-float},
  \code{make-long-float}}
\index{M!\code{make-short-float}}
\index{M!\code{make-single-float}}
\index{M!\code{make-double-float}}
\index{M!\code{make-long-float}}

\DSyntax{}

\code{make-short-float} \varname{bytes}
$\Rightarrow$ \varname{result}\\
\code{make-single-float} \varname{bytes}
$\Rightarrow$ \varname{result}\\
\code{make-double-float} \varname{bytes}
$\Rightarrow$ \varname{result}\\
\code{make-long-float} \varname{bytes}
$\Rightarrow$ \varname{result}

\DArgsNValues{}

\varname{bytes} -- An integer or bit-vector representing the binary
pattern of a floating point number.\\
\varname{result} -- the resulting floating point number or \code{NAN}.

\DDescription{}

The functions construct a floating point number of appropriate
float type starting from the bit content of
\varname{bytes}.

If \varname{bytes} corresponds to the byte pattern of a \emph{NaN},
then a \code{NAN} is returned.

The actual float type returned by the functions is the widest one
supported by the implementation; i.e., a call to
\code{make-long-float} may return a \code{double-float}.

\DExceptional{}

The functions signals a \code{type-error} if \varname{bytes} is not of
the type described above

\DSeeAlso{}

\code{NAN}, \code{Q-NAN}, \code{S-NAN}, \code{make-float}.

\DNotes{}

These functions are, in one form or another, already present in \CL{}
implementations.

Implementations may supply a larger set of these functions, e.g.,
\code{make-quad-float}.

\end{document}