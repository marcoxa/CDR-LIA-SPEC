\documentclass[../Float-Infinities-Nan.tex]{subfiles}
\begin{document}

\DDictionaryItem{Functions \code{is-nan}, \code{nanp},\\
  \code{is-quiet-nan},
  \code{quiet-nan-p},\\
  \code{is-signaling-nan},
  \code{signaling-nan-p}
}
\index{I!\code{is-nan}}
\index{N!\code{nanp}}
\index{I!\code{is-quiet-nan}}
\index{Q!\code{quiet-nan-p}}
\index{I!\code{is-signaling-nan}} 
\index{S!\code{signaling-nan-p}}

\DSyntax{}

\code{is-nan} \varname{x} $\Rightarrow$ \textit{boolean}\\
\code{nanp} \varname{x} $\Rightarrow$ \textit{boolean}\\
\code{is-quiet-nan} \varname{x} $\Rightarrow$ \textit{boolean}\\
\code{quiet-nan-p} \varname{x} $\Rightarrow$ \textit{boolean}\\
\code{is-signaling-nan} \varname{x} $\Rightarrow$ \textit{boolean}\\
\code{signaling-nan-p} \varname{x} $\Rightarrow$ \textit{boolean}\\

\DArgsNValues{}

\varname{x} -- any \CL{} object.

\DDescription{}

The function \code{is-nan} (respectively \code{nanp} etc.) returns \code{T}
whenever \varname{x} is a (representation of an IEEE) NaN.  Otherwise
it returns \code{NIL}. The \emph{quiet} and \emph{signaling} versions
operate similarly.

\DExamples{}

\begin{alltt}
CL prompt> \codeprompt{(is-nan nan)}
\textit{T}

CL prompt> \codeprompt{(nanp nan)}
\textit{T}

CL prompt> \codeprompt{(is-nan 42)}
\textit{NIL}

CL prompt> \codeprompt{(is-nan "NaN")}
\textit{NIL}
\end{alltt}

\end{document}