%%%% -*- Mode: LaTeX -*-

%%%% nan.tex

\documentclass[../Float-Infinities-Nan.tex]{subfiles}
\begin{document}

\DDictionaryItem{Variables \code{NAN}, \code{S-NAN}, \code{Q-NAN}}
\index{N!NaNs!\code{NAN}}
\index{N!NaNs!\code{S-NAN}}
\index{N!NaNs!\code{Q-NAN}}

\DValues{}

All \emph{implementation-dependent values}.


\DDescription{}

The values \code{NAN}, \code{Q-NAN}, and \code{S-NAN} hold a
representation of a ``not a number'' object.  The object can either be
a \emph{quiet} (\code{Q-NAN}) or a \emph{signaling} (\code{S-NAN})
\emph{NaN} (see~\cite{2008:IEEE-754}).  \code{NAN} is always a
\emph{quiet NaN}.


\DExamples{}

The actual values of \code{NAN} vary from implementation to
implementation.  Here are two examples of how \code{NAN} can be
represented in two implementations:

\begin{alltt}
SBCL> \codeprompt{NAN}
\textit{#<DOUBLE-FLOAT quiet NaN>}
\textcolor{red}{;;; E.g., the result of (sb-kernel:make-double-float -524288 0)}
\end{alltt}

\begin{alltt}
LW> \codeprompt{NAN}
\textit{1D+-0} \textcolor{red}{#| 1D+-0 is double-float not-a-number |#}
\end{alltt}

\DNotes{}

\noindent
It is to be understood that testing for equality of two \code{NAN}s is
not meaningful.  Especially testing for \code{eq} or \code{eql}.

\noindent
It is also understood that, \code{(numberp nan)} should return \code{T}.

\DSeeAlso{}

\code{is-nan}, \code{nanp}, \code{is-quiet-nan}, \code{quiet-nan-p},
\code{is-signaling-nan}, \code{signaling-nan-p}.

\end{document}