\documentclass[../Rounding.tex]{subfiles}
\begin{document}

\DDictionaryItem{Function \code{set-rounding-mode}}
\index{S!\code{set-rounding-mode}}

\DSyntax{}

\code{set-rounding-mode} \varname{rounding-mode}
$\Rightarrow$ \varname{result}, \varname{success}

\DArgsNValues{}

\varname{rounding-mode} -- a \code{rounding-mode}\\
\varname{result} -- a \code{rounding-mode}\\
\varname{success} -- a boolean


\DDescription{}

The function sets the rounding mode.  If the setting of the rounding
mode is successful, then \varname{rounding-node} is returned as
\varname{result} and \varname{success} is \code{T}.  Otherwise, the
rounding mode before the the call is returned with \varname{success}
\code{NIL}.\marginnote{Should it instead signal an error?}

\end{document}