\documentclass[../Rounding.tex]{subfiles}
\begin{document}

\DDictionaryItem{Type \code{rounding-mode}}
\index{R!\code{rounding-mode}}

\DSupertypes{}

\code{rounding-mode}, \ldots, \code{T}

\DDescription{}

The \code{rounding-mode} type is defined to be:
\begin{alltt}
(member :indeterminable
        :zero
        :nearest
        :positive-infinity
        :negative-infinity)
\end{alltt}
The meaning of these values correspond to a direction of floating
point rounding.


\DNotes{}

The keywords used correspond to the \CLang{} Library \cite{2018:C18}
\textbf{\code{FLT\_ROUNDS}} values of:

\vspace*{3mm}

\begin{tabular}{rl}
  \textbf{\code{-1}} & indeterminable.\\
  \textbf{\code{0}}  & toward zero.\\
  \textbf{\code{1}}  & toward nearest.\\
  \textbf{\code{2}}  & toward positive infinity.\\
  \textbf{\code{3}}  & toward negative infinity.\\
\end{tabular}

\vspace*{3mm}

As per the \CLang{} Library standard, \CL{} implementations can extend the
type \code{rounding-modes} with other keywords representing
\emph{implementation dependent} rounding modes.

\end{document}