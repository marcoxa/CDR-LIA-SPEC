\documentclass[../Rounding.tex]{subfiles}
\begin{document}

\DDictionaryItem{Macros \code{round-to-zero}, \code{round-to-near},\\
  \code{round-upward}, \code{round-downward}}
\index{R!\code{round-to-zero}}
\index{R!\code{round-to-near}}
\index{R!\code{round-upward}}
\index{R!\code{round-downward}}

\DSyntax{}

\code{round-to-zero} \code{\&body} \varname{body}
$\Rightarrow$ \varname{results}\\
\code{round-to-near} \code{\&body} \varname{body}
$\Rightarrow$ \varname{results}\\
\code{round-upward} \code{\&body} \varname{body}
$\Rightarrow$ \varname{results}\\
\code{round-downward} \code{\&body} \varname{body}
$\Rightarrow$ \varname{results}

\DArgsNValues{}

\varname{body} -- a sequence of forms; i.e., an implicit \code{progn}.\\
\varname{results} -- the value(s) returned by \varname{body}.

\DDescription{}

The macros evaluate \varname{body} within an environment where the
rounding mode is set to to the macro's namesake.  The rounding mode is
reset to the one surrounding the macro call upon returning or raising
a condition (as in \code{unwind-protect}).

\DExceptional{}

None \marginnote{Maybe signal error if the rounding mode cannot be set?}

\end{document}