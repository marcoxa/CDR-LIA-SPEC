\documentclass[../Rounding.tex]{subfiles}
\begin{document}

\DDictionaryItem{Macro \code{with-rounding-mode}}
\index{W!\code{with-rounding-mode}}

\DSyntax{}

\code{with-rounding-mode} \code{(} \varname{rm} \code{)} \code{\&body}
\varname{body}\\
$\Rightarrow$ \varname{results}

\DArgsNValues{}

\varname{rm} -- An item of type \code{rounding-modes}.\\
\varname{body} -- An implicit \code{progn} of code.\\
\varname{results} -- The value (or values) returned by \varname{body}.

\DDescription{}

The code in \varname{body} is executed with a floating point
environment where the rounding mode is set to \varname{rm}.  The
previous rounding mode is saved before executing \varname{body} and it
is restored (as if using \code{unwind-protect}) upon exit.

\DExceptional{}

The macro \code{with-rounding-mode} performs a minimal code-walk of
\varname{body} and if it finds some floating point operation which
potentially may not respect the rounding mode \varname{rm} issues a
warning.
%
It is assumed that this warning will be raised at macro-expansion
time.

\DExamples{}

The following examples may be from two implementations behaving
differently with respect to their handling of rounding modes at the
\CL{} package level.

\begin{alltt}
CL(A) prompt> \codeprompt{(with-rounding-mode (:positive-infinity)
                  (cl:* 2 21.0))}
\textit{42.0}
\end{alltt}

\begin{alltt}
CL(B) prompt> \codeprompt{(with-rounding-mode (:positive-infinity)
                  (cl:* 2 21.0))}
\textcolor{red}{Warning:
the function CL:* may not respect the new rounding mode :POSITIVE-INFINITY.}
\textit{42.0}
\end{alltt}

\end{document}