%%%% -*- Mode: LaTeX -*-

%%%% fpe-test-notifications.tex

\documentclass[../Exception-Handling.tex]{subfiles}
\begin{document}

\DDictionaryItem{Function \code{fpe-test-notifications}}
\index{F!\code{fpe-test-notifications}}

\DSyntax{}

\code{fpe-test-notifications} \code{\&rest} \varname{excps}
$\Rightarrow$ \varname{excp-set}

\DArgsNValues{}

\varname{excps} -- A list of \code{fpe-notification} items.\\
\varname{excp-set} -- An object of type \code{fpe-notification-set}.

\DDescription{}

The function tests which of the \varname{excps} is set (i.e., whether
the corresponding flag is set in the underlying floating point
environment) and returns an object of type \code{fpe-notification-set}
with the corresponding flag set.

\DExamples{}

The following example (adapted from \cite{2018:C18}) shows how a piece of
code may decide how to \emph{handle} either
\code{floating-point-invalid-operation} or
\code{floating-point-overflow} (cfr., \cite{1996:ANSIHyperSpec}.)

\begin{alltt}
(let ((fpe-excps \textcolor{blue}{(fpe-test-notifications :overflow :invalid)}))
   (when (fpe-check-notifications :invalid)
     (signal 'cl:floating-point-invalid-operation))
   (when (fpe-check-notifications :overflow)
     (signal 'cl:floating-point-overflow))
   )
\end{alltt}

\DNotes{}

The \code{fpe-test-notifications} function accesses the current floating
point environment.  \CL{} implementations may have made different
choices about if, when, and how to signal the standard \CL{} floating
point conditions.  That is, the above example may or may not work in a
given \CL{} implementation, as the code that actually set the floating
environment exception flags may have already signaled either
\code{cl:floating-point-invalid-operation} or
\code{cl:floating-point-overflow}, and the some corresponding handling
code may have already cleared the flags.

\DSeeAlso{}

\code{fpe-notification}, \code{fpe-notification-set},
\code{fpe-check-notifications},\\
\code{cl:floating-point-invalid-operation},
\code{cl:floating-point-overflow}.

\end{document}