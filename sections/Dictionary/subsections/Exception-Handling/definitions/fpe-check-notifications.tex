%%%% -*- Mode: LaTeX -*-

%%%% fpe-check-notifications.tex

\documentclass[../Exception-Handling.tex]{subfiles}
\begin{document}

\DDictionaryItem{Function \code{fpe-check-notifications}}
\index{F!\code{fpe-check-notifications}}

\DSyntax{}

\code{fpe-check-notifications} \varname{excp-set} \code{\&rest} \varname{excps}
$\Rightarrow$ \varname{result}

\DArgsNValues{}

\varname{excp-set} -- An object of type \code{fpe-notification-set}\\
\varname{excps} -- A list of \code{fpe-notification} items.\\
\varname{result} -- A boolean.

\DDescription{}

The function checks whether \emph{all} of the \varname{excps} flags
are set in the \varname{excp-set} and returns a boolean indicating the
result.  If the \varname{excps} is empty, then the function returns
\code{NIL}.

\DExamples{}

The following example (adapted from \cite{2018:C18}) shows how a piece of
code may decide how to \code{signal} either
\code{floating-point-invalid-operation} or
\code{floating-point-overflow} (cfr., \cite{1996:ANSIHyperSpec}.)

\begin{alltt}
(let ((fpe-excps (fpe-test-notifications :overflow :invalid)))
   (when \textcolor{blue}{(fpe-check-notifications :invalid)}
     (signal 'cl:floating-point-invalid-operation))
   (when \textcolor{blue}{(fpe-check-notifications :overflow)}
     (signal 'cl:floating-point-overflow))
   )
\end{alltt}

\DNotes{}

Note that the value returned when \varname{excps} is empty is nor what
\CL{} users may expect from a function that looks like a call to
\code{(and)}.

The notes about \code{fpe-test-notifications} regarding the example above
apply also in the case of \code{fpe-check-notifications}.


\DSeeAlso{}

\code{fpe-notification}, \code{fpe-notification-set},
\code{fpe-test-notifications},\\
\code{cl:floating-point-invalid-operation},
\code{cl:floating-point-overflow}.

\end{document}