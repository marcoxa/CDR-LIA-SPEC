%%%% -*- Mode: LaTeX -*-

%%%% supported-notification.tex

\documentclass[../Exception-Handling.tex]{subfiles}
\begin{document}

\DDictionaryItem{Functions \code{supported-notification-p},\\
  \code{all-supported-notifications-p},
  \code{some-supported-notifications-p}}
\index{S!\code{supported-notification-p}}
\index{A!\code{all-supported-notifications-p}}
\index{S!\code{some-supported-notifications-p}}

\DSyntax{}

\code{supported-notification-p}
\varname{exception} $\Rightarrow$ \textit{boolean}\\
\code{all-supported-notifications-p} \code{\&rest}
\varname{exceptions} $\Rightarrow$ \textit{boolean}\\
\code{some-supported-notifications-p} \code{\&rest}
\varname{exceptions} $\Rightarrow$ \textit{boolean}

\DArgsNValues{}

\varname{exception} -- a \CL{} object of type 
\code{fpe-notification}.\\
\varname{exceptions} -- a list of \CL{} objects each of type 
\code{fpe-notification}.


\DDescription{}

The functions return a true value if the exceptions passed as
arguments are supported by the implementation.

The function \code{supported-notification-p} returns true if 
\varname{exception} is supported by the implementation, and \code{NIL}
otherwise.

The function \code{all-supported-notifications-p} returns true if all the
elements in \varname{exceptions} are supported by the implementation,
and \code{NIL} otherwise.

The function \code{some-supported-notifications-p} returns true if any the
elements in \varname{exceptions} is supported by the implementation,
and \code{NIL} otherwise.

\DExamples{}

\begin{alltt}
CL prompt> \codeprompt{(supported-notification-p :divide-by-zero)}
\textit{T} \textcolor{red}{; Or could be be NIL.}

CL prompt> \codeprompt{(some-supported-notifications-p :divide-by-zero :inexact)}
\textit{T} \textcolor{red}{; Assuming the previous operation returned true.}

CL prompt> \codeprompt{(all-supported-notification-p :divide-by-zero :inexact)}
\textit{NIL} \textcolor{red}{; Or could be be T.}
\end{alltt}

\DExceptional{}

The functions signal a \code{type-error} if \varname{exception}
is not of type \code{fpe-notification} or if \varname{exceptions} contains
objects not of type \code{fpe-notification}.

\end{document}