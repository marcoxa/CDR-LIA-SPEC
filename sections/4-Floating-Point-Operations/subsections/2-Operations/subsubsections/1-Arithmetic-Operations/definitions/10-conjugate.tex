\documentclass[../Arithmetic-Operations.tex]{subfiles}
\begin{document}

\DDictionaryItem{Function \code{conjugate}}
\index{C!\code{conjugate}}

\DSyntax{}

\code{conjugate} \varname{n} \RArrow{} \varname{c}

\DArgsNValues{}

\varname{n}, \varname{c} -- \CL{} \clterm{number}s.

\DDescription{}

When the value of \code{(realpart} \varname{n}\code{)} and
\code{(imagpart} \varname{n}\code{)} are not
\clieeeterm{NaNs} or \clieeeterm{infinity} the function
\code{conjugate} assume the usual behavior specified for \CL{} (cfr.
Section~12.2 of \cite{1996:ANSIHyperSpec}).

When \code{conjugate} is called with \varname{n} being a
\clieeeterm{quiet NaN} then \varname{c} is a
\clieeeterm{quiet NaN}.

When \code{conjugate} is called with \code{(realpart}
\varname{n}\code{)}  being a \clieeeterm{positive infinity} (or,
vice-versa \clieeeterm{negative infinity}) then 
\code{(realpart} \varname{c}\code{)} is a
\clieeeterm{positive infinity} (or, vice-versa \clieeeterm{negative
  infinity}).

When \code{conjugate} is called with \code{(imagpart}
\varname{n}\code{)}  being a \clieeeterm{positive infinity} (or,
vice-versa \clieeeterm{negative infinity}) then 
\code{(imagpart} \varname{c}\code{)} is a
\clieeeterm{negative infinity} (or, vice-versa \clieeeterm{positive
infinity}).

\DExceptional{}


There are different exceptional situations to be considered.

\begin{enumerate}
\item When \code{conjugate} is called with 
  \varname{n} being a \emph{signaling NaN}, then the
  \clname{cl:floating-point-invalid-operation} error is signaled.

\item If \varname{n} is not a \CL{} \clterm{number}
  then the \code{conjugate}
  might signal a \clname{cl:type-error}.
\end{enumerate}

\DSeeAlso{}

\code{-}

\end{document}