\documentclass[../Comparisons-Predicates.tex]{subfiles}
\begin{document}

\DDictionaryItem{Functions \code{sqrt}, \code{sqrt.<>}, \code{sqrt.<},
    \code{sqrt.>}}
  \index{S!\code{sqrt}}
  \index{S!\code{sqrt.<>}}
  \index{S!\code{sqrt.<}}
  \index{S!\code{sqrt.>}}
  
  
  \DSyntax{}
  
  \code{sqrt} \varname{n} \RArrow{} \varname{root}\\
  \code{sqrt.<>} \varname{n} \RArrow{} \varname{root}\\
  \code{sqrt.<} \varname{n} \RArrow{} \varname{root}\\
  \code{sqrt.>} \varname{n} \RArrow{} \varname{root}
  
  \DArgsNValues{}
  
  \varname{n} -- A \clterm{number}.\\
  \varname{root} -- A \clterm{number}.
  
  \DDescription{}
  
  The functions, \code{sqrt}, \code{sqrt.<>}, \code{sqrt.<}, and
  \code{sqrt.>} compute the \emph{square root} of \varname{n}.  They
  behave like \code{cl:sqrt} on well behaved \varname{n} values.
  \code{sqrt.<>} always computes \varname{root} rounding to nearest;
  \code{sqrt.<} always computes \varname{root} rounding downward;
  \code{sqrt.>} always computes \varname{root} rounding upward. Instead
  \code{sqrt} computes \varname{root} according to the current rounding
  mode.
  
  If \varname{n} is a \clliaterm{quiet NaN} then \varname{root} is also
  a \clliaterm{quiet NaN}.
  
  
  \DExceptional{}
  
  If \varname{} is a \clliaterm{signaling NaN} then if the notification
  style is NACF then a\\
  \code{floating-point-invalid-operation} is
  signaled, with a \clliaterm{quiet NaN} as a continuation value.  If
  the notification style is NRI then the \code{:invalid} indicator is
  recorded and a \clliaterm{quiet NaN} is returned as continuation
  value.
  
  
  A \code{type-error} is signaled if \varname{n} is not a number
  
  \DNotes{}
  
  The LIA specification suggests to call \code{sqrt.<} and \code{sqrt.>}
  as \code{sqrtDwn} and \code{sqrtUp} (cfr., LIA1
  \cite{2012:LIA1}). This suggestion goes against the traditional \CL{}
  naming verbosity ``feature'', plus it assumes case-sensitivity, which
  \CL{} does not have in default mode.  Therefore the more evocative
  names \code{sqrt.<} and \code{sqrt.>} (and \code{sqrt.<>}) are
  introduced.
  
\end{document}