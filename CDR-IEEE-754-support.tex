%%%% -*- Mode: LaTeX -*-

%%%% CDR-IEEE-754-support.tex --
%%%% Minimalistic support for (a subset) of IEEE 754 for Common Lisp.


\documentclass[10pt,fleqn]{article}

\usepackage{latexsym}
\usepackage{epsfig}
\usepackage{alltt}
\usepackage{tabulary}
\usepackage{color}
\usepackage[margin=4cm]{geometry}

% \usepackage[OT1]{fontenc}
% \usepackage{bold-extra}
% \renewcommand{\ttdefault}{cmt}

\usepackage{lmodern}            % This works for decent bold 'tt'.


\newcommand{\tm}{$^\mathsf{tm}$}
\newcommand{\cfr}{\emph{cfr.}}

\newcommand{\Lisp}{\textsf{Lisp}}
\newcommand{\CL}{\textsf{Common~Lisp}}
\newcommand{\Quicklisp}{\textsf{Quicklisp}}

\newcommand{\CLang}{\textsf{C}}
\newcommand{\Java}{\textsc{Java}}

\newcommand{\checkcite}[1]{{\textbf{[Missing Citation: #1]}}}
\newcommand{\checkref}[1]{{\textbf{[Missing Reference: #1]}}}

\newcommand{\missingpart}[1]{{\ }\vspace{2mm}\\
{\textbf{[Still Missing: #1]}}\\
\vspace{2mm}}

\newcommand{\marginnote}[1]{%
\marginpar{\begin{small}\begin{em}
{\raggedright #1}
\end{em}\end{small}}}

\newcommand{\undecided}[2]{%
  \vspace*{3mm}\noindent\fbox{\textbf{TBD: #1}{\ }\newline\emph{#2}}}



%%% CL 

\newcommand{\code}[1]{\texttt{#1}}

\newcommand{\term}[1]{\texttt{#1}}
\newcommand{\nonterm}[1]{\textit{$<$#1$>$}}


\newcommand{\kwd}[1]{\texttt{:#1}}

\newcommand{\clieeeterm}[1]{\textit{#1}}

\newcommand{\varname}[1]{\textit{#1}}

\newcommand{\clterm}[1]{\textit{#1}}
\newcommand{\clname}[1]{\texttt{#1}}

\newcommand{\codeprompt}[1]{\textcolor{blue}{\textbf{#1}}}

\newcommand{\RArrow}{$\Rightarrow$}


%%% Useful Names.

\newcommand{\IEEEFPStd}{IEEE-754}
\newcommand{\IECFPStd}{IEC-60559}


%%% ANSI-Spec Like Macros.

% \newcommand{\DDictionaryItem}[1]{\subsection{#1}\vspace*{-9pt}\hrulefill}
\newcommand{\DDictionaryItem}[1]{\vspace*{6pt}\noindent\hrulefill\vspace*{-9pt}\subsection*{#1}}

\newcommand{\DSyntax}{\subsubsection*{Syntax:}}
\newcommand{\DSupertypes}{\subsubsection*{Supertypes:}}
\newcommand{\DArgsNValues}{\subsubsection*{Arguments and Values:}}
\newcommand{\DDescription}{\subsubsection*{Description:}}
\newcommand{\DExamples}{\subsubsection*{Examples:}}
\newcommand{\DExceptional}{\subsubsection*{Exceptional Situations:}}
\newcommand{\DNotes}{\subsubsection*{Notes:}}
\newcommand{\DSeeAlso}{\subsubsection*{See Also:}}



%%%% Document Title.


\title{
\LARGE{\bfseries A Palimpsest of ``Language Independent Arithmetic'' in \CL{}}}

\author{
  Marco Antoniotti\\
  Dipartimento di Informatica, Sistemistica e Comunicazione\\
  Universit\`{a} degli Studi di Milano Bicocca\\
  Viale Sarca 336, U14, Milan (MI), \textsc{Italy}\\
  \vspace{2mm}
  \texttt{marco.antoniotti} at \texttt{unimib.it},
  \texttt{mantoniotti} at \texttt{common-lisp.net}}


% \author{TBD}

%\date{}


%\includeonly{}

\begin{document}

\maketitle

\begin{abstract}
  This document presents a set of names, functions and macros that aim
  at providing a common ground for the support of the \IEEEFPStd{}
  \cite{2008:IEEE-754} and of the \emph{Language Independent
    Arithmetic} (LIA) standards \cite{2012:LIA1,2001:LIA2,2004:LIA3}
  in \CL{}.

  The set of names and functions (and ancillary information) is
  intended to be minimalistic; essentially for \IEEEFPStd{} constants
  (e.g., \code{NaN}) and some functionalities that are needed to
  control floating point computations.
\end{abstract}

\newpage

\tableofcontents

\newpage

\section{Introduction}

The ANSI \CL{} Specification \cite{1996:ANSIHyperSpec} hints at possible
compliance with IEEE ``Floating Points'' (in retrospect, \IEEEFPStd{}) in
the description of the \code{*features*} variable, where:
\begin{description}
\item[\code{:ieee-floating-point}]
  If present, indicates that the
  implementation \emph{purports to conform}  to the requirements of
  \emph{IEEE Standard for Binary Floating-Point Arithmetic}. 
\end{description}

After the publication of the \CL{} standard \cite{1994:ANSICL}, other
standards were published by IEEE and ISO/IEC specifying the
Application Programming Interface (API) for, by then well establishes,
commonly used arithmetic and mathematical operations.  These
specifications are the \emph{IEEE Standard for Floating-Point
  Arithmetic} (\IEEEFPStd{}) \cite{2008:IEEE-754} specification (which
will be referred as \IEEEFPStd{} in this text) and the \emph{ISO/IEC
  Information technology -- Language independent arithmetic}
\cite{2012:LIA1,2001:LIA2,2004:LIA3} (in three parts, which will be
referred as LIA1, LIA2 and LIA3, or simply LIA, in this text).

\vspace*{3mm}

\noindent
Alas, the actual state of ``compliance'' with these specifications,
among different \CL{} implementations varies wildly, especially the
LIA ones, with some implementations publicizing the
\code{:ieee-floating-point} feature while providing an unclear subset
of functionalities of features\footnote{We speak of the sin and not of
  the sinner; you can check for yourself what happens in the various
  implementations.}.  This is in contrast with the current state of
affairs in other languages and language ecosystems; especially
languages designed and built \emph{after} the publication of the
\IEEEFPStd{} and LIA documents.

\vspace*{3mm}

The goal of this document is to address some of this state of affairs
by providing a specification for a minimalistic subset of
names,\marginnote{Not very ``minimalistic'' anymore\ldots}
functions and macros to be used in conjunction with \IEEEFPStd{}
concepts.  The following issues will especially 
be addressed:
\begin{itemize}
\item Names for \code{NaN}s and ``infinities''.
  
\item Functions and macros to handle with the underlying
  \emph{floating point environment} and the interplay between \IEEEFPStd{}
  notion of \emph{signalling} vs.~the \CL{} notion of
  \clterm{condition signalling}.
  
\item Functions and macros to handle \emph{rounding modes}; with
  special care devoted to incentivate non-invasive, yet informative,
  implementations.
  
\item Clarifying the behavior of the \CL{} set of mathematical
  functions (cfr.,~Section~12.2 of \cite{1996:ANSIHyperSpec}) with respect
  to the specification contained herein.

\item Providing a clear set of functionalities to handle
  \emph{exceptional situations} and a clear definition of the behavior
  of each operator and function with respect the \IEEEFPStd{} and LIA
  specifications.
\end{itemize}

\vspace*{3mm}

The content of this document borrows ideas from the \CLang{}
specification \cite{2018:C18}, and the documentation of
several \CL{} implementations.


\subsection{Impact on Current \CL{} Implementations}

It is understood that different \CL{} implementations already have
made some assumptions about the \emph{floating point environment} they
are dealing with.  In particular, the implementation of several \CL{}
math functions may be sensitive to the
setting of \emph{rounding modes}.  E.g., it is quite possible that all
\CL{} implementations quietly assume a \emph{round to nearest} mode in
the implementation of the \CL{} standard floating point operations.
Another example is the treatment of comparison operators with respect
to NaNs, as depicted by the example below\footnote{There are ways to
  tell SBCL how to construct a NaN.}.

\vspace*{3mm}

\noindent
SBCL signals a \code{CL:FLOATING-POINT-INEXACT} condition on a
comparison involving \emph{quiet} NaNs.
\begin{alltt}
SBCL> \codeprompt{(< 42 quiet-nan)}
debugger invoked on a FLOATING-POINT-INEXACT:
  arithmetic error FLOATING-POINT-INEXACT signalled

Type HELP for debugger help, or (SB-EXT:EXIT) to exit from SBCL.

restarts (invokable by number or by possibly-abbreviated name):
  0: [ABORT] Exit debugger, returning to top level.

(SB-KERNEL:TWO-ARG-< 42 #<DOUBLE-FLOAT quiet NaN>)
SBCL 0]
\end{alltt}

\vspace*{3mm}

\noindent
Lispworks just returns \code{NIL}.
\begin{alltt}
LW> \codeprompt{(< 42 1D+-0)} \textcolor{red}{; LW uses this notation to denote NaNs.}
\textit{NIL}
\end{alltt}

\vspace*{3mm}

\noindent
CCL signals a \code{CL:FLOATING-POINT-INVALID-OPERATION} condition on a
comparison involving NaNs; but in this case it is unknown whether
\code{1D+-0} represents a \emph{quiet} or \emph{signalling} NaN.
\begin{alltt}
CCL> \codeprompt{(< 42 1D+-0)} \textcolor{red}{; CCL uses a notation similar to LW.}
> Error: FLOATING-POINT-INVALID-OPERATION detected
> While executing: CCL::FIXNUM-DFLOAT-COMPARE, in process Listener(4).
> Type cmd-. to abort, cmd-\ for a list of available restarts.
> Type :? for other options.
CCL 1 > 
\end{alltt}

As a consequence, this specification takes care not to impose
constraints on programs already running, or to require implementations to
actually adapt their cores to make the \CL{} package math functions in
order to comply.

Therefore, in order to be self-contained, this specification provides
facilities (described below) that allow a \CL{} programmer to actually
decide how to use \IEEEFPStd{} compliant code.  Most important, and
possibly most annoying, this specification must define \emph{symbols}
that corresponds to (some of) the operations described in
\cite{2008:IEEE-754}, in order to provide the programmer with some
certainty about the behavior of floating point operations.

All in all, this specification is grafting onto the \CL{}
specification some functionality which was fully defined after the
publication of \cite{1996:ANSIHyperSpec}.



\section{Description}

The \IEEEFPStd{} and LIA standards introduce a number of ``names'' for
certain special values: \code{NaN}s and infinities, as an example.
Several \CL{} implementations provide such concepts; alas, not in a
consensual way.  Moreover, the standards introduce ways to handle
\emph{exceptional situations}. At the time of this writing, the \CL{}
programmer does not have much control over \emph{how} to exploit the
richness of the specifications.

The goal of this document is to provide a minimal
consensus to give the \CL{} programmer ways to be able to work with
facilities in line with the \IEEEFPStd{} (\IECFPStd{}) and LIA
standards.  The specifications contained in this document are as
\textsf{common-lisp}-ish as possibile, re-using as much the style and
naming conventions established in the ANSI~\CL{} Specification
\cite{1994:ANSICL,1996:ANSIHyperSpec}.



\subsection{\code{CL-MATH-LIA-2019} Package}
\label{sect:package}

The implementations of this specification will provide a package named
(or nicknamed) \code{CL-MATH-LIA-2019}.  All the symbols named in the rest
of the document are \code{export}ed from the above mentioned package.



\undecided{Features or APIs?}{Should \emph{features} or \emph{APIs} be
  provided to allow selective checks of this specification?}

\subsection{Floating Point Numbers}

This document contains a small number of simple facilities that appear
to be present in most \CL{} implementations or that are available as
libraries.  E.g., this document describes a \code{make-float} function
and a \code{parse-float} function, which has been available as a
separate library for quite some time in the community\footnote{The
  \code{parse-float} function and ancillary ones can be downloaded
  from \Quicklisp{} \cite{2008:Beane:Quicklisp}.}.


\subsection{Special Values}

This document provides specifications for \IEEEFPStd{}/\IECFPStd{}
\emph{special values} like \emph{infinities} (e.g.,
\code{single-float-positive-infinity}) and \emph{not-a-number}
(\code{NAN}).  This specification defines all these names.

\subsubsection{``Continuation'' and ``Exceptional'' Values}

The LIA specifications ensure that certain special values are passed
on as \emph{continuation} values downstream a computation.  This often
happens in conjunction with a \emph{notification}, that is, in
presence of an \emph{exceptional} situation (cfr.,
Section~\ref{sect:notifications}).

This specification clarifies when and \emph{how} such continuation
values are passed on -- the most common example being
\clieeeterm{quiet NaNs}.  This specification also clarifies when and
how \emph{exceptional values} are to be considered; also,
\clieeeterm{infinities} and \clieeeterm{NaN}s used as
\emph{continuation values} are not to be intended as exceptional
values (cfr., \cite{2012:LIA1} Section~4.2.3, 4.2.8, 4.2.9).



\subsection{Rounding Modes}

One of the functionalities described in \IEEEFPStd{} (\IECFPStd{}) is
the control of \emph{rounding modes}.  Rounding modes control is
important in some numerical applications and libraries.  E.g., they
are necessary to build \emph{interval arithmetic} libraires (cfr.,
\checkcite{Interval Arithmetic}).

The \CLang{} standard interface for floating point rounding modes
control (cfr., \cite{C18} Section~7.6.3) provides the \code{fgetround}
and \code{fsetround} functions that can get and set the rounding mode;
the rounding modes being defined as \code{FE\_DOWNWARD},
\code{FE\_TONEAREST},\\
\code{FE\_TOWARDZERO} and \code {FE\_UPWARD} (which are \CLang{}
macros).  In order to change the rounding mode of (floating point)
operations, a \CLang{} programmer invokes the \code{fsetround}
function manually establishing a new state of the processing
machinery.

Within \CL{}, a programmer expects a number of facilities to simplify
coding.  To this end, apart from the expected definition of a
\code{rounding-mode} type defined as
\begin{alltt}
(member :indeterminable
        :zero
        :nearest
        :positive-infinity
        :negative-infinity)
\end{alltt}
and of the corresponding \code{get-rounding-mode} and
\code{set-rounding-mode}, this document also defines the macros
\code{with-rounding-mode}, \code{round-to-zero}, \code{round-to-near},
\code{round-upward}, and \code{round-downward}.  Their intende use is
to localize and automate the establishing of a given rounding mode for
a piece of code.

\vspace*{3mm}

\noindent
As an example, consider the following code snippet:
\begin{alltt}
CL prompt> \codeprompt{(round-upward (* 2 21.0))}
\textit{42.0}
\end{alltt}
In this case the intent of the programmer is to ensure that the
rounding mode in effect while executing the multiplication is
\emph{toward positive infinity}.  Upon returning its value, the
\code{round-upward} macro, the rounding mode is reset to the value
before its invokation.

\subsubsection{Default Rounding Mode}

All \CL{} implementations appear to assume that the floating point
rounding mode is set to \emph{round to nearest}.  Hence this
specification states that the default rounding mode is set \emph{round
  to nearest}.

\subsubsection{Interaction with the \CL{} Reader}

The \CL{} reader builds floating point numbers according to the rules
specified in \cite{1994:ANSICL}.  This specification states that the
\CL{} reader subsystem is not affected by changes of the rounding
mode.  That is,
\begin{alltt}
CL prompt> \codeprompt{(round-upward (read))}
42.0
\textit{42.0}
\end{alltt}
will produce a \code{42.0} result that is rounded according to the
standard rules of the \CL{} reader standard floating point parsing
rules, overriding the \code{round-upward} settings.



\subsection{Notifications and Exception Handling}
\label{sect:notifications}

The floating point and complex arithmetic exceptional situation
handling machinery described in
\cite{2012:LIA1,2001:LIA2,2004:LIA3,2008:IEEE-754} (e.g.,
\cite{2001:LIA2} Section~6, \emph{Notification}) balances three
viewpoints that different engineers have.

\begin{itemize}
\item The hardware instruction set designer\footnote{Just a label; it
    is used for the purpose of illustration.}.
\item The programming language designer.
\item The specification implementor.
\end{itemize}

The \CL{} ANSI Specification provides the following conditions that
may be raised by an implementation (in an \emph{implementation
  dependent} way) in conjunction with floating point operations.
\begin{description}
\item \code{floating-point-invalid-operation},
\item \code{floating-point-inexact},
\item \code{floating-point-overflow},
\item \code{floating-point-underflow}.
\end{description}
Of course, the condition \code{division-by-zero} may also be signaled
by floating point operations.


The entries in Section~\ref{sect:fpe-dictionary} specify an interface
similar the \CLang{} Library interface to the \emph{Floating Point
  Environment} provided by \verb|<fenv.h>| \cite{2018:C18}.


\subsubsection{Interaction Between the ``Low Level'' \IEEEFPStd{} and LIA
  ``Signalling'' Machinery and \CL{} Condition Handling}

The \IEEEFPStd{} and LIA specifications (cfr., Section~7 of
\cite{2008:IEEE-754} and Sections~4.1.3, 4.1.4 and~6 of
\cite{2012:LIA1}) appear to imply that the actual handling of
``exceptional'' situations should follow the steps below.  Given an
operation $f(x, \ldots)$.

\begin{enumerate}
\item Check the arguments of $f$ for special cases regarding
  \clieeeterm{NaN}s and \clieeeterm{infinities}.
\item Decide whether a \clieeeterm{IEEE exception} should be
  \emph{signalled}.
  \begin{enumerate}
  \item Handle the exception \emph{by default}.
  \item Raise the flags to indicate what exception was signalled (and
    handled by default).
  \end{enumerate}
\end{enumerate}

The specifications also assume that the \emph{notification} of an
\emph{exceptional situation} should either be recorded ``somewhere''
(in the \CLang{} specification, Section~7.6, in an object of type
\code{fenv\_t}) and that ``catastrophic'' events should ensue from HW
traps and signals.

The specifications indicate that a language \emph{may} provide
alternative modes of ``handling'' such exceptions, acknowledging the
presence of ``exception handling'' machinery in most modern languages
\missingpart{Reference to specs ``Alternate Exception Handling''};
and \CL{} is not an exception, if not for the much richer set of
features that it provides with its ``Condition System''.




\section{\IEEEFPStd{} and LIA \CL{}-related Dictionary}

\subsection{Floats, Infinities and NaNs}

\DDictionaryItem{Function \code{make-float}}

\DSyntax{}

\code{make-float} \varname{bytes} \code{\&optional}
\varname{float-type}
$\Rightarrow$ \varname{float}

\DArgsNValues{}

\varname{bytes} -- An integer or bit-vector representing the binary
pattern of a floating point number.\\
\varname{float-type} -- A recognizable subtype of \code{float},
defaulting to \code{*read-default-float-format*}.\\
\varname{result} -- the resulting floating point number or \code{NAN}.


\DDescription{}

The function constructs a floating point number of appropriate
\varname{float-type}, starting from the bit content of
\varname{bytes}.

If \varname{bytes} corresponds to the byte pattern of a \emph{NaN},
then a \code{NAN} is returned, regardless of \varname{float-type}.

\DExceptional{}

The function signals a \code{type-error} if either \varname{bytes} or
\varname{float-type} are not as described above.

\DSeeAlso{}

\code{NAN}.


\DNotes{}

This function is, in one form or another, already present in \CL{}
implementations.


\DDictionaryItem{Functions
  \code{make-short-float},
  \code{make-single-float},\\
  \code{make-double-float},
  \code{make-long-float}}

\DSyntax{}

\code{make-short-float} \varname{bytes}
$\Rightarrow$ \varname{result}\\
\code{make-single-float} \varname{bytes}
$\Rightarrow$ \varname{result}\\
\code{make-double-float} \varname{bytes}
$\Rightarrow$ \varname{result}\\
\code{make-long-float} \varname{bytes}
$\Rightarrow$ \varname{result}

\DArgsNValues{}

\varname{bytes} -- An integer or bit-vector representing the binary
pattern of a floating point number.\\
\varname{result} -- the resulting floating point number or \code{NAN}.

\DDescription{}

The functions construct a floating point number of appropriate
float type starting from the bit content of
\varname{bytes}.

If \varname{bytes} corresponds to the byte pattern of a \emph{NaN},
then a \code{NAN} is returned.

The actual float type returned by the functions is the widest one
supported by the implementation; i.e., a call to
\code{make-long-float} may return a \code{double-float}.

\DExceptional{}

The functions signals a \code{type-error} if \varname{bytes} is not of
the type described above

\DSeeAlso{}

\code{NAN}, \code{make-float}.

\DNotes{}

These functions are, in one form or another, already present in \CL{}
implementations.

Implementations may supply a larger set of these functions, e.g.,
\code{make-quad-float}.


\DDictionaryItem{Values \code{NAN}}

\subsubsection*{Value:}

An \emph{implementation-dependent value}.


\DDescription{}

The value \code{NAN} holds a representation of a ``not a number''
object.  The object can either be a \emph{quiet} or a
\emph{signalling} \code{NAN} (see~\cite{2008:IEEE-754}).


\DExamples{}

The actual values of \code{NAN} vary from implementation to
implementation.  Here are two examples of how \code{NAN} can be
represented in two implementations:

\begin{alltt}
SBCL> \codeprompt{NAN}
\textit{#<DOUBLE-FLOAT quiet NaN>}
\textcolor{red}{;;; E.g., the result of (sb-kernel:make-double-float -524288 0)}
\end{alltt}

\begin{alltt}
LW> \codeprompt{NAN}
\textit{1D+-0} \textcolor{red}{#| 1D+-0 is double-float not-a-number |#}
\end{alltt}

\DNotes{}

\noindent
It is to be understood that testing for equality of two \code{NAN}s is
not meaningful.  Especially testing for \code{eq} or \code{eql}.

\noindent
It is also understood that, \code{(numberp nan)} should return \code{T}.

\DSeeAlso{}

\code{is-nan}, \code{nanp}, \code{is-quiet-nan}, \code{quiet-nan-p},
\code{is-signalling-nan}, \code{signalling-nan-p}.


\DDictionaryItem{Functions \code{is-nan}, \code{nanp},
  \code{is-quiet-nan}, \code{quiet-nan-p},\\\code{is-signalling-nan},
  \code{signalling-nan-p}}

\DSyntax{}

\code{is-nan} \varname{x} $\Rightarrow$ \textit{boolean}\\
\code{nanp} \varname{x} $\Rightarrow$ \textit{boolean}\\
\code{is-quiet-nan} \varname{x} $\Rightarrow$ \textit{boolean}\\
\code{quiet-nan-p} \varname{x} $\Rightarrow$ \textit{boolean}\\
\code{is-signalling-nan} \varname{x} $\Rightarrow$ \textit{boolean}\\
\code{signalling-nan-p} \varname{x} $\Rightarrow$ \textit{boolean}\\

\DArgsNValues{}

\varname{x} -- any \CL{} object.

\DDescription{}

The function \code{is-nan} (respectively \code{nanp} etc.) returns \code{T}
whenever \varname{x} is a (representation of an IEEE) NaN.  Otherwise
it returns \code{NIL}. The \emph{quiet} and \emph{signalling} versions
operate similarly.

\DExamples{}
%\subsubsection*{Example:}

\begin{alltt}
CL prompt> \codeprompt{(is-nan nan)}
\textit{T}

CL prompt> \codeprompt{(nanp nan)}
\textit{T}

CL prompt> \codeprompt{(is-nan 42)}
\textit{NIL}

CL prompt> \codeprompt{(is-nan "NaN")}
\textit{NIL}
\end{alltt}



\DDictionaryItem{Constant Variables\\
  \code{long-float-positive-infinity},
  \code{long-float-negative-infinity},\\
  \code{double-float-positive-infinity},
  \code{double-float-negative-infinity},\\
  \code{single-float-positive-infinity},
  \code{single-float-negative-infinity},\\
  \code{short-float-positive-infinity},
  \code{short-float-negative-infinity},\\
  \code{+infL0}, 
  \code{-infL0},
  \code{+infD0}, 
  \code{-infD0},
  \code{+infF0}, 
  \code{-infF0},
  \code{+infS0}, 
  \code{-infS0},
  }

\subsubsection*{Value:}

The value of each of these constants is
\emph{implementation-dependent}.


\DDescription{}

The value of each of these constants must conform with the format
(i.e., the floating point type) codified in the name.


\DExamples{}

\begin{alltt}
SBCL> \codeprompt{single-float-positive-infinity}
\textit{#.SB-EXT:SINGLE-FLOAT-POSITIVE-INFINITY}
\end{alltt}

\begin{alltt}
LW> \codeprompt{single-float-positive-infinity}
\textit{+1F++0} \textcolor{red}{#| +1F++0 is single-float plus-infinity |#}
\end{alltt}

\begin{alltt}
LW> \codeprompt{-infD0}
\textit{-1D++0} \textcolor{red}{#| -1D++0 is double-float plus-infinity |#}
\end{alltt}

\begin{alltt}
LW> \codeprompt{-infL0}
\textit{-1D++0} \textcolor{red}{#| -1D++0 is double-float plus-infinity |#}
\textcolor{red}{;;; Note that in this case the widest float representation
;;; available is DOUBLE-FLOAT.}
\end{alltt}


\DNotes{}

The ``short'' names and the ``long ones'' are completely equivalent.
A possible implementation would be the following:
\begin{alltt}
(define-symbol-macro -infD0 double-float-negative-infinity)
\end{alltt}


\DDictionaryItem{Functions \code{is-infinity}, \code{infinityp}}

\DSyntax{}

\code{is-infinity} \varname{x} $\Rightarrow$ \textit{boolean}\\
\code{infinityp} \varname{x} $\Rightarrow$ \textit{boolean}

\DArgsNValues{}

\varname{x} -- any \CL{} object.

\DDescription{}

The function \code{is-infinity} (respectively \code{infinityp}) returns \code{T}
whenever \varname{x} is a (representation of an IEEE) infinity.  Otherwise
it returns \code{NIL}.


\subsubsection*{Example:}

\begin{alltt}
CL prompt> \codeprompt{(is-infinity double-float-negative-infinity)}
\textit{T}

CL prompt> \codeprompt{(infinityp nan)}
\textit{NIL}

CL prompt> \codeprompt{(is-infinity 42)}
\textit{NIL}

CL prompt> \codeprompt{(is-infinity "NaN")}
\textit{NIL}
\end{alltt}






\subsection{Floating Point Exception Handling Dictionary}
\label{sect:fpe-dictionary}

The following are the defined symbols (variables, constants, functions
and macros) pertaining to the handling of the low-level
\clieeeterm{floating point environment}.

\DDictionaryItem{Type \code{fpe-exception}}

\DSupertypes{}

\code{fpe-exception}, \ldots, \code{T}

\DDescription{}

The \code{fpe-exception} type is defined to be:
\begin{alltt}
(member :divide-by-zero
        :inexact
        :invalid
        :overflow
        :underflow)
\end{alltt}
The meaning of these values correspond to a the possible exceptions
supported by an implementation.

\DSeeAlso{}

\code{fpe-exceptions-set}, \code{+fpe-all-exceptions+}.


\DDictionaryItem{Type \code{fpe-exceptions-set}}

\DSupertypes{}

\code{fpe-exceptions-set}, \ldots, \code{T}.

\DDescription{}

Objects of this type have an \emph{implementation dependent}
representation. They represent sets of \code{fpe-exception} objects.

\DNotes{}

The actual implementation may be a \code{list} or an
\code{(unsigned-byte 8)}.  Users should not rely on a particular
underlying implementation.




\DDictionaryItem{Constant Value \code{+fpe-all-exceptions+}}

\subsubsection*{Value:}

A value of type \code{fpe-exception-set}.

\DDescription{}

The value \code{+fpe-all-exceptions+} holds a representation the set of
exceptions -- whose members are of type \code{fpe-exception} -- that
are supported by the implementation.


\DSeeAlso{}

\code{fpe-exception}, \code{fpe-exception-set},
\code{all-supported-exceptions-p},\\
\code{some-supported-exceptions-p},
\code{supported-exception-p}.


\DDictionaryItem{Functions \code{supported-exception-p},\\
  \code{all-supported-exceptions-p},
  \code{some-supported-exceptions-p}}

\DSyntax{}

\code{supported-exception-p}
\varname{exception} $\Rightarrow$ \textit{boolean}\\
\code{all-supported-exceptions-p} \code{\&rest}
\varname{exceptions} $\Rightarrow$ \textit{boolean}\\
\code{some-supported-exceptions-p} \code{\&rest}
\varname{exceptions} $\Rightarrow$ \textit{boolean}

\DArgsNValues{}

\varname{exception} -- a \CL{} object of type 
\code{fpe-exception}.\\
\varname{exceptions} -- a list of \CL{} objects each of type 
\code{fpe-exception}.


\DDescription{}

The functions returns a true value if the exceptions passed as
arguments are supported by the implementation.

The function \code{supported-exception-p} returns true if 
\varname{exception} is supported by the implementation, and \code{NIL}
otherwise.

The function \code{all-supported-exceptions-p} returns true if all the
elements in \varname{exceptions} are supported by the implementation,
and \code{NIL} otherwise.

The function \code{some-supported-exceptions-p} returns true if any the
elements in \varname{exceptions} is supported by the implementation,
and \code{NIL} otherwise.

\DExamples{}

\begin{alltt}
CL prompt> \codeprompt{(supported-exception-p :divide-by-zero)}
\textit{T} \textcolor{red}{; Or could be be NIL.}

CL prompt> \codeprompt{(some-supported-exceptions-p :divide-by-zero :inexact)}
\textit{T} \textcolor{red}{; Assuming the previous operation returned true.}

CL prompt> \codeprompt{(all-supported-exception-p :divide-by-zero :inexact)}
\textit{NIL} \textcolor{red}{; Or could be be T.}
\end{alltt}

\DExceptional{}

The functions signal a \code{type-error} if \varname{exception}
is not of type \code{fpe-exception} or if \varname{exceptions} contains
objects not of type \code{fpe-exception}.


\DDictionaryItem{fpe-test-exceptions}

\DSyntax{}

\code{fpe-test-exceptions} \code{\&rest} \varname{excps}
$\Rightarrow$ \varname{excp-set}

\DArgsNValues{}

\varname{excps} -- A list of \code{fpe-exception} items.\\
\varname{excp-set} -- An object of type \code{fpe-exception-set}.

\DDescription{}

The function tests which of the \varname{excps} is set (i.e., whether
the corresponding flag is set in the underlying floating point
environment) and returns an object of type \code{fpe-exception-set}
with the corresponding flag set.

\DExamples{}

The following example (adapted from \cite{2018:C18}) shows how a piece of
code may decide how to \code{signal} either
\code{floating-point-invalid-operation} or
\code{floating-point-overflow} (cfr., \cite{1996:ANSIHyperSpec}.)

\begin{alltt}
(let ((fpe-excps \textcolor{blue}{(fpe-test-exceptions :overflow :invalid)}))
   (when (fpe-check-exceptions :invalid)
     (signal 'cl:floating-point-invalid-operation))
   (when (fpe-check-exceptions :overflow)
     (signal 'cl:floating-point-overflow))
   )
\end{alltt}

\DNotes{}

The \code{fpe-test-exceptions} function accesses the current floating
point environment.  \CL{} implementations may have made different
choices about if, when, and how to signal the standard \CL{} floating
point conditions.  That is, the above example may or may not work in a
given \CL{} implementation, as the code that actually set the floating
environment exception flags may have already signalled either
\code{cl:floating-point-invalid-operation} or
\code{cl:floating-point-overflow}, and the some corresponding handling
code may have already cleared the flags.

\DSeeAlso{}

\code{fpe-exception}, \code{fpe-exception-set},
\code{fpe-check-exceptions},\\
\code{cl:floating-point-invalid-operation},
\code{cl:floating-point-overflow}.





\DDictionaryItem{fpe-check-exceptions}

\DSyntax{}

\code{fpe-check-exceptions} \varname{excp-set} \code{\&rest} \varname{excps}
$\Rightarrow$ \varname{result}

\DArgsNValues{}

\varname{excp-set} -- An object of type \code{fpe-exception-set}\\
\varname{excps} -- A list of \code{fpe-exception} items.\\
\varname{result} -- A boolean.

\DDescription{}

The function checks whether \emph{all} of the \varname{excps} flags
are set in the \varname{excp-set} and returns a boolean indicating the
result.  If the \varname{excps} is empty, then the function returns
\code{NIL}.

\DExamples{}

The following example (adapted from \cite{2018:C18}) shows how a piece of
code may decide how to \code{signal} either
\code{floating-point-invalid-operation} or
\code{floating-point-overflow} (cfr., \cite{1996:ANSIHyperSpec}.)

\begin{alltt}
(let ((fpe-excps (fpe-test-exceptions :overflow :invalid)))
   (when \textcolor{blue}{(fpe-check-exceptions :invalid)}
     (signal 'cl:floating-point-invalid-operation))
   (when \textcolor{blue}{(fpe-check-exceptions :overflow)}
     (signal 'cl:floating-point-overflow))
   )
\end{alltt}

\DNotes{}

Note that the value returned when \varname{excps} is empty is nor what
\CL{} users may expect from a function that looks like a call to
\code{(and)}.

The notes about \code{fpe-test-exceptions} regarding the example above
apply also in the case of \code{fpe-check-exceptions}.



\DSeeAlso{}

\code{fpe-exception}, \code{fpe-exception-set},
\code{fpe-test-exceptions},\\
\code{cl:floating-point-invalid-operation},
\code{cl:floating-point-overflow}.








\subsection{Floating Point Environment}

The \CLang{} specification \cite{2018:C18}, Section~7.6, defines the \emph{Floating
  Point Environment} for a \CLang{} implementation.  Various \CL{}
implementations provide access to the floating point environment, but
with a wide range of interfaces, which boils down tho the assumed
representation of the \CLang{} \code{fenv\_t} type, described in
\cite{2018:C18}.

This specification describes an interface which should accomodate the
current -- at the tie of this writing -- treatment of the floating
point environment provided by various \CL{} implementations.


\DDictionaryItem{Type \code{floating-point-environment}}

\DSupertypes{}

\code{floating-point-environment}, \ldots, \code{T}

\DDescription{}

The actual \code{floating-point-environment} definition is
\emph{implementation dependent}.  The definition is used to specify
the type of arguments being used by the functions comprising the
interface.


\DDictionaryItem{Accessors \code{fpe-traps}, \code{fpe-rounding-mode},
  \code{fpe-current-exceptions}, \code{fpe-accrued-exceptions},
  \code{fpe-precision}, \code{fpe-fast-mode-p}}

\DSyntax{}

\code{fpe-traps} \varname{fpe} $\Rightarrow$ \varname{traps}\\
\code{fpe-rounding-mode} \varname{fpe} $\Rightarrow$ \varname{rounding-mode}\\
\code{fpe-current-exceptions} \varname{fpe} $\Rightarrow$ \varname{exceptions}\\
\code{fpe-accrued-exceptions} \varname{fpe} $\Rightarrow$ \varname{exceptions}\\
\code{fpe-precision} \varname{fpe} $\Rightarrow$ \varname{precision}\\
\code{fpe-fast-mode-p} \varname{fpe} $\Rightarrow$ \varname{result}

\DArgsNValues{}

\varname{fpe} -- An object of type \code{floating-point-environment}.\\
\varname{rounding-mode} -- A member of the type \code{rounding-mode}.\\
\varname{exceptions} -- An object of type \code{fpe-exception-set}.\\
\varname{precision} -- One of the one of the integers 24, 53 and 64.\\
\varname{result} -- A \emph{boolean}.

\DDescription{}

The ``accessors'' (readers -- functions) extract information from
\varname{fpe}.


\DNotes{}

See also Section~\ref{sect:rounding} (below.)

\DSeeAlso{}

\code{floating-point-environment}, \code{rounding-mode},
\code{fpe-exception-set}. 


\DDictionaryItem{Function \code{set-floating-point-environment}}

\DSyntax{}

\begin{tabbing}
\code{set-floating-point-environment} \= \code{\&key}\\
\>\varname{traps}\\
\>\varname{rounding-mode}\\
\>\varname{current-exceptions}\\
\>\varname{precision}\\
\>\code{\&allow-other-keys}\\
$\Rightarrow$ \varname{modes}
\end{tabbing}


\DArgsNValues{}

\varname{traps} -- A list of the exception conditions that should cause
traps.\\
\varname{rounding-mode} -- The rounding mode to use when the result is
not exact.\\
\varname{current-exceptions} -- The argument is used to set the current
set of exceptions.\\
\varname{precision} -- An integer.\\
\varname{modes} -- An a-list containing the current floating
point modes; the indicators are keywords.

\DDescription{}

This function sets options controlling the floating-point
hardware. If a keyword is not supplied, then the current value is
preserved.

The possible values for each of the keywords are the
following.

\begin{itemize}
\item \varname{traps} is a list that can contain the keywords
  \code{:underflow}, \code{:overflow}, \code{:inexact}, \code{:invalid},
  \code{:divide-by-zero}, and \code{:denormalized-operand}.

\item \varname{rounding-mode} is the rounding mode to use when the result is
  not exact; it can assume the values \code{:nearest},
  \code{:positive-infinity}, \code{:negative-infinity} and
  \code{:zero}.

\item \varname{current-exceptions} is used to set the exception flags. The
  main use is setting the accrued exceptions to \code{NIL} to clear
  them.

\item \varname{precision} can be one of the integers 24, 53 and 64, standing for
  the internal precision of the mantissa.
\end{itemize}

\DExamples{}

None.


\DNotes{}

None.


\DExceptional{}

The function can always result in a no-op if access to the underlying
hardware is not fully supported.  When this happens
\code{set-floating-point-environment} must issue a warning.


\DSeeAlso{}

\code{get-floating-point-environment}.


\DDictionaryItem{Function \code{get-floating-point-environment}}

\DSyntax{}

\code{get-floating-point-environment} \textit{$<$no arguments$>$}
$\Rightarrow$ \varname{modes}

\DArgsNValues{}

\varname{modes} --  An object of type \code{floating-point-environment}.


\DDescription{}

The function returns an a-list that represents the current state of
the floating point modes in use at the time.  The format of the
returned \varname{modes} a-list is such to be
usable as an \code{apply} last argument for
\code{set-floating-point-environment}.


\DExamples{}

\begin{alltt}
SBCL> \codeprompt{(get-floating-point-environment)}
\textit{(:TRAPS (:OVERFLOW :INVALID :DIVIDE-BY-ZERO)
 :ROUNDING-MODE :NEAREST
 :CURRENT-EXCEPTIONS (:INEXACT)
 :FAST-MODE NIL
 :PRECISION :53-BIT)}
\end{alltt}


\DSeeAlso{}

\code{set-floating-point-environment}.


\DDictionaryItem{Function \code{default-floating-point-environment}}

\DSyntax{}

\code{default-floating-point-environment} \textit{$<$no arguments$>$}
$\Rightarrow$ \varname{fpe}

\DArgsNValues{}

\varname{fpe} -- An object of type \code{floating-point-environment}.

\DDescription{}

The function returns an object of type
\code{floating-point-environment} that represents the \emph{default}
floating point environment in use by the implementation.

% The format of the returned \varname{fpe} a-list is such to be usable
% as an \code{apply} last argument for
% \code{set-floating-point-environment}.

\DExamples{}

\begin{alltt}
SBCL> \codeprompt{(default-floating-point-environment)}
\textit{(:TRAPS (:OVERFLOW :INVALID :DIVIDE-BY-ZERO)
 :ROUNDING-MODE :NEAREST
 :CURRENT-EXCEPTIONS (:INEXACT)
 :FAST-MODE NIL
 :PRECISION :53-BIT)}
\textcolor{red}{;;; The format of the result for SBCL is incidental.
;;; The fpe-* readers can be made to work with such representation.}
\end{alltt}

\DNotes{}

The function \code{default-floating-point-environment} should always return
the same (or \code{equalp}) value.

\DSeeAlso{}

\code{set-floating-point-environment}, \code{get-floating-point-environment}.


\newpage
\DDictionaryItem{Macro \code{with-floating-point-environment}}

\DSyntax{}

% \code{with-floating-point-environment} (\textit{\code{\&key}}
% \varname{traps}
% \varname{rounding-mode}
% \varname{current-exceptions}
% \varname{precision}
% \varname{\code{\&allow-other-keys}}) \code{\&body} \varname{body}
% $\Rightarrow$ \varname{results}

\begin{tabbing}
\code{with-floating-point-environment} \=(\=\code{\&key}\\
\>\>                                \varname{traps}\\
\>\>                                \varname{rounding-mode}\\
\>\>                                \varname{current-exceptions}\\
\>\>                                \varname{precision}\\
\>\>                                \code{\&allow-other-keys})\\
\> \code{\&body} \varname{body}\\
$\Rightarrow$ \varname{results}
\end{tabbing}




\DArgsNValues{}

\varname{traps} -- A list of the exception conditions that should cause
traps.\\
\varname{rounding-mode} -- The rounding mode to use when the result is
not exact.\\
\varname{current-exceptions} -- The argument is used to set the current
set of exceptions.\\
\varname{precision} -- An integer.\\
\varname{results} -- One or more \CL{} objects.


\DDescription{}

The \code{with-floating-point-environment} macro executes \varname{body} in a
an environment where the floating point modes are determined by the
values passed as arguments to the macro.  Upon termination (either
normal or exceptional) of the code in \varname{body} the floating
point modes are restored to those in effect before the execution of\\
\code{with-floating-point-environment}.

As for \code{set-floating-point-environment} the values that the arguments
can take are the following:

\begin{itemize}
\item \varname{traps} is a list that can contain the keywords
  \code{:underflow}, \code{:overflow}, \code{:inexact}, \code{:invalid},
  \code{:divide-by-zero}, and \code{:denormalized-operand}.

\item \varname{rounding-mode} is the rounding mode to use when the result is
  not exact; it can assume the values \code{:nearest},
  \code{:positive-infinity}, \code{:negative-infinity} and
  \code{:zero}.

\item \varname{current-exceptions} is used to set the exception flags. The
  main use is setting the accrued exceptions to \code{NIL} to clear
  them.

\item \varname{precision} can be one of the integers 24, 53 and 64,
  standing for the internal precision of the mantissa.
\end{itemize}



\noindent
\varname{results} is the value (or values) returned by \varname{body}.


\DNotes{}

When called with an empty arguments list,
\code{with-floating-point-environment} is a no-op and \varname{body}
is executed as-is.

\DExceptional{}

The macro \code{with-floating-point-environment} performs a minimal
code-walk of \varname{body} and if it finds some floating point
operation which potentially may not respect the changed environment as
described by the specified arguments, it then issues a warning.


\DSeeAlso{}

\code{get-floating-point-environment},
\code{set-floating-point-environment}\\
\code{with-rounding-mode}.



\subsection{Rounding}
\label{sect:rounding}

Having control over IEEE rounding modes is necessary to implement a
number of numerical algorithms and data structures. The following
entries provide access to the IEEE facilities.


\DDictionaryItem{Type \code{rounding-mode}}

\DSupertypes{}

\code{rounding-mode}, \ldots, \code{T}

\DDescription{}

The \code{rounding-mode} type is defined to be:
\begin{alltt}
(member :indeterminable
        :zero
        :nearest
        :positive-infinity
        :negative-infinity)
\end{alltt}
The meaning of these values correspond to a direction of floating
point rounding.


\DNotes{}

The keywords used correspond to the \CLang{} Library \cite{2018:C18}
\textbf{\code{FLT\_ROUNDS}} values of:

\vspace*{3mm}

\begin{tabular}{rl}
  \textbf{\code{-1}} & indeterminable.\\
  \textbf{\code{0}}  & toward zero.\\
  \textbf{\code{1}}  & toward nearest.\\
  \textbf{\code{2}}  & toward positive infinity.\\
  \textbf{\code{3}}  & toward negative infinity.\\
\end{tabular}

\vspace*{3mm}

As per the \CLang{} Library standard, \CL{} implementations can extend the
type \code{rounding-modes} with other keywords representing
\emph{implementation dependent} rounding modes.


\DDictionaryItem{Function \code{get-rounding-mode}}

\DSyntax{}

\code{get-rounding-mode} \textit{$<$no arguments$>$}
$\Rightarrow$ \varname{rounding-mode}

\DArgsNValues{}

\varname{rounding-mode} -- a value of type \code{rounding-modes}.

\DDescription{}

The function returns the current rounding mode.  The value
\code{:indeterminate} is returned if such rounding mode cannot be
determined.

\DSeeAlso{}

 \code{rounding-modes}.


\DDictionaryItem{Function \code{set-rounding-mode}}

\DSyntax{}

\code{set-rounding-mode} \varname{rounding-mode}
$\Rightarrow$ \varname{result}, \varname{success}

\DArgsNValues{}

\varname{rounding-mode} -- a \code{rounding-mode}\\
\varname{result} -- a \code{rounding-mode}\\
\varname{success} -- a boolean


\DDescription{}

The function sets the rounding mode.  If the setting of the rounding
mode is succesful, then \varname{rounding-node} is returned as
\varname{result} and \varname{success} is \code{T}.  Otherwise, the
rounding mode before the the call is returned with \varname{success}
\code{NIL}.\marginnote{Should it instead signal an error?}


\DDictionaryItem{Macro \code{with-rounding-mode}}

\DSyntax{}

\code{with-rounding-mode} \code{(} \varname{rm} \code{)} \code{\&body}
\varname{body}\\
$\Rightarrow$ \varname{results}

\DArgsNValues{}

\varname{rm} -- An item of type \code{rounding-modes}.\\
\varname{body} -- An implicit \code{progn} of code.\\
\varname{results} -- The value (or values) returned by \varname{body}.

\DDescription{}

The code in \varname{body} is executed with a floating point
environment where the rounding mode is set to \varname{rm}.  The
previous rounding mode is saved before executing \varname{body} and it
is restored (as if using \code{unwind-protect}) upon exit.

\DExceptional{}

The macro \code{with-rounding-mode} performs a minimal code-walk of
\varname{body} and if it finds some floating point operation which
potentially may not respect the rounding mode \varname{rm} issues a
warning.
%
It is assumed that this warning will be raised at macro-expansion
time.

\DExamples{}

The following examples may be from two implementations behaving
differently with respect to their handling of rounding modes at the
\CL{} package level.

\begin{alltt}
CL(A) prompt> \codeprompt{(with-rounding-mode (:positive-infinity)
                  (cl:* 2 21.0))}
\textit{42.0}
\end{alltt}

\begin{alltt}
CL(B) prompt> \codeprompt{(with-rounding-mode (:positive-infinity)
                  (cl:* 2 21.0))}
\textcolor{red}{Warning:
the function CL:* may not respect the new rounding mode :POSITIVE-INFINITY.}
\textit{42.0}
\end{alltt}

\DDictionaryItem{Macros \code{round-to-zero}, \code{round-to-near},
  \code{round-upward},\\
  \code{round-downward}}

\DSyntax{}

\code{round-to-zero} \code{\&body} \varname{body}
$\Rightarrow$ \varname{results}\\
\code{round-to-near} \code{\&body} \varname{body}
$\Rightarrow$ \varname{results}\\
\code{round-upward} \code{\&body} \varname{body}
$\Rightarrow$ \varname{results}\\
\code{round-downward} \code{\&body} \varname{body}
$\Rightarrow$ \varname{results}



\section{Floating Point \IEEEFPStd{} Respecting Operations}
\label{sect:fp-operations}

A \CL{} implementation or a \CL{} library implementing this
specification will provide the following \emph{non-computational
  operations} (to follow the language of Section~5 of \cite{2008:IEEE-754})
alongside some \CL{} features to allow for read-time (cfr.,
\emph{translation-time}) evaluations.


\subsection{Non-computational \CL{} Enviroment \IEEEFPStd{} Queries}

\DDictionaryItem{Function \code{is-cdr-ieee-754-conformant}}

\DSyntax{}

\code{is-cdr-ieee-754-conformant} \varname{$<$no argument$>$}
$\Rightarrow$ \varname{result}

\DArgsNValues{}

\varname{result} -- a generalized boolean.

\DDescription{}

This function returns a non-\code{NIL} value if the \CL{}
implementation or the \CL{} library implements the specification of
this document.

If this function returns non-\code{NIL}, so will
\code{is-cdr-ieee-745-constants-providing},
\code{is-cdr-ieee-754-environment-providing} and
\code{is-cdr-ieee-745-operation-providing}.

\DSeeAlso{}

\code{is-cdr-ieee-745-constants-providing},
\code{is-cdr-ieee-754-environment-providing} and
\code{is-cdr-ieee-745-operation-providing}.



\DNotes{}

\cite{2008:IEEE-754} specifies two predicates \code{is754version1985} and
\code{is754version2008}, but they imply conformance to the full
\IEEEFPStd{} standard.



\DDictionaryItem{Function \code{is-cdr-ieee-745-constants-providing}}

\DSyntax{}

\code{is-cdr-ieee-745-constants-providing} \varname{$<$no argument$>$}
$\Rightarrow$ \varname{result}

\DArgsNValues{}

\varname{result} -- a generalized boolean.

\DDescription{}

This function returns a non-\code{NIL} value if the \CL{}
implementation or the \CL{} library only provides the names of the
constants naming NaNs and infinities.


\DDictionaryItem{Function \code{is-cdr-ieee-745-environment-providing}}

\DSyntax{}

\code{is-cdr-ieee-745-environment-providing} \varname{$<$no argument$>$}
$\Rightarrow$ \varname{result}

\DArgsNValues{}

\varname{result} -- a generalized boolean.

\DDescription{}

This function returns a non-\code{NIL} value if the \CL{}
implementation or the \CL{} library provides the operations for
accessing and manipulating floating point exceptions
\checkref{Sections references} and the floating
point environment.


\DDictionaryItem{Function \code{is-cdr-ieee-745-operation-providing}}

\DSyntax{}

\code{is-cdr-ieee-745-operation-providing} \varname{$<$no argument$>$}
$\Rightarrow$ \varname{result}

\DArgsNValues{}

\varname{result} -- a generalized boolean.


\DDescription{}

This function returns a non-\code{NIL} value if the \CL{}
implementation or the \CL{} library provides separate implementations
of the \emph{computational operations} that do conform to the \IEEEFPStd{}
mandated behavior.


\DDictionaryItem{Function \code{is-cl-using-cdr-ieee-745}}

\DSyntax{}

\code{is-cl-using-cdr-ieee-745} \varname{$<$no argument$>$}
\RArrow \varname{result}

\DArgsNValues{}

\varname{result} -- a generalized boolean.


\DDescription{}

This function returns a non-\code{NIL} value if the mathematical
operations in the ANSI \CL{} standard \cite{1996:ANSIHyperSpec} respect the
IEEE~745 specification.

\DNotes{}

This is a stronger requirement than the
presence of the \code{ieee-floating-point} feature in
\code{*features*}.


\DDictionaryItem{Features \code{:cdr-ieee-745-constants-providing},\\
  \code{:cdr-ieee-745-environment-providing} and\\
  \code{:cdr-ieee-745-operation-providing}}

\DDescription{}

These features are present in the \code{*features*} list whenever the
corresponding function, \code{is-}\emph{feature} returns a
non-\code{NIL} value.

\subsection{Numeric Operations}

The following \CL{} operations, broken down according to the
classification in Section~12.1.1 of \cite{1996:ANSIHyperSpec} are exported
form the \code{CL-MATH-IEEE-2019} package
(cfr.,~\ref{sect:package}).

If \code{is-cdr-ieee-754-operation-providing} returns non-\code{NIL}
(and the\\
\code{:cdr-ieee-754-operation-providing} is in
\code{*features*}, then the operations are also implemented, otherwise
each of them signal an ``not implemented'' error;\\
see \code{ieee-754-not-implemented-item}.

The list of operations
recommended by \cite{2008:IEEE-754} is more extensive than the list of
operations provided by \CL{} (cfr., Table~9.1 in \cite{2008:IEEE-754}).
%
The tables referenced below provide correspondances for the \CL{} functions,
especially regarding the exceptions (conditions) that must be
signalled.

\paragraph{Rounding Modes.} In particular, all the operations listed
in Sections~\ref{sect:arith-ops} and~\ref{sect:transc-ops} respect the
\clieeeterm{rounding mode} set in the floating point
environment in effect when the operation is executed.

\paragraph{Underflow and Overflow.}  The operations listed signal
\code{cl:floating-point-overflow} and
\code{cl:floating-point-underflow} according to the rules established
in \cite{2008:IEEE-754} and Section~12.1.4.3 of \cite{1996:ANSIHyperSpec}.



\subsubsection{Arithmetic Operations}
\label{sect:arith-ops}

\begin{table}[h!]
  \centering
  \begin{tt}
    \begin{tabular}{lll}
      * & 1+ & \ldots \\
      + & 1- & \ldots \\
      - & incf & conjugate\\
      / & decf & \\
    \end{tabular}
  \end{tt}
  \caption{The basic \CL{} arithmetic operations}
  \label{table:cl-arit-ops}
\end{table}

\noindent
Each of the functions in Table~\ref{table:cl-arit-ops} must be further
specified with respect to \cite{1996:ANSIHyperSpec} in order to adhere to
the requirements of \IEEEFPStd{}.  The descriptions and the references for
each function further specify the standard \CL{} behavior with respect
to \emph{NaN}s, \emph{infinities} and floating point exceptions.

\vspace*{3mm}

\noindent
Note that the \code{gcd}, and \code{lcm} functions are not
present in the above table, which corresponds to Figure~12-1 of
\cite{1996:ANSIHyperSpec}.


\DDictionaryItem{Functions \code{+}, \code{*}}

\DSyntax{}

\code{+} \varname{a} \varname{b} \RArrow \varname{result}\\
\code{*} \varname{a} \varname{b} \RArrow \varname{result}\\
\code{+} \code{\&rest} \varname{ns} \RArrow \varname \code{n}\\
\code{*} \code{\&rest} \varname{ns} \RArrow \varname \code{n}\\

\DArgsNValues{}

\varname{a}, \varname{b}, \varname{result} -- Floating Point numbers.\\
\varname{numbers} -- A, possibly empty, list of numbers.\\
\varname{n} -- A number.



\DDescription{}

The dyadic versions of the functions \code{+}, and \code{*}, when operating on
floating point numbers (or numbers that eventually are converted to
floating point numbers according to \CL{} float contagion rules -- cfr.,
Section~12.1.4 of \cite{1996:ANSIHyperSpec}) assume the behavior of the
underlying \IEEEFPStd{} specification \cite{2008:IEEE-754}.  It is assumed that
the multiple argument versions are eventually built upon the dyadic
ones (zero and one argument versions can be seen as special cases from
the point of view of this specification).

The dyadic versions of the functions \code{+}, and \code{*} behave
according to the \IEEEFPStd{} operations described in Section~5.4.1 of
\cite{2008:IEEE-754}, which is assumed to be the usual behavior specified
for \CL{} by \cite{1996:ANSIHyperSpec} when \varname{a} and \varname{b} are
not \emph{NaN}s or \emph{infinities}.

\noindent
The operations from \cite{2008:IEEE-754} are:

\vspace*{3mm}

\noindent
\textit{formatOf}-\textbf{addition}(\varname{a}, \varname{b})\\
\textit{formatOf}-\textbf{multiplication}(\varname{a}, \varname{b})

\vspace*{3mm}

\noindent
where \textit{formatOf} describes the resulting floating point
format.  As already mentioned, the actual floating point format of
\varname{result} is dictated by the \CL{} standard.

When \code{+}, or \code{*} is called with either \varname{a} or
\varname{b} being a \emph{quiet NaN}, and neither is a
\emph{signalling NaN} then \varname{result} is a (quiet) \code{NAN}.
No error (floating point exception) is signalled in this case.

When \code{+} is called with \varname{a} being an 
\clieeeterm{infinity} and \varname{b} being a finite \clterm{number}
(or viceversa), then \varname{result} is \varname{a} (or, viceversa, 
\varname{b}).

When \code{*} is called with \varname{a} being an
\clieeeterm{infinity} and \varname{b} being a non-zero finite
\clterm{number} (or viceversa), then \varname{result} is \varname{a}
(or, viceversa, \varname{b}).

\DExceptional{}

There are different exceptional situations to be considered.

\begin{enumerate}
\item When \code{+} is called with either \varname{a} or \varname{b}
  being a \emph{signalling NaN}, then the\\
  \clname{cl:floating-point-invalid-operation} error is signalled.

\item When \code{+} is called with \varname{a} being a
  \clieeeterm{positive infinity} and \varname{b} being a
  \clieeeterm{negative infinity} (or viceversa), then the
  \clname{cl:floating-point-invalid-operation} error is signalled.

\item When \code{*} is called with \varname{a} being an
  \clieeeterm{infinity} and \varname{b} being a \clieeeterm{zero}
  value (or viceversa), then the
  \clname{cl:floating-point-invalid-operation} error is signalled.

\item If some of \varname{a}, \varname{b}, or any element of a non-empty
  \varname{numbers} is not a \CL{} \clterm{number} then the function
  might signal a \clname{cl:type-error}.
\end{enumerate}



\DDictionaryItem{Function \code{-}}

\DSyntax{}

\code{-} \varname{a} \varname{b} \RArrow \varname{result}\\
\code{-} \varname{a} \code{\&rest} \varname{ns} \RArrow \varname \code{n}\\

\DArgsNValues{}

\varname{a}, \varname{b}, \varname{result} -- Floating Point numbers.\\
\varname{numbers} -- A, possibly empty, list of numbers.\\
\varname{n} -- A number.



\DDescription{}

The dyadic version of the function \code{-}, when operating on
floating point numbers (or numbers that eventually are converted to
floating point numbers according to \CL{} float contagion rules -- cfr.,
Section~12.1.4 of \cite{1996:ANSIHyperSpec}) assume the behavior of the
underlying \IEEEFPStd{} specification \cite{2008:IEEE-754}.  It is assumed that
the multiple argument versions are eventually built upon the dyadic
ones (the one argument version can be seen as a special case from
the point of view of this specification).

The dyadic version of the functions \code{-} behaves
according to the \IEEEFPStd{} operations described in Section~5.4.1 of
\cite{2008:IEEE-754}, which is assumed to be the usual behavior specified
for \CL{} by \cite{1996:ANSIHyperSpec} when \varname{a} and \varname{b} are
not \emph{NaN}s or \emph{infinities}.

\noindent
The operation from \cite{2008:IEEE-754} is:

\vspace*{3mm}

\noindent
\textit{formatOf}-\textbf{subtraction}(\varname{a}, \varname{b})

\vspace*{3mm}

\noindent
where \textit{formatOf} describes the resulting floating point
format.  As already mentioned, the actual floating point format of
\varname{result} is dictated by the \CL{} standard.

When \code{-} is called with either \varname{a} or \varname{b} being a
\emph{quiet NaN}, and neither is a \emph{signalling NaN} then
\varname{result} is a (quiet) \code{NAN}.  No error (floating point
exception) is signalled in this case.

When \code{-} is called with \varname{a} being an 
\clieeeterm{infinity} and \varname{b} being a finite \clterm{number}
(or viceversa), then \varname{result} is \varname{a}; when \code{-} is
called with \varname{b} being an \clieeeterm{infinity} and \varname{a}
being a finite \clterm{number}, then \varname{result}
is $-$\varname{b}.


\DExceptional{}

There are different exceptional situations to be considered.

\begin{enumerate}
\item When \code{-} is called with either \varname{a} or \varname{b}
  being a \emph{signalling NaN}, then the\\
  \clname{cl:floating-point-invalid-operation} error is signalled.

\item When \code{-} is called with \varname{a} being a
  \clieeeterm{positive infinity} and \varname{b} being a
  \clieeeterm{positive infinity}, or \varname{a} being a
  \clieeeterm{negative infinity} and \varname{b} being a
  \clieeeterm{positive infinity}, then the\\
  \clname{cl:floating-point-invalid-operation} error is signalled.

\item If some of \varname{a}, \varname{b}, or any element of a non-empty
  \varname{numbers} is not a \CL{} \clterm{number} then the function
  might signal a \clname{cl:type-error}.
\end{enumerate}


\DDictionaryItem{Function \code{/}}

\DSyntax{}

\code{/} \varname{a} \varname{b} \RArrow \varname{result}\\
\code{/} \varname{n} \code{\&rest} \varname{ns} \RArrow \varname \code{r}\\

\DArgsNValues{}

\varname{a}, \varname{b}, \varname{result} -- Floating Point numbers.\\
\varname{numbers} -- A, possibly empty, list of numbers.\\
\varname{n}, \varname{r} -- A number.



\DDescription{}

The dyadic version of the function \code{/}, when operating on
floating point numbers (or numbers that eventually are converted to
floating point numbers according to \CL{} float contagion rules -- cfr.,
Section~12.1.4 of \cite{1996:ANSIHyperSpec}) assume the behavior of the
underlying \IEEEFPStd{} specification \cite{2008:IEEE-754}.  It is assumed that
the multiple argument versions are eventually built upon the dyadic
ones (the one argument version can be seen as a special case from
the point of view of this specification).

The dyadic version of the functions \code{/} behaves
according to the \IEEEFPStd{} operations described in Section~5.4.1 of
\cite{2008:IEEE-754}, which is assumed to be the usual behavior specified
for \CL{} by \cite{1996:ANSIHyperSpec} when \varname{a} and \varname{b} are
not \emph{NaN}s or \emph{infinities}.


\noindent
The operation from \cite{2008:IEEE-754} is:

\vspace*{3mm}

\noindent
\textit{formatOf}-\textbf{division}(\varname{a}, \varname{b})

\vspace*{3mm}

\noindent
where \textit{formatOf} describes the resulting floating point
format.  As already mentioned, the actual floating point format of
\varname{result} is dictated by the \CL{} standard.

When \code{/} is called with either \varname{a} or \varname{b} being a
\emph{quiet NaN}, and neither is a \emph{signalling NaN} then
\varname{result} is a (quiet) \code{NAN}.  No error (floating point
exception) is signalled in this case.

When \code{/} is called with \varname{a} being an 
\clieeeterm{infinity} and \varname{b} being a finite \clterm{number}
(or viceversa), then \varname{result} is \varname{a} with its sign
possibly changed; when \code{/} is
called with \varname{b} being an \clieeeterm{infinity} and \varname{a}
being a finite \clterm{number}, then \varname{result}
is \clieeeterm{zero}.


\DExceptional{}

There are different exceptional situations to be considered.

\begin{enumerate}
\item When \code{/} is called with either \varname{a} or \varname{b}
  being a \emph{signalling NaN}, then the\\
  \clname{cl:floating-point-invalid-operation} error is signalled.

\item When \code{/} is called with \varname{a} being an 
  \clieeeterm{infinity} and \varname{b} being an 
  \clieeeterm{infinity} as well, then the\\
  \clname{cl:floating-point-invalid-operation} error is signalled.

\item When \code{/} is called with \varname{a} being a
  \clieeeterm{zero} and \varname{b} being an 
  \clieeeterm{zero} as well, then the\\
  \clname{cl:floating-point-invalid-operation} error is signalled.

\item When the dyadic version of \code{/} is called with \varname{a}
  being a finite \clterm{number} and \varname{b} being a
  \clieeeterm{zero}, or when the monadic version of \code{/} is called
  with \varname{n} being a \clieeeterm{zero}, then the
  \clname{cl:division-by-zero} error is signalled.

\item If some of \varname{a}, \varname{b}, or a
  ny element of a non-empty
  \varname{numbers} is not a \CL{} \clterm{number} then the function
  will signal a \clname{cl:type-error}.
\end{enumerate}

\noindent
The monadic version of the function \code {/} behaves as \code{(/ 1}
\varname{n}\code{)} with respect to exceptions being signalled.


\DDictionaryItem{Functions \code{1+}, \code{1-}}

\DSyntax{}

\code{1+} \varname{n} \RArrow ~ \varname{result}\\
\code{1-} \varname{n} \RArrow ~ \varname{result}\\

\DArgsNValues{}

\varname{n} -- A \clterm{number}.\\
\varname{result} -- A \clterm{number}.


\DDescription{}

The functions \code{1+}, \code{1-}, behave
as the \CL{} counterparts on non-\clieeeterm{NaN}s and
non-\clieeeterm{infinities} (see \cite{1996:ANSIHyperSpec}).

The functions \code{1+}, \code{1-} return a \clieeeterm{quiet NaN} as \varname{result}
when \varname{n} is a \clieeeterm{quiet NaN}; they return an
\clieeeterm{infinity} equal to \varname{n}, when \varname{n} is an
\clieeeterm{infinity}.

\DExceptional{}

There are different exceptional situations to be considered.

\begin{enumerate}
\item When \code{1+} or \code{1-} is called with \varname{n}
  being a \emph{signalling NaN}, then the\\
  \clname{cl:floating-point-invalid-operation} error is signalled.

\item If \varname{n} is not a \CL{} \clterm{number} then the function
  signals a \clname{cl:type-error}.
\end{enumerate}

\DSeeAlso{}

\code{+}, \code{-}.


\DDictionaryItem{Macros \code{incf}, \code{decf}}

\DSyntax{}

\code{incf} \varname{place} \code{\&optional} \varname{d} \RArrow ~ \varname{result}\\
\code{decf} \varname{place} \code{\&optional} \varname{d} \RArrow ~ \varname{result}

\DArgsNValues{}

\varname{d} -- A \clterm{number}, the default is \code{1}.\\
\varname{result} -- A \clterm{number}.\\
\varname{place} -- A \CL{} \clterm{place}.\\


\DDescription{}

The macros \code{incf} and \code{decf} behave
as the \CL{} counterparts on non-\clieeeterm{NaN}s and
non-\clieeeterm{infinities} (see \cite{1996:ANSIHyperSpec}).

The macros \code{incf} and \code{decf} return a \clieeeterm{quiet NaN}
as \varname{result} when \varname{place} or \varname{d} is a
\clieeeterm{quiet NaN}; they return an \clieeeterm{infinity} equal to
\varname{n}, when \varname{n} is an \clieeeterm{infinity}.

\DExceptional{}

The conditions signalled are those described for \code{+} and \code{-}.

\DNotes{}

The default for \varname{d} is coerced to the appropriate floating
point \code{1.0}, depending on the floating point format of
\varname{place}.

\DSeeAlso{}

\code{+}, \code{-}, \code{1+}, \code{1-}.


\DDictionaryItem{Function \code{conjugate}}

\DSyntax{}

\code{conjugate} \varname{n} \RArrow{} \varname{c}

\DArgsNValues{}

\varname{n}, \varname{c} -- \CL{} \clterm{number}s.

\DDescription{}

The function \code{conjugate} behaves like the corresponding
\code{cl:conjugate} function, when \varname{n} is a \clterm{real}
number that is not a \clieeeterm{NaN}.

\noindent
When \varname{n} is a \clieeeterm{quiet NaN} then \varname{c} ia a
\clieeeterm{quiet NaN}.

\DExceptional{}

When \varname{n} is a \clieeeterm{signalling NaN} then the
\clname{cl:floating-point-invalid-operation} error is signalled.

When \varname{n} is a \clterm{complex number}, then \code{conjugate}
can signal the same set of conditions as if computing
\code{(- (cl:imagpart} \varname{n}\code{))} while constricting the result
\varname{c}.

The function \code{conjugate} signals a \code{type-error} if
\varname{n} is not a \clterm{number}.


\DSeeAlso{}

\code{-}.






\subsubsection{Exponential, Logarithms and Trigonometry Operations}
\label{sect:transc-ops}

\begin{tt}
  \begin{tabular}{lll}
    \#| abs |\# & cos & signum\\
    acos &  cosh &  sin\\
    acosh & exp  &  sinh\\
    asin &  expt &  sqrt\\
    asinh & isqrt &  tan\\
    atan &  log &   tanh\\
    atanh & phase & \\
    cis & \#| pi |\# & \\
  \end{tabular}
\end{tt}

\vspace*{3mm}

\noindent
The above table corresponds to Figure~12-2 of \cite{1996:ANSIHyperSpec}.
The ``commented'' entries will not be described as they either don't
operate on floating point numbers or do not have a semantic different
from the \CL{} standard.

\noindent
The listed \CL{} functions have correspondances in the \cite{2008:IEEE-754}
specification, but with some key differences, e.g., \code{cl:log} returns
a \clterm{complex number} for negative values.  In the following each
function will be further specified to comply with the \cite{2008:IEEE-754}
standard.

\vspace*{3mm}

The following functions will also be specified for completeness,
according to Section~9 of \cite{2008:IEEE-754}.


\DDictionaryItem{Functions \code{asin}, \code{acos}, \code{atan}}

\DSyntax{}

\code{asin} \varname{n} \RArrow \varname{radians}\\
\code{acos} \varname{n} \RArrow \varname{radians}\\
\code{atan} \varname{n1} \code{\&optional} \varname{n2} \RArrow
\varname{radians}

\DArgsNValues{}

\varname{n}, \varname{n1}, \varname{n2} -- A \clieeeterm{floating
  point number}\\
\varname{radians} -- A \clieeeterm{floating point number}

\DDescription{}

The functions \code{asin}, \code{acos}, \code{atan} compute the the
arc sine, arc cosine and arc tangent of a number.  Their behavior is
the one described in \cite{1996:ANSIHyperSpec} for regular floating point
numbers.

The functions return a \clieeeterm{quiet NaN} if \varname{n},
\varname{n1}, or \varname{n2} is a \clieeeterm{quiet NaN}.

The behavior of \code{asin}, \code{acos}, and \code{atan} in case of
\clterm{complex} arguments is also the one described in
\cite{1996:ANSIHyperSpec}.

\DNotes{}

The functions \code{asin} and \code{acos} do not signal a
\code{cl:floating-point-invalid-operation} when \varname{n} is a
\clterm{real} outside the $[-1, 1]$ interval; they quietly return a
\clterm{complex} number.



\DExceptional{}

If \varname{n}, \varname{n1}, or \varname{n2} is a
\clieeeterm{signalling NaN}, then \code{asin}, \code{acos} and
\code{atan} signal the\\
\clname{cl:floating-point-invalid-operation} error.

According to \cite{2008:IEEE-754} there are several issues to be
considered.

If \varname{n}, \varname{n1}, or \varname{n2} are not \clterm{number}s
a \code{type-error} is signalled by \code{acos}, \code{asin}, and
\code{atan}.  If \varname{n1} and \varname{n2} are both supplied to
\code{atan}, and they are not both \clterm{real} numbers, then a
\code{type-error} is signalled.




\subsubsection*{Numeric Comparison and Predicates}

\begin{tt}
  \begin{tabular}{lll}
    /= &  >= & \\
    < &   & plusp\\
    <= &  max & zerop\\
    = & min & \\
    > & minusp & \\
  \end{tabular}
\end{tt}

\vspace*{3mm}

\noindent
Note that the \code{evenp} and \code{oddp} functions are not present
in the above table,  which corresponds to Figure~12-3 of
\cite{1996:ANSIHyperSpec}.

\paragraph{Correspondances with \IEEEFPStd{}.} The list of operations
recommended by \cite{2008:IEEE-754} is more extensive than the list of
operations provided by \CL{} (cfr., Table~9.1 in \cite{2008:IEEE-754}).
The tables referenced below provide correspondances for the \CL{} functions,
especially regarding the exceptions (conditions) that must be
signalled.

\begin{table}[h]
  \begin{tabulary}{\textwidth}{|L|L|L|}
    \hline
    \CL{} Function & IEEE-745 Function & Exceptional Situations\\
    \hline\hline
    \multicolumn{3}{|l|}{Arithmetic}\\\hline
    \code{+}
    & \textit{F\_\textbf{addition}}(\varname{a}, \varname{b})
    & Long list of cases
    \\\hline
    
  \end{tabulary}
\end{table}

\DDictionaryItem{Error \code{ieee-754-not-implemented-item}}

\DSupertypes{}

\code{ieee-754-not-implemented-item}, \code{error},
\code{cl:arithmetic-error}, \ldots, \code{T}


\nocite{2012:LIA1,2001:LIA2,2004:LIA3}
\nocite{1994:ANSICL}

\bibliographystyle{plain}
\bibliography{CDR-CL-LIA}

% \begin{thebibliography}{9}

% \bibitem{IEEE-754}
%   \textit{{IEEE} Standard for Floating-Point Arithmetic}, IEEE Std
%   754$^{\mathrm{tm}}$-2008, IEEE Computer Society, 2008.
  
% \bibitem{C18}
%   \textit{Programming languages -- C},
%   International Standard \emph{ISO/IEC 9899-2018}, 2018.
  
% \bibitem{ANSIHyperSpec}
%   \textit{The \CL{} Hyperspec},
%   published online at\\
%   \texttt{http://www.lisp.org/HyperSpec/FrontMatter/index.html}, 1994.

% \end{thebibliography}


\appendix

\section{Copying and License}

This work may be distributed and/or modified under the conditions of
the \emph{LaTeX Project Public License} (LPPL), either version 1.3 of this license
or (at your option) any later version. The latest version of this
license is in \texttt{http://www.latex-project.org/lppl.txt} and version 1.3 or
later is part of all distributions of LaTeX version 2005/12/01 or
later.

\noindent
This work has the LPPL maintenance status `maintained'.

\noindent
The Current Maintainer of this work is Marco Antoniotti.

\end{document}

%%%% end of file --CDR-IEEE-754-support.tex --
