%%%% -*- Mode: LaTeX -*-

%%%% CDR-IEEE-754-support.tex --
%%%% Minimalistic support for (a subset) of IEEE 754 for Common Lisp.


\documentclass[fleqn]{article}

\usepackage{latexsym}
\usepackage{epsfig}
\usepackage{alltt}
\usepackage{color}
\usepackage[margin=4cm]{geometry}

% \usepackage[OT1]{fontenc}
% \usepackage{bold-extra}
% \renewcommand{\ttdefault}{cmt}

\usepackage{lmodern}            % This works for decent bold 'tt'.


\newcommand{\tm}{$^\mathsf{tm}$}
\newcommand{\cfr}{\emph{cfr.}}

\newcommand{\Lisp}{\textsf{Lisp}}
\newcommand{\CL}{\textsf{Common~Lisp}}

\newcommand{\Java}{\textsc{Java}}

\newcommand{\checkcite}[1]{{\textbf{[Missing Citation: #1]}}}
\newcommand{\checkref}[1]{{\textbf{[Missing Reference: #1]}}}

\newcommand{\missingpart}[1]{{\ }\vspace{2mm}\\
{\textbf{[Still Missing: #1]}}\\
\vspace{2mm}}

\newcommand{\marginnote}[1]{%
\marginpar{\begin{tiny}\begin{em}
{\raggedright #1}
\end{em}\end{tiny}}}

%\addtolength{\textwidth}{2cm}


%%% CL 

\newcommand{\code}[1]{\texttt{#1}}

\newcommand{\term}[1]{\texttt{#1}}
\newcommand{\nonterm}[1]{\textit{$<$#1$>$}}


\newcommand{\kwd}[1]{\texttt{:#1}}

\newcommand{\varname}[1]{\textit{#1}}

\newcommand{\codeprompt}[1]{\textcolor{blue}{\textbf{#1}}}


%%% ANSI-Spec Like Macros.

% \newcommand{\DDictionaryItem}[1]{\subsection{#1}\vspace*{-9pt}\hrulefill}
\newcommand{\DDictionaryItem}[1]{\vspace*{6pt}\noindent\hrulefill\vspace*{-9pt}\subsection*{#1}}

\newcommand{\DSyntax}{\subsubsection*{Syntax:}}
\newcommand{\DArgsNValues}{\subsubsection*{Arguments and Values:}}
\newcommand{\DDescription}{\subsubsection*{Description:}}
\newcommand{\DExamples}{\subsubsection*{Examples:}}
\newcommand{\DExceptional}{\subsubsection*{Exceptional Situations:}}
\newcommand{\DNotes}{\subsubsection*{Notes:}}
\newcommand{\DSeeAlso}{\subsubsection*{See Also:}}



%%%% Document Title.


\title{
\LARGE{\bfseries Minimalistic Support for (a subset of) IEEE~745 in \CL{}}}

% \author{
% Marco Antoniotti\\
% Dipartimento di Informatica, Sistemistica e Comunicazione\\
% Universit\`{a} degli Studi di Milano Bicocca\\
% Viale Sarca 336, U14, Milan (MI), \textsc{Italy}\\
% \vspace{2mm}
% \texttt{mantoniotti} at \texttt{common-lisp.net},
% \texttt{marco.antoniotti} at \texttt{unimib.it}}

\author{TBD}

%\date{}


%\includeonly{}

\begin{document}

\maketitle

\begin{abstract}
This document presents a set of names and functions that aim at
providing a common ground for the support of (a subset of) IEEE~754 in
\CL{}.

The set of names and functions (and ancillary information) is intented
to be minimalistic; essentially for IEEE~754 constants (e.g.,
\code{NaN}) and some functionalities that are needed to control
floating point computations.
\end{abstract}


\section{Introduction}

The ANSI \CL{} Specification \cite{ANSIHyperSpec} hints at possible
compliance with IEEE ``Floating Points'' (in retrospect, IEEE~754) in
the description of the \code{*features*} variable, where:
\begin{description}
\item[\code{:ieee-floating-point}]
  If present, indicates that the
  implementation \emph{purports to conform}  to the requirements of
  \emph{IEEE Standard for Binary Floating-Point Arithmetic}. 
\end{description}

\noindent
Alas, the actual state of ``compliance'' among different
implementations varies wildly, with some implementations publicizing
the \code{:ieee-floating-point} feature while providing an unclear
subset of features\footnote{We talk of the sin and not of the sinner;
  you can check for yourself what happens in the various
  implementations.}.  This is in contrast with the current state of
affairs in other languages; especially languages designed
\emph{after} the publication of IEEE~754.

\vspace*{3mm}

The goal of this document is to address some of this state of affairs
by providing a specification for a minimalistic subset of names,
functions and macros to be used in conjunction with IEEE~754
concepts.  Two issues especially will 
be addressed:
\begin{itemize}
\item Names for \code{NaN}s and ``infinities''.
\item Functions and macros to handle \emph{rounding modes}; with
  special care devoted to incentivate non-invasive, yet informative,
  implementations.
\end{itemize}

\vspace*{3mm}

Most of the contents of this document borrow ideas from
\cite{IEEE-754}, \cite{C18}, and the documentation of
several \CL{} implementations.



\section{Description}

The IEEE~754 standard introduces a number of ``names'' for certain
special values: \code{NaN}s and infinities, as an example.  Several
\CL{} implementations provide such concepts, alas, not in a consensual
way.  The goal of this document is to provide such a minimal
consensus.

\subsection{\code{CL-MATH-IEEE-2019} Package}

The implementations of this specification will provide a package named
(or nicknamed) \code{CL-MATH-IEEE-2019}.  All the symbols named in the rest
of the document are \code{export}ed from the above mentioned package.




\section{IEEE~754 \CL{}-related Dictionary}

\subsection{Floats, Infinities and NaNs}

\DDictionaryItem{Function \code{make-float}}

\DSyntax{}

\code{make-float} \varname{bytes} \code{\&optional}
\varname{float-type}
$\Rightarrow$ \varname{float}

\DArgsNValues{}

\varname{bytes} -- An integer or bit-vector representing the binary
pattern of a floating point number.\\
\varname{float-type} -- A recognizable subtype of \code{float},
defaulting to \code{*read-default-float-format*}.\\
\varname{result} -- the resulting floating point number or \code{NAN}.


\DDescription{}

The function constructs a floating point number of appropriate
\varname{float-type}, starting from the bit content of
\varname{bytes}.

If \varname{bytes} corresponds to the byte pattern of a \emph{NaN},
then a \code{NAN} is returned, regardless of \varname{float-type}.

\DExceptional{}

The function signals a \code{type-error} if either \varname{bytes} or
\varname{float-type} are not as described above.

\DSeeAlso{}

\code{NAN}.


\DNotes{}

This function is, in one form or another, already present in \CL{}
implementations.


\DDictionaryItem{Functions
  \code{make-short-float},
  \code{make-single-float},\\
  \code{make-double-float},
  \code{make-long-float}}

\DSyntax{}

\code{make-short-float} \varname{bytes}
$\Rightarrow$ \varname{result}\\
\code{make-single-float} \varname{bytes}
$\Rightarrow$ \varname{result}\\
\code{make-double-float} \varname{bytes}
$\Rightarrow$ \varname{result}\\
\code{make-long-float} \varname{bytes}
$\Rightarrow$ \varname{result}

\subsubsection*{Arguments and Values:}

\varname{bytes} -- An integer or bit-vector representing the binary
pattern of a floating point number.\\
\varname{result} -- the resulting floating point number or \code{NAN}.

\subsubsection*{Description:}

The functions construct a floating point number of appropriate
float type starting from the bit content of
\varname{bytes}.

If \varname{bytes} corresponds to the byte pattern of a \emph{NaN},
then a \code{NAN} is returned.

The actual float type returned by the functions is the widest one
supported by the implementation; i.e., a call to
\code{make-long-float} may return a \code{double-float}.

\subsubsection*{Exceptional Situations:}

The functions signals a \code{type-error} if \varname{bytes} is not of
the type described above

\subsubsection*{See Also:}

\code{NAN}, \code{make-float}.

\subsubsection*{Notes:}

These functions are, in one form or another, already present in \CL{}
implementations.

Implementations may supply a larger set of these functions, e.g.,
\code{make-quad-float}.


\DDictionaryItem{Value \code{NAN}}

\subsubsection*{Value:}

An \emph{implementation-dependent value}.


\subsubsection*{Description:}

The value \code{NAN} holds a representation of a ``not a number'' object.


\subsubsection*{Examples:}

The actual value of \code{NAN} varies from implementation to
implementation.  Here are two examples of how \code{NAN} can be
represented in two implementations:

\begin{alltt}
SBCL> \codeprompt{NAN}
\textit{#<DOUBLE-FLOAT quiet NaN>}
\textcolor{red}{;;; E.g., the result of (sb-kernel:make-double-float -524288 0)}
\end{alltt}

\begin{alltt}
LW> \codeprompt{NAN}
\textit{1D+-0} \textcolor{red}{#| 1D+-0 is double-float not-a-number |#}
\end{alltt}

\subsubsection*{Notes:}

\noindent
It is to be understood that testing for equality of two \code{NAN}s is
not meaningful.

\noindent
It is also understood that, \code{(numberp nan)} should return \code{T}.

\subsubsection*{See Also:}

\code{is-nan, nanp}


\DDictionaryItem{Functions \code{is-nan}, \code{nanp}}

\subsubsection*{Syntax:}

\code{is-nan} \varname{x} $\Rightarrow$ \textit{boolean}\\
\code{nanp} \varname{x} $\Rightarrow$ \textit{boolean}

\subsubsection*{Arguments and Values:}

\varname{x} -- any \CL{} object.

\subsubsection*{Description:}

The function \code{is-nan} (respectively \code{nanp}) returns \code{T}
whenever \varname{x} is a (representation of an IEEE) NaN.  Otherwise
it returns \code{NIL}.


\subsubsection*{Example:}

\begin{alltt}
CL prompt> \codeprompt{(is-nan nan)}
\textit{T}

CL prompt> \codeprompt{(nanp nan)}
\textit{T}

CL prompt> \codeprompt{(is-nan 42)}
\textit{NIL}

CL prompt> \codeprompt{(is-nan "NaN")}
\textit{NIL}
\end{alltt}



\DDictionaryItem{Constant Variables\\
  \code{long-float-positive-infinity},
  \code{long-float-negative-infinity},\\
  \code{double-float-positive-infinity},
  \code{double-float-negative-infinity},\\
  \code{single-float-positive-infinity},
  \code{single-float-negative-infinity},\\
  \code{short-float-positive-infinity},
  \code{short-float-negative-infinity},\\
  \code{+infL0}, 
  \code{-infL0},
  \code{+infD0}, 
  \code{-infD0},
  \code{+infF0}, 
  \code{-infF0},
  \code{+infS0}, 
  \code{-infS0},
  }

\subsubsection*{Value:}

The value of each of these constants is
\emph{implementation-dependent}.


\subsubsection*{Description:}

The value of each of these constants must conform with the format
(i.e., the floating point type) codified in the name.


\subsubsection*{Examples:}

\begin{alltt}
SBCL> \codeprompt{single-float-positive-infinity}
\textit{#.SB-EXT:SINGLE-FLOAT-POSITIVE-INFINITY}
\end{alltt}

\begin{alltt}
LW> \codeprompt{single-float-positive-infinity}
\textit{+1F++0} \textcolor{red}{#| +1F++0 is single-float plus-infinity |#}
\end{alltt}

\begin{alltt}
LW> \codeprompt{-infD0}
\textit{-1D++0} \textcolor{red}{#| -1D++0 is double-float plus-infinity |#}
\end{alltt}

\begin{alltt}
LW> \codeprompt{-infL0}
\textit{-1D++0} \textcolor{red}{#| -1D++0 is double-float plus-infinity |#}
\textcolor{red}{;;; Note that in this case the widest float representation
;;; available is DOUBLE-FLOAT.}
\end{alltt}


\subsubsection*{Notes:}

The ``short'' names and the ``long ones'' are completely equivalent.
A possible implementation would be the following:
\begin{alltt}
(define-symbol-macro -infD0 double-float-negative-infinity)
\end{alltt}


\DDictionaryItem{Functions \code{is-infinity}, \code{infinityp}}

\subsubsection*{Syntax:}

\code{is-infinity} \varname{x} $\Rightarrow$ \textit{boolean}\\
\code{infinityp} \varname{x} $\Rightarrow$ \textit{boolean}

\subsubsection*{Arguments and Values:}

\varname{x} -- any \CL{} object.

\subsubsection*{Description:}

The function \code{is-infinity} (respectively \code{infinityp}) returns \code{T}
whenever \varname{x} is a (representation of an IEEE) infinity.  Otherwise
it returns \code{NIL}.


\subsubsection*{Example:}

\begin{alltt}
CL prompt> \codeprompt{(is-infinity double-float-negative-infinity)}
\textit{T}

CL prompt> \codeprompt{(infinityp nan)}
\textit{NIL}

CL prompt> \codeprompt{(is-infinity 42)}
\textit{NIL}

CL prompt> \codeprompt{(is-infinity "NaN")}
\textit{NIL}
\end{alltt}



\subsection{Low Level Exception Handling}

The \CL{} ANSI Specification provides the following conditions that
may be raised by an implementation (in an \emph{implementation
  dependent} way) in conjunction with floating point operations.
\begin{description}
\item \code{floating-point-invalid-operation},
\item \code{floating-point-inexact},
\item \code{floating-point-overflow},
\item \code{floating-point-underflow}.
\end{description}
Of course, the condition \code{division-by-zero} may also be signaled by floating
point operations.


The following entries specify an interface similar the C Library
interface to the \emph{Floating Point Environment} provided by
\verb|<fenv.h>| \cite{C18}.


\DDictionaryItem{Type \code{fe-exception}}

\DDescription{}

The \code{fe-exception} type is defined to be:
\begin{alltt}
(member :divide-by-zero
        :inexact
        :invalid
        :overflow
        :underflow)
\end{alltt}
The meaning of these values correspond to a the possible exceptions
supported by an implementation.

\DSeeAlso{}

\code{+fe-all-exceptions+}



\DDictionaryItem{Constant Value \code{+fe-all-exceptions+}}

\subsubsection*{Value:}

An \emph{implementation-dependent value}.

\subsubsection*{Description:}

The value \code{+fe-all-exceptions+} holds a representation the set of
exceptions -- whose members are of type \code{fe-exception} -- that
are supported by the implementation.

\subsubsection*{Notes:}

A simple implementation could be a list, but this specification does
not impose such a constraint.


\subsubsection*{See Also:}

\code{fe-exception}, \code{all-supported-exceptions-p},
\code{some-supported-exceptions-p},\\
\code{supported-exception-p}.


\DDictionaryItem{Functions \code{supported-exception-p},\\
  \code{all-supported-exceptions-p},
  \code{some-supported-exceptions-p}}

\subsubsection*{Syntax:}

\code{supported-exception-p}
\varname{exception} $\Rightarrow$ \textit{boolean}\\
\code{all-supported-exceptions-p} \code{\&rest}
\varname{exceptions} $\Rightarrow$ \textit{boolean}\\
\code{some-supported-exceptions-p} \code{\&rest}
\varname{exceptions} $\Rightarrow$ \textit{boolean}

\subsubsection*{Arguments and Values:}

\varname{exception} -- a \CL{} object of type 
\code{fe-exception}.\\
\varname{exceptions} -- a list of \CL{} objects each of type 
\code{fe-exception}.


\subsubsection*{Description:}

The functions returns a true value if the exceptions passed as
arguments are supported by the implementation.

The function \code{supported-exception-p} returns true if 
\varname{exception} is supported by the implementation, and \code{NIL}
otherwise.

The function \code{all-supported-exceptions-p} returns true if all the
elements in \varname{exceptions} are supported by the implementation,
and \code{NIL} otherwise.

The function \code{some-supported-exceptions-p} returns true if any the
elements in \varname{exceptions} is supported by the implementation,
and \code{NIL} otherwise.

\DExamples{}

\begin{alltt}
CL prompt> \codeprompt{(supported-exception-p :divide-by-zero)}
\textit{T} \textcolor{red}{; Or could be be NIL.}

CL prompt> \codeprompt{(some-supported-exceptions-p :divide-by-zero :inexact)}
\textit{T} \textcolor{red}{; Assuming the previous operation returned true.}

CL prompt> \codeprompt{(all-supported-exception-p :divide-by-zero :inexact)}
\textit{NIL} \textcolor{red}{; Or could be be T.}
\end{alltt}

\subsubsection*{Exceptional Situations:}

The functions signal a \code{type-error} if \varname{exception}
is not of type \code{fe-exception} or if \varname{exceptions} contains
objects not of type \code{fe-exception}.









\subsection{Rounding}

Having control over IEEE rounding modes is necessary to implement a
number of numerical algorithms and data structures. The following
entries provide access to the IEEE facilities.


\DDictionaryItem{Type \code{rounding-modes}}

\subsubsection{Description:}

The \code{rounding-modes} type is defined to be:
\begin{alltt}
(member :indeterminable
        :zero
        :nearest
        :positive-infinity
        :negative-infinity)
\end{alltt}
The meaning of these values correspond to a direction of floating
point rounding.


\subsubsection*{Notes:}

The keywords used correspond to the C Library \cite{C18}
\textbf{\code{FLT\_ROUNDS}} values of:

\vspace*{3mm}

\begin{tabular}{rl}
  \textbf{\code{-1}} & indeterminable.\\
  \textbf{\code{0}}  & toward zero.\\
  \textbf{\code{1}}  & toward nearest.\\
  \textbf{\code{2}}  & toward positive infinity.\\
  \textbf{\code{3}}  & toward negative infinity.\\
\end{tabular}

\vspace*{3mm}

As per the C Library standard, \CL{} implementations can extend the
type \code{rounding-modes} with other keywords representing
\emph{implementation dependent} rounding modes.



\DDictionaryItem{Function \code{set-floating-point-modes}}

\subsubsection*{Syntax:}

\begin{tabbing}
\code{set-floating-point-modes} \= \textit{\code{\&key}}\\
\>\varname{traps}\\
\>\varname{rounding-mode}\\
\>\varname{current-exceptions}\\
\>\varname{precision}\\
\>\varname{\code{\&allow-other-keys}}\\
$\Rightarrow$ \varname{modes}
\end{tabbing}


\subsubsection*{Arguments and Values:}

\varname{traps} -- A list of the exception conditions that should cause
traps.\\
\varname{rounding-mode} -- The rounding mode to use when the result is
not exact.\\
\varname{current-exceptions} -- The argument is used to set the current
set of exceptions.\\
\varname{precision} -- An integer.\\
\varname{modes} -- An a-list containing the current floating
point modes; the indicators are keywords.

\subsubsection*{Description:}

This function sets options controlling the floating-point
hardware. If a keyword is not supplied, then the current value is
preserved.

The possible values for each of the keywords are the
following.

\begin{itemize}
\item \varname{traps} is a list that can contain the keywords
  \code{:underflow}, \code{:overflow}, \code{:inexact}, \code{:invalid},
  \code{:divide-by-zero}, and \code{:denormalized-operand}.

\item \varname{rounding-mode} is the rounding mode to use when the result is
  not exact; it can assume the values \code{:nearest},
  \code{:positive-infinity}, \code{:negative-infinity} and
  \code{:zero}.

\item \varname{current-exceptions} is used to set the exception flags. The
  main use is setting the accrued exceptions to \code{NIL} to clear
  them.

\item \varname{precision} can be one of the integers 24, 53 and 64, standing for
  the internal precision of the mantissa.
\end{itemize}

\subsubsection*{Examples:}

None.


\subsubsection*{Notes:}

None.


\subsubsection*{Exceptional Situations:}

The function can always result in a no-op if access to the underlying
hardware is not fully supported.  When this happens
\code{set-floating-point-modes} must issue a warning.


\subsubsection*{See also:}

\code{get-floating-point-modes}.


\DDictionaryItem{Function \code{get-floating-point-modes}}

\subsubsection*{Syntax:}

\code{get-floating-point-modes} \textit{$<$no arguments$>$}
$\Rightarrow$ \varname{modes}

\subsubsection*{Arguments and Values:}

\varname{modes} -- An a-list containing the current floating
point modes; the indicators are keywords.


\subsubsection*{Description:}

The function returns an a-list that represents the current state of
the floating point modes in use at the time.  The format of the
returned \varname{modes} a-list is such to be
usable as an \code{apply} last argument for
\code{set-floating-point-modes}.


\subsubsection*{Examples:}

\begin{alltt}
SBCL> \codeprompt{(get-floating-point-modes)}
\textit{(:TRAPS (:OVERFLOW :INVALID :DIVIDE-BY-ZERO)
 :ROUNDING-MODE :NEAREST
 :CURRENT-EXCEPTIONS (:INEXACT)
 :FAST-MODE NIL
 :PRECISION :53-BIT)}
\end{alltt}


\subsubsection*{See also:}

\code{set-floating-point-modes}.


\DDictionaryItem{Function \code{default-floating-point-modes}}

\subsubsection*{Syntax:}

\code{default-floating-point-modes} \textit{$<$no arguments$>$}
$\Rightarrow$ \varname{modes}

\subsubsection*{Arguments and Values:}

\varname{modes} -- An a-list containing the current floating
point modes; the indicators are keywords.

\subsubsection*{Description:}

The function returns an a-list that represents the \emph{default}
state of the floating point modes in use by the implementation.  The
format of the returned \varname{modes} a-list is such to be usable as
an \code{apply} last argument for \code{set-floating-point-modes}.

\subsubsection*{Examples:}

\begin{alltt}
SBCL> \codeprompt{(default-floating-point-modes)}
\textit{(:TRAPS (:OVERFLOW :INVALID :DIVIDE-BY-ZERO)
 :ROUNDING-MODE :NEAREST
 :CURRENT-EXCEPTIONS (:INEXACT)
 :FAST-MODE NIL
 :PRECISION :53-BIT)}
\end{alltt}

\subsubsection*{Notes:}

The function \code{default-floating-point-modes} should always return
the same (or \code{equal}) value.

\subsubsection*{See also:}

\code{set-floating-point-modes}, \code{get-floating-point-modes}.


\newpage
\DDictionaryItem{Macro \code{with-floating-point-modes}}

\DSyntax{}

% \code{with-floating-point-modes} (\textit{\code{\&key}}
% \varname{traps}
% \varname{rounding-mode}
% \varname{current-exceptions}
% \varname{precision}
% \varname{\code{\&allow-other-keys}}) \varname{\code{\&body}} \varname{body}
% $\Rightarrow$ \varname{results}

\begin{tabbing}
\code{with-floating-point-modes} \=(\=\textit{\code{\&key}}\\
\>\>                                \varname{traps}\\
\>\>                                \varname{rounding-mode}\\
\>\>                                \varname{current-exceptions}\\
\>\>                                \varname{precision}\\
\>\>                                \varname{\code{\&allow-other-keys}})\\
\>                           \varname{\code{\&body}} \varname{body}\\
$\Rightarrow$ \varname{results}
\end{tabbing}




\subsubsection*{Arguments and Values:}

\varname{traps} -- A list of the exception conditions that should cause
traps.\\
\varname{rounding-mode} -- The rounding mode to use when the result is
not exact.\\
\varname{current-exceptions} -- The argument is used to set the current
set of exceptions.\\
\varname{precision} -- An integer.\\
\varname{results} -- One or more \CL{} objects.


\subsubsection*{Description:}

The \code{with-floating-point-modes} macro executes \varname{body} in a
an environment where the floating point modes are determined by the
values passed as arguments to the macro.  Upon termination (either
normal or exceptional) of the code in \varname{body} the floating
point modes are restored to those in effect before the execution of
\code{with-floating-point-modes}.

As for \code{set-floating-point-modes} the values that the arguments
can take are the following:

\begin{itemize}
\item \varname{traps} is a list that can contain the keywords
  \code{:underflow}, \code{:overflow}, \code{:inexact}, \code{:invalid},
  \code{:divide-by-zero}, and \code{:denormalized-operand}.

\item \varname{rounding-mode} is the rounding mode to use when the result is
  not exact; it can assume the values \code{:nearest},
  \code{:positive-infinity}, \code{:negative-infinity} and
  \code{:zero}.

\item \varname{current-exceptions} is used to set the exception flags. The
  main use is setting the accrued exceptions to \code{NIL} to clear
  them.

\item \varname{precision} can be one of the integers 24, 53 and 64,
  standing for the internal precision of the mantissa.
\end{itemize}



\noindent
\varname{results} is the value (or values) returned by \varname{body}.


\subsubsection*{Notes:}

When called with an empty arguments list,
\code{with-floating-point-modes} is a no-op.


\subsubsection*{See Also:}

\code{get-floating-point-modes}, \code{set-floating-point-modes}.


\DDictionaryItem{Function \code{get-rounding-mode}}

\subsubsection*{Syntax:}

\code{get-rounding-mode} \textit{$<$no arguments$>$}
$\Rightarrow$ \varname{rounding-mode}

\DArgsNValues{}

\varname{rounding-mode} -- a value of type \code{rounding-modes}.

\DDescription{}

The function returns the current rounding mode.  The value
\code{:indeterminate} is returned if such rounding mode cannot be
determined.

\DSeeAlso{}

 \code{rounding-modes}.


\DDictionaryItem{Function \code{set-rounding-mode}}

\subsubsection*{Syntax:}

\code{set-rounding-mode} \varname{rounding-mode}
$\Rightarrow$ \varname{result}, \varname{success}

\DArgsNValues{}

\varname{rounding-mode} -- a \code{rounding-mode}\\
\varname{result} -- a \code{rounding-mode}\\
\varname{success} -- a boolean


\DDescription{}

The function sets the rounding mode.  If the setting of the rounding
mode is succesful, then \varname{rounding-node} is returned as
\varname{result} and \varname{success} is \code{T}.  Otherwise, the
rounding mode before the the call is returned with \varname{success}
\code{NIL}.\marginnote{Should it instead signal an error?}




\begin{thebibliography}{9}

\bibitem{IEEE-754}
  \textit{{IEEE} Standard for Floating-Point Arithmetic}, IEEE Std
  754$^{\mathrm{TM}}$-2008, IEEE Computer Society, 2008.
  
\bibitem{C18}
  \textit{Programming languages -- C},
  International Standard \emph{ISO/IEC 9899-2018}, 2018.
  
\bibitem{ANSIHyperSpec}
  \textit{The \CL{} Hyperspec},
  published online at\\
  \texttt{http://www.lisp.org/HyperSpec/FrontMatter/index.html}, 1994.

\end{thebibliography}


\appendix

\section{Copying and License}

This work may be distributed and/or modified under the conditions of
the \emph{LaTeX Project Public License} (LPPL), either version 1.3 of this license
or (at your option) any later version. The latest version of this
license is in \texttt{http://www.latex-project.org/lppl.txt} and version 1.3 or
later is part of all distributions of LaTeX version 2005/12/01 or
later.

\noindent
This work has the LPPL maintenance status `maintained'.

\noindent
The Current Maintainer of this work is Marco Antoniotti.

\end{document}

%%%% end of file --CDR-IEEE-754-support.tex --
