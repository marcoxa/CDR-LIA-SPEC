%%%% -*- Mode: LaTeX -*-

%%%% CDR-IEEE-754-support.tex --
%%%% Minimalistic support for (a subset) of IEEE 754 for Common Lisp.


\documentclass[10pt,fleqn]{article}

\usepackage{latexsym}
\usepackage{epsfig}
\usepackage{alltt}
\usepackage{tabulary}
\usepackage{color}

\usepackage[margin=3cm]{geometry}

% \usepackage[linktocpage=true]{hyperref} % Causes some problems.
  \usepackage[depth=4]{bookmark}

\usepackage{makeidx}

% \usepackage[OT1]{fontenc}
% \usepackage{bold-extra}
% \renewcommand{\ttdefault}{cmt}

\usepackage{lmodern}            % This works for decent bold 'tt'.

\usepackage{subfiles}


\newcommand{\tm}{$^\mathsf{tm}$}
\newcommand{\cfr}{\emph{cfr.}}

\newcommand{\Lisp}{\textsf{Lisp}}
\newcommand{\CL}{\textsf{Common~Lisp}}

\newcommand{\CMUCL}{\textsf{CMUCL}}
\newcommand{\SBCL}{\textsf{SBCL}}
\newcommand{\CCL}{\textsf{CCL}}
\newcommand{\ACL}{\textsf{ACL}}
\newcommand{\ABCL}{\textsf{ABCL}}


\newcommand{\Quicklisp}{\textsf{Quicklisp}}

\newcommand{\CLang}{\textsf{C}}
\newcommand{\Java}{\textsc{\textsf{Java}}}
\newcommand{\Fortran}{\textsc{\textsf{Fortran}}}

\newcommand{\checkcite}[1]{{\textbf{[Missing Citation: #1]}}}
\newcommand{\checkref}[1]{{\textbf{[Missing Reference: #1]}}}

\newcommand{\missingpart}[1]{{\ }\vspace{2mm}\\
{\textbf{[Still Missing: #1]}}\\
\vspace{2mm}}

\newcommand{\marginnote}[1]{%
\marginpar{\begin{small}\begin{em}
{\raggedright #1}
\end{em}\end{small}}}

\newcommand{\undecided}[2]{%
  \vspace*{3mm}\noindent\fbox{\parbox{\textwidth}{\textbf{TBD: #1}{\ }\newline\emph{#2}}}}



%%% CL 

\newcommand{\code}[1]{\texttt{#1}}

\newcommand{\term}[1]{\texttt{#1}}
\newcommand{\nonterm}[1]{\textit{$<$#1$>$}}


\newcommand{\kwd}[1]{\texttt{:#1}}

\newcommand{\clieeeterm}[1]{\textit{#1}}
\newcommand{\clliaterm}[1]{\textit{#1}}

\newcommand{\varname}[1]{\textit{#1}}

\newcommand{\clterm}[1]{\textit{#1}}
\newcommand{\clname}[1]{\texttt{#1}}

\newcommand{\codeprompt}[1]{\textcolor{blue}{\textbf{#1}}}
\newcommand{\codelia}[1]{\textcolor{blue}{#1}}

\newcommand{\RArrow}{$\Rightarrow$}



%%% Useful Names.

\newcommand{\IEEEFPStd}{IEEE-754}
\newcommand{\IECFPStd}{IEC-60559}
% \newcommand{\IECLIA123}{ISO/IEC-10967-1,2,3}
\newcommand{\IECLIA}{ISO/IEC-10967}



%%% ANSI-Spec Like Macros.

\newcommand{\ANSICL}{\textsf{ANSI~CL}}

% \newcommand{\DDictionaryItem}[1]{\subsection{#1}\vspace*{-9pt}\hrulefill}
\newcommand{\DDictionaryItem}[1]{\vspace*{6pt}\noindent\hrulefill\vspace*{-9pt}\subsection*{#1}}

\newcommand{\DSyntax}{\subsubsection*{Syntax:}}
\newcommand{\DSupertypes}{\subsubsection*{Supertypes:}}
\newcommand{\DArgsNValues}{\subsubsection*{Arguments and Values:}}
\newcommand{\DDescription}{\subsubsection*{Description:}}
\newcommand{\DExamples}{\subsubsection*{Examples:}}
\newcommand{\DExceptional}{\subsubsection*{Exceptional Situations:}}
\newcommand{\DNotes}{\subsubsection*{Notes:}}
\newcommand{\DSeeAlso}{\subsubsection*{See Also:}}
\newcommand{\DValues}{\subsubsection*{Values:}}


%%% Spec special names.

\newcommand{\CLLIAPKG}{\code{CL-MATH-LIA-2020}}




%%%% Document Title.


\title{
\LARGE{\bfseries A Palimpsest of ``Language Independent Arithmetic'' in \CL{}}}

\author{
  Marco Antoniotti\\
  Dipartimento di Informatica, Sistemistica e Comunicazione\\
  Universit\`{a} degli Studi di Milano Bicocca\\
  Viale Sarca 336, U14, Milan (MI), \textsc{Italy}\\[2mm]
  \texttt{marco.antoniotti} at \texttt{unimib.it},\\
  \texttt{mantoniotti} at \texttt{common-lisp.net}}


% \author{TBD}

%\date{}


%\includeonly{}


\makeindex


\begin{document}

\maketitle


\begin{abstract}
  This document presents a set of names, functions and macros that aim
  at providing a common ground for the support of the \IEEEFPStd{}
  \cite{2008:IEEE-754} and of the \emph{Language Independent
    Arithmetic} (LIA) standards \cite{2012:LIA1,2001:LIA2,2004:LIA3}
  in \CL{}.

  The set of names and functions (and ancillary information) is
  intended to map almost one to one with the LIA specification; extra
  \CL{} related functionalities are also introduced to give a
  programmer the maximum flexibility, while allowing the writing of
  programs in a straightforward way if so desired.
\end{abstract}

\newpage

\tableofcontents

\newpage

\section{Introduction}
\subfile{sections/Introduction/Introduction.tex}
\newpage

\section{Description}
\subfile{sections/Description/Description.tex}
\newpage

\section{LIA \CL{}-related Dictionary}
\subfile{sections/Dictionary/Dictionary.tex}
\newpage

\section{Floating Point \IEEEFPStd{} Respecting Operations}
\subfile{sections/Floating-Point-Operations/Floating-Point-Operations.tex}
\newpage

\nocite{2012:LIA1,2001:LIA2,2004:LIA3}
\nocite{1994:ANSICL}

\bibliographystyle{plain}
\bibliography{CDR-CL-LIA}

% \begin{thebibliography}{9}

% \bibitem{IEEE-754}
%   \textit{{IEEE} Standard for Floating-Point Arithmetic}, IEEE Std
%   754$^{\mathrm{tm}}$-2008, IEEE Computer Society, 2008.
  
% \bibitem{C18}
%   \textit{Programming languages -- C},
%   International Standard \emph{ISO/IEC 9899-2018}, 2018.
  
% \bibitem{ANSIHyperSpec}
%   \textit{The \CL{} Hyperspec},
%   published online at\\
%   \texttt{http://www.lisp.org/HyperSpec/FrontMatter/index.html}, 1994.

% \end{thebibliography}


\appendix

\section{Example Programs}
\subfile{sections/Appendix-Example-Program/Example-Program.tex}

\section{Copying and License}
\subfile{sections/License/License.tex}

\printindex

\end{document}

%%%% end of file --CDR-IEEE-754-support.tex --
